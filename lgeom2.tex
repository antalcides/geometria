\documentclass[11pt,spanish,letter]{book}
%%%%%%%%%%%%%%%%%%%%%%%%%%% paquetes %%%%%%%%%%%%%%%%%%%%%%%%%%%%%%%%%%
\usepackage{calc}
\usepackage{chngcntr}%para referenciar cualquier entorno
\usepackage[notref,color]{showkeys} %para mostrar etiquetas de las referencias
\usepackage{pifont}
\usepackage{tikz}
\usetikzlibrary{shapes,snakes}
\usepackage{pgf}
\usetikzlibrary{calc}
\usetikzlibrary{arrows}
\usepackage{epstopdf}
%\usepackage{floatflt}
\usepackage{multirow}
\usepackage{bigstrut}
\usepackage{type1ec}       % Tildes en el pdf
\usepackage{times}          % Usar tipo Times-Roman
\usepackage[T1]{fontenc}    % Usar la codificación T1
%\usepackage[spanish]{babel}
%\usepackage{gnuplot-lua-tikz}
%\usepackage{maxiplot}
%\usepackage{gnuplottex}
\usepackage{pgfplots}
\usepackage{amsmath,amssymb}
%\usepackage{helvet}
\usepackage[T1]{fontenc}
%\geometry{verbose}
\usepackage{booktabs}
\usepackage{courier}
\usepackage{units}
\usepackage{url}
\usepackage{float}
\usepackage{xcolor}
\usepackage{mathpazo}
\usepackage{amsfonts}
\usepackage{fancyvrb}
\usepackage{multicol}
\usepackage[utf8x]{inputenc}
%\usepackage[latin1]{inputenc}
\usepackage[calcwidth]{titlesec}
\usepackage{graphicx}
\usepackage[oneline,flushleft,footnotesize]{caption2}
\usepackage{enumerate}
\usepackage{comment}
\usepackage{ifthen}
\usepackage{cancel}
\usepackage{layout}
\usepackage{footnote}
\usepackage{hyperref}
%%%%%%%%%%%%%%%%%%%%%%%%%%%%%%%%%%%%% Fancy %%%%%%%%%%%%%%%%%%%%%%%%%%%%%%%%%%%
%$$$$$$$$$$$$$$$$$$$$$$$$$$$$$$$$$$$$$$$$$$$$$$$$$$$$$$$$$$$$$$$$$$$$$$$$$$$$
%%%%%%%%%%%%%%%%%%%%%Página par en blanco al final del capítulo%%%%%%%%%%%%%%%%%%%
%%%%%%%%%%%%%%%%%%%%%%%%%%%%%% Abre Introducción de tipos  @  %%%%%%%%%%%
 \makeatletter
\def\clearpage{\ifvmode
\ifnum \@dbltopnum =\m@ne \ifdim \pagetotal <\topskip \hbox{} \fi
\fi \fi
\newpage
\thispagestyle{empty} \write\m@ne{} \vbox{} \penalty -\@Mi }
%%%%%%%%%%%%%%%%%%%%%%%%%%%%%
\def\@endchapter{\vfil\newpage
              \if@twoside
               \if@openright
                \null
                \thispagestyle{empty}%
                \newpage
               \fi
              \fi
              \if@tempswa
                \twocolumn
              \fi}
%%%%%%%%%%%%%%%%%%%%%%%%%%%%%%%%%%%%%%%%%%%%%%%%%%%%%%%%%%%%%%%%%%%%%%%%%%%%%%%

%%%%%%%%%%%%%%%%%%%%%%%%%%%%%%% Definicion de subsecciones%%%%%%%%%%%%%%%%%%%%%%
%%%%%%%%%%%%%%%%%%%%%%%%%%%%%%%%%%%%%%%%%%%%%%%%%%%%%%%%%%%%%%%%%%%%%%%%%%%%%%%%%%%%%%%%%%%%%%%%%%%%%%%%%%%%
\newcommand{\blackline}{{\setlength{\parindent}{0mm}\color{black}\rule{\linewidth}{0.3mm}}}
\newcommand{\blueline}{{\setlength{\parindent}{0mm}\color{blue}\rule{\linewidth}{1mm}}}
\newcommand{\greenline}{{\setlength{\parindent}{0mm}\color{green}\rule{\linewidth}{1mm}}}
% \renewcommand\section{\@startsection {section}{1}{-10pt}%
%                                    {3.5ex \@plus 1ex \@minus .2ex
%                                    }%
%                                    {2.3ex\@plus .2ex
%                                    \vskip 0.2\p@ \blackline \vskip 1pt}%
%                                    {\centering\sc\normalsize\color{black}}}

\renewcommand\subsection{\@startsection {subsection}{1}{-5pt}%
                                   {3.5ex \@plus 1ex \@minus .2ex
                                   }%
                                   {2.3ex\@plus .2ex
                                   \vskip 0.2\p@ \vskip 1pt}%
                                   {\centering\sc\large\color{blue}}}

\renewcommand\subsubsection{\@startsection {subsubsection}{1}{-5pt}%
                                   {3.5ex \@plus 1ex \@minus .2ex
                                   }%
                                   {2.3ex\@plus .2ex
                                   \vskip 0.2\p@ \vskip 1pt}%
                                   {\centering\sc\large\color{red}}}

%%%%%%%%%%%%%%%%%%%%%%%%%%%%%%%%%%%%%%%%%%%%%%%%%%%%%%%%%%%%%%%%%%%%%%%%%%%%%%%%%%%%%%%%%%%%%%%%%%%%%%%%%%%%&&%%
\setlength{\parindent}{0pt}
%%%%%%%%%%%%%tabla de contenido%%%%%%%%%%%%
  \newif\if@toc \@tocfalse
\renewcommand\tableofcontents{ \if@twocolumn
\@restonecoltrue\onecolumn \else \@restonecolfalse \fi
\begingroup                     \@toctrue
\chapter*{\contentsname}
\endgroup
\pagestyle{empty} \baselineskip=18pt plus .5pt minus .5pt
{\raggedleft P\'{a}gina \par\vskip-\parskip} \@starttoc{toc}
\newpage
\if@restonecol\twocolumn\fi \baselineskip=\normalbaselineskip
}\let\par \par
 %%%%%%%%%%%%%%%%%%%%%%%%%%%%%%%%%%%%%%%%Tipos en caption%%%%%%%%%%%%%%%%%
\renewcommand{\captionfont}{\rmfamily}
\renewcommand{\captionlabelfont}{\footnotesize\bfseries \sffamily}
\renewcommand\footnoterule{  \kern-3\p@
\hrule\@width.2\columnwidth \kern2.6\p@}
 %%%%%%%%%%%%%%%%%%%%%%%%%%%%%%%%%%%%%%%%%%%%%%%Encajonado%%%%%%%%%%%%%%%%%%%%%
% \setlength{\marginparwidth}{5cm} \setlength{\marginparsep}{0.5cm}
% \setlength{\textheight}{22 cm} \setlength{\textwidth}{15cm}
% \oddsidemargin -1cm
% \evensidemargin -1cm
% \topmargin-1 cm
% \headsep+22pt
% \headheight9pt
% \hoffset -0.8cm
% \voffset -1.8cm
% \footskip 1cm
\usepackage[paperheight=25cm,%
paperwidth=19cm,%
centering,%
left=1.5cm,%
right=2cm,%
top=2.5cm,%
bottom=1.5cm,%
headheight=0.5cm,%
headsep=10pt,%
%footskip=1cm,%
marginparsep=20pt,
pdftex=false
]{geometry}
\usepackage[frame,center,letter,pdflatex,cam]{crop}
%%%%%%%%%%%%%%%%%%%%%%%%%%%%%%%%%%%%%%%%%%%%%%%%%% %%%%%%%%%%%%%%%%%%%%%%%%%%%%%%%%%%%%%%%%%%%%%%%

%%%%%%%%%%%%%%%%%%%%%%%%%%%titulo de secciones%%%%%%%%%%%%%%%%%%%%%%
\def\indexname{\'{\I}ndice}
\def\contentsname{ Contenido}
\def\listfigurename{Tabla de figuras}
\def\bibname{\Large Bibliograf\'{\i}a}
\def\tablename{Tabla}
\def\proofname{Demostraci\'on}
\def\appendixname{Ap\'endice}
\def\chaptername{ Cap\'{\i}tulo}
\def\figurename{Figura}
\renewcommand{\tablename}{Tabla}
\renewcommand{\listtablename}{\'{\I}ndice de Tablas}
\newdimen\myboxnot
%%%%%%%%%%%%%%%%%%%%%%%%%%%%%%%%%%%definicion de capitulos%%%%%%%%%%%%%%%%%%%%%
\newcommand{\nomproprebib}[2]{#1 \textsc{#2}}
\newcommand{\entreebiblivre}[4]{#1, \emph{#2}, #3, #4.}
\newcommand{\entreebibarticle}[7]{#1, \og #2\fg{}, \emph{#3} \textbf{#4} (#5), p.~#6-#7.}
\newcommand{\ifempty}[3]{\ifx#1\empty#2\else#3\fi}
\newcommand{\thechapterimage}{}
\newcommand{\chapterimage}[1]{\renewcommand*{\thechapterimage}{#1}}
\newcommand{\chapterheadfont}{\fontfamily{phv}\fontsize{20pt}{24pt}\selectfont}
\newcommand{\sectionheadfont}{\fontfamily{phv}\fontsize{18pt}{20pt}\selectfont}
% \newcommand{\chapimage}{\begin{figure*}[H]
%                                                 \centering
% \includegraphics[width=1.2\paperwidth]{\thechapterimage}
%                        \end{figure*}
% }
\newcommand{\newchaptercmd}[1]{%
 \begin{figure}[H]
 %\centering
\begin{tikzpicture}[overlay,remember picture]
\begin{scope}
\clip (current page.north west) rectangle ($ (current page.north east) + (5cm,-12cm)$);
\fill[blue!20] (current page.north west) rectangle
($ (current page.north east) + (5cm,-12cm)$);
\node[below=-0.35cm] at (current page.north)
{\ifempty{\thechapterimage}{}{\centering\includegraphics[width=2\paperwidth]{\thechapterimage}}};
\end{scope}
\node[right=2cm,rectangle,draw,color=white,fill=white,opacity=0.7,inner ysep=10pt,
inner xsep=15pt,rounded corners=0.75cm] at ($ (current page.north west) + (0cm,-7cm)$)
{\chapterheadfont\vphantom{pb}\phantom{#1}\hspace{\paperwidth}\null};
\node[right=2cm,inner ysep=10pt,inner xsep=15pt] at ($ (current page.north west) + (0cm,-7cm)$)
{\chapterheadfont\vphantom{pb}\color{red!60!black}\chapterlabel#1};
\end{tikzpicture}\end{figure}
}
\titleformat{\chapter}
{\gdef\chapterlabel{}}
{\gdef\chapterlabel{\thechapter.\ }}
{0pt}
{\newchaptercmd}
{}
\titlespacing*{\chapter}{0pt}{0pt}{8cm}
%%%%%%%%%%%%%%%%%%%%%%%%%%%%%%%%%%%%%%%%%%%%%%%%%%%%%%%%%%%%%%%%%%%%%%%%%%%%%%
%%%%%%%%%%%%%%%%%%%%%%%%%%%%%%%%%5 definicion seccion %%%%%%%%%%%%%%%%%%%%%%%%%
\newcommand{\newsectioncmd}[1]{ %\vskip 5pt
\hspace*{-25pt}
        \tikzstyle{mybox} = [ draw=blue, fill=red!10, very thick,
    rectangle,opacity=0.7,inner ysep=10pt,inner xsep=15pt,rounded corners=15pt]
         \tikzstyle{fancytitle} =[fill=blue, text=white]
                        \begin{tikzpicture}
                 \node [mybox] (box){ \begin{minipage}[r][15pt]{0.75\paperwidth}

\chapterheadfont\vphantom{pb}\phantom{#1}\hspace{0.7\paperwidth}\null
                                \end{minipage}};
                             \node[right=-5cm,inner ysep=10pt,inner xsep=15pt]
at (-3cm,0cm)
{\sectionheadfont\vphantom{pb}\color{red!60!black}\thesection \, #1};
                        \end{tikzpicture}


\vskip -10pt
}
\titleformat{\section}
{\gdef\sectionlabel{}}
{\gdef\sectionlabel{\thesection.\ }}
{0pt}
{\newsectioncmd}
{}
\titlespacing*{\section}{12pt}{2cm}{20pt}
%%%%%%%%%%%%%%%%%%%%%%%%%%%%%%%%%%%%%%%%%%%%%%%%%%%%%%%%%%%%%%%%%%%%%%%%%
%%%%%%%%%%%%%%%%%%%%%%%%%%%%%%%%%%%%%Encabezado %%%%%%%%%%%%%%%%%%%
\newpagestyle{estilonew}[\small\sffamily]{\headrule
\sethead[\thesection. \ \sectiontitle][][A.Olivo]{Geometr\'{\i}a con regla y comp\'as}{}{}
\footrule\setfoot{}{\usepage}{}}
\pagestyle{estilonew}
\renewcommand{\makeheadrule}{%
\makebox[0pt][l]{\rule[.9\baselineskip]{\linewidth}{0.8pt}}%
\rule[-.4\baselineskip]{\linewidth}{0.8pt}}
%%%%%%%%%%%%%%%%%%%%%%%%%%%%%%%%%%% Entornos %%%%%%%%%%%%%%%%%%%%%%%%%%%%%%%
%$$$$$$$$$$$$$$$$$$$$$$$$$$$$$$$$$$$$$$$$$$$$$$$$$$$$$$$$$$$$$$$$$$$$$$$$$
 %%%%%%%%%%%%%%%%%%%%%%%%%%%% Lista %%%%%%%%%%%%%%%%%%%%%%%%%%%%%%%%%%%%%
 %%%%%%%%%%%%%%%%%%%%%%%%%%%%%%%%%%%Lista    %%%%%%%%%%%%%%%%%%%%%%%%%%%%%
\newenvironment{lista}{
\begin{itemize}
 \renewcommand{\labelitemi}{{
 \colorbox{blue!90!black}{\color{white}{\ding{42}}}
 }}
}{\end{itemize}}
%%%%%%%%%%%%%%%%%%%%%%%%%%%%%%%%%%%%%%%%%%%%%%%%%%%%%%%%%%%%%%%%%%%
%%%%%%%%%%%%%%%%%%%%%%%%%%%%%%%%%%%%%%Termino  Ideas  %%%%%%%%%%%%%%%%%%%%%%%%%%%%%%
\newcounter{ideas}[section]
\newenvironment{ideas}[2]{\vskip 5pt
\refstepcounter{ideas}
   \begin{tikzpicture}
\tikzstyle{mybox} = [draw=blue, fill=yellow!20, very thick,    
    rectangle, rounded corners, inner sep=20pt, inner ysep=20pt]   
\tikzstyle{fancytitle} =[fill=blue, text=white, ellipse]
 %\hspace*{-15pt}
   \node [mybox] (box){ %
 \begin{minipage}[r]{0.9\textwidth}
                                     #1
                                          \end{minipage}};
\node[fancytitle, right=10pt] at (box.north west) {T\'ermino.\theideas.\
\bf{#2}};
\node[fancytitle, rounded corners] at (box.east) {$\clubsuit$};  
 \end{tikzpicture} 
          \hspace*{10pt}                }
                              {\vskip 5pt }%
\counterwithin{ideas}{section}
%%%%%%%%%%%%%%%%%%%%%%%%%%%%%%%%%%%%%%%%%%%% Ejemplo%%%%%%%%%%%%%%%%%%%%
\newcounter{ejemplo}[section]% para ligar el contador con la seccion.
\newenvironment{ejemplo}[1]{\vskip 5pt
\refstepcounter{ejemplo}%para adicionar un valor+ al contador y poder referenciarlo
\begin{tikzpicture}
\tikzstyle{mybox} = [draw=red, fill=blue!20, very thick,
    rectangle, rounded corners, inner sep=10pt, inner ysep=20pt]
\tikzstyle{fancytitle} =[fill=red, text=black]
%\hspace*{-10pt}
\node [mybox] (box){%
     \begin{minipage}[r]{0.95\textwidth} 
                                     {#1}
                                 \end{minipage}};
 \node[fancytitle, right=10pt] at (box.north west) {Ejemplo. \theejemplo\hspace*{50pt}};
 \node[fancytitle, rounded corners] at (box.east) {$\blacktriangle$};                    
      \end{tikzpicture} 
                        \hspace*{10pt}         }{
 
                \vskip 5pt
 }%\greenline \vskip 0.1pt}
\counterwithin{ejemplo}{section} %Para que en la referencia aparezca el counter Chapter.section.ejemplo 
%%%%%%%%%%%%%%%%%%%%%%%%%%%%%%%%%%%%%% Axioma %%%%%%%%%%%%%%%%%%%%%%%%%%%%%%
\newcounter{axioma}[section]
\newenvironment{axioma}[2]{\vskip 5pt
\refstepcounter{axioma}
  \begin{tikzpicture}
\tikzstyle{mybox} = [draw=blue, fill=red!10, very thick,
    rectangle, rounded corners, inner sep=10pt, inner ysep=20pt]
\tikzstyle{fancytitle} =[fill=blue, text=white]
\node [mybox] (box){ %
     \begin{minipage}[r]{\textwidth}
                                    {#1}
                                 \end{minipage}};
 \node[fancytitle, right=10pt] at (box.north west)
{Postulado.\theaxioma. \ \bf{#2}};
 \node[fancytitle, rounded corners] at (box.east) {$\clubsuit$};
     \end{tikzpicture}
              \hspace*{10pt}                }{

                \vskip 5pt
 }%\greenline \vskip 0.1pt}
\counterwithin{axioma}{section}
%%%%%%%%%%%%%%%%%%%%%%%%%%%%%%%%%%%%%%%%%%%%%%%%%%%%%%%%%%%%%%%%%%%%
%%%%%%%%%%%%%%%%%%%%%%%%%%%%%%%%%%%%%% postulado %%%%%%%%%%%%%%%%%%%%%%%%%%%%%%
\newcounter{postulado}[section]
\newenvironment{postulado}[2]{\vskip 5pt
\refstepcounter{postulado}
  \begin{tikzpicture}
\tikzstyle{mybox} = [draw=blue, fill=red!10, very thick,
    rectangle, rounded corners, inner sep=10pt, inner ysep=20pt]
\tikzstyle{fancytitle} =[fill=blue, text=white]
\node [mybox] (box){ %
     \begin{minipage}[r]{\textwidth} 
                                    {#1}
                                 \end{minipage}};
 \node[fancytitle, right=10pt] at (box.north west)
{Postulado.\thepostulado. \ \bf{#2}};
 \node[fancytitle, rounded corners] at (box.east) {$\clubsuit$};                    
     \end{tikzpicture} 
              \hspace*{10pt}                }{
 
                \vskip 5pt 
 }%\greenline \vskip 0.1pt}
\counterwithin{postulado}{section}
%%%%%%%%%%%%%%%%%%%%%%%%%%%%%%%%%%%%%%%%%%%%%%%%%%%%%%%%%%%%%%%%%%%%
%%%%%%%%%%%%%%%%%%%%%%%%%%%%%%%%%%%%%% Conjetura %%%%%%%%%%%%%%%%%%%%%%%%%%%%%%
\newcounter{conjetura}[section]
\newenvironment{conjetura}[2]{\vskip 5pt
\refstepcounter{conjetura}
  \begin{tikzpicture}
\tikzstyle{mybox} = [draw=blue, fill=green!20, very thick,
    rectangle, rounded corners, inner sep=20pt, inner ysep=20pt]
\tikzstyle{fancytitle} =[fill=blue, text=white, ellipse]
\node [mybox] (box){ %
     \begin{minipage}[r]{0.9\textwidth} 
                                    {#1}
                                 \end{minipage}};
 \node[fancytitle, right=10pt] at (box.north west) {Conjetura.
\theconjetura. \ \bf{#2}};
 \node[fancytitle, rounded corners] at (box.east) {$\clubsuit$};                    
     \end{tikzpicture} 
              \hspace*{10pt}                }{
 
                \vskip 5pt 
 }%\greenline \vskip 0.1pt}
\counterwithin{conjetura}{section}
%%%%%%%%%%%%%%%%%%%%%%%%%%%%%%%%%%%%%%%%%%%%%%%%%%%%%%%%%%%%%%%%%%%%
%%%%%%%%%%%%%%%%%%%%%%%%%%%%%%%%%%%%%% Definicion %%%%%%%%%%%%%%%%%%%%%%%%%%%%%%
\newcounter{definicion}[section]
\newenvironment{definicion}[2]{\vskip 5pt
\refstepcounter{definicion}
  \begin{tikzpicture}
\tikzstyle{mybox} = [draw=blue, fill=yellow!20, very thick,
    rectangle, rounded corners, inner sep=20pt, inner ysep=20pt]
\tikzstyle{fancytitle} =[fill=red!60, text=white, ellipse]
\stepcounter{definicion}
\hspace*{-15pt}
\node [mybox] (box){ %
     \begin{minipage}[r]{\textwidth} 
                                    {#1}
                                 \end{minipage}};
 \node[fancytitle, right=10pt] at (box.north west) {Definici\'on.
\thedefinicion. \ \bf{#2}};
 \node[fancytitle, rounded corners] at (box.east) {$\heartsuit$};                    
     \end{tikzpicture} 
              \hspace*{10pt}                }{
 
                \vskip 5pt 
 }%\greenline \vskip 0.1pt}
\counterwithin{definicion}{section}
%%%%%%%%%%%%%%%%%%%%%%%%%%%%%%%%%%%%%%%%%%%%%%%%%%%%%%%%%%%%%%%%%%%%
%%%%%%%%%%%%%%%%%%%%%%%%%%%%%%%%%%%%%% Teorema %%%%%%%%%%%%%%%%%%%%%%%%%%%%%%
\newcounter{teorema}[chapter]
\newenvironment{teorema}[2]{\vskip 5pt
  \begin{tikzpicture}
\refstepcounter{teorema}
\tikzstyle{mybox} = [draw=blue, fill=red!10, very thick,
    rectangle, rounded corners, inner sep=10pt, inner ysep=20pt]
\tikzstyle{fancytitle} =[fill=blue, text=white, ellipse]
\stepcounter{teorema}
\hspace*{-15pt}
\node [mybox] (box){ %
     \begin{minipage}[r]{\textwidth}
                                    {#1}
                                 \end{minipage} };
 \node[fancytitle, right=10pt] at (box.north west) {Teorema. \theteorema.  \ \bf{#2}};
 \node[fancytitle, rounded corners] at (box.east) {$\clubsuit$};                    
     \end{tikzpicture} 
              \hspace*{10pt}                }{
 
                \vskip 5pt 
 }%\greenline \vskip 0.1pt}
\counterwithin{teorema}{chapter}
%%%%%%%%%%%%%%%%%%%%%%%%%%%%%%%%%%%%%%%%%%%%%%%%%%%%%%%%%%%%%%%%%%%%
%%%%%%%%%%%%%%%%%%%%%%%%%%%%%%%%%%%%%%%%%%%%%%%%%% construccion%%%%%%%%%%%%%%%%%%%%
\newcounter{construccion}[section]
\newenvironment{construccion}[2]{\vskip 5pt
\refstepcounter{construccion}
\begin{tikzpicture}
\tikzstyle{mybox} = [draw=red, fill=blue!20, very thick,
    rectangle, rounded corners, inner sep=10pt, inner ysep=20pt]
\tikzstyle{fancytitle} =[fill=red, text=black]
\node [mybox] (box){%
     \begin{minipage}[r]{\textwidth}
                                     {#1}
                                 \end{minipage}};
 \node[fancytitle, right=10pt] at (box.north west) {Construcci\'on.
\theconstruccion \ \bf{#2}};
 \node[fancytitle, rounded corners] at (box.east) {$\bigstar$};
      \end{tikzpicture}
                        \hspace*{10pt}         }{

                \vskip 5pt
 }%\greenline \vskip 0.1pt}
\counterwithin{construccion}{section}
%%%%%%%%%%%%%%%%%%%%%%%%%%%%%%%%%%%%%% NoDefinicion %%%%%%%%%%%%%%%%%%%%%%%%%%%%%%
\newcounter{tndefinido}[section]
\newenvironment{tndefinido}[2]{\vskip 5pt
\refstepcounter{tndefinido}
  \begin{tikzpicture}
\tikzstyle{mybox} = [draw=blue, fill=yellow!20, very thick,
    rectangle, rounded corners, inner sep=20pt, inner ysep=20pt]
\tikzstyle{fancytitle} =[fill=blue, text=white, ellipse]
\stepcounter{tndefinido}
 \hspace*{-15pt}
\node [mybox] (box){ %
     \begin{minipage}[r]{0.9\textwidth}
                                    #1
                                 \end{minipage}};
 \node[fancytitle, right=10pt] at (box.north west) {T\'ermino no definido.
\thetndefinido. \ \bf{#2}};
 \node[fancytitle, rounded corners] at (box.east) {$\clubsuit$};  
     \end{tikzpicture} 
                             }
                             {\vskip 5pt }%\greenline \vskip 0.1pt}
\counterwithin{tndefinido}{section}
%%%%%%%%%%%%%%%%%%%%%%%%%%%%%%%%%%%%%%%%%%%%%%%%%%%%%%%%%%%%%
\fontfamily{pbk}
\selectfont
%%%%%%%%%%%%%%%%%%%%%%%%%%%%%% Figura %%%%%%%%%%%%%%%%%%%%%%%%%%
\newenvironment{figura}[3]{\begin{figure}[H]
\centering
                               #1
                              \caption{#2}
                              \label{#3}
                              \end{figure}
}{ \vskip 5pt }
%%%%%%%%%%%%%%%%%%%%%%%%cambiando margen%%%%%%%%%%%%%%%%
\newenvironment{changemargin}[2]
{
\begin{list}{}
{
\setlength{\topsep}{0pt}
\setlength{\evensidemargin}{0pt}%
\setlength{\oddsidemargin}{0pt}
\setlength{\leftmargin}{#1}%
\setlength{\rightmargin}{#2}%
\setlength{\listparindent}{\parindent}%
\setlength{\itemindent}{\parindent}%
\setlength{\parsep}{\parskip}%
}
\item[]
}
{\end{list}}

%%%%%%%%%%%%%%%%%%%%%%%%%%%%%%%%%% Notas %%%%%%%%%%%%%%%%%%%%%%%%%%%%%%%%%%%%%%%
\newenvironment{notas}[2]{\vskip 5pt  \colorbox{blue!30}{\color{white}{#1:\, #2}}
                                     
                                     }{ \vskip 5pt
                                        %\blackline
                                        \vskip 0.2pt}
   %%%%%%%%%%%%%%%%%%%%%%%%%%%%%%%%%%%%% Prueba %%%%%%%%%%%%%%%%%%%%%%%%%%%%%%% 
\newenvironment{prueba}[2]{\renewcommand{\arraystretch}{2.06}
\newcommand{\tn}{\tabularnewline}
\newcommand{\rr}{\raggedright}
\fontsize{7}{6}\selectfont
\vskip 5pt  {
 \colorbox{red!30}{\color{white}{Prueba:}}
 }\vskip 5pt
 %
\begin{center}
\begin{tabular}
[t]{ll||l}\hline
\multicolumn{2}{c||}{Afirmaciones} & Razones\\\hline\hline
1. & #1 & \multicolumn{1}{|l}{Dado}\\
#2
\end{tabular}\end{center}
\ \ \  \vskip 5pt}{\hfill  \raggedright{ \rule{0.5em}{0.5em} }\vskip 5pt }
 %%%%%%%%%%%%%%%%%%%%%%%%%%%%%%%%%%%%%%%%%%%%%%%%%%%%%%%%%%%%%%%%%%%%%%%%%%%
  %%%%%%%%%%%%%%%%%%%%%%%%%%%%%%%%%%%%% Solucion %%%%%%%%%%%%%%%%%%%%%%%%%%%%%%%
\newenvironment{sol}[2]{\renewcommand{\arraystretch}{2.06}
\newcommand{\tn}{\tabularnewline}
\newcommand{\rr}{\raggedright}
\fontsize{7}{6}\selectfont
\vskip 5pt  {
 \colorbox{red!30}{\color{white}{Soluci\'on:}}
 }\vskip 5pt
 %
\begin{center}
 \begin{tabular}
[t]{ll||l}\hline
\multicolumn{2}{c||}{Afirmaciones} & Razones\\\hline\hline
1. & #1 & \multicolumn{1}{|l}{Dado}\\
#2
\end{tabular}\end{center}
\ \ \  \vskip 5pt}{\hfill  \raggedright{ \rule{0.5em}{0.5em} }\vskip 5pt }
 %%%%%%%%%%%%%%%%%%%%%%%%%%%%%%%%%%%Ejemplocon grafica
%%%%%%%%%%%%%%%%%%%%%%%%%%%%%%%%%%%%% % % %
%   \newcounter{ejemplo}
% \newenvironment{ejemplo}[1]{\vskip 5pt
% \stepcounter{ejemplo}
% \begin{tikzpicture}
% \tikzstyle{mybox} = [draw=red, fill=blue!20, very thick,
%     rectangle, rounded corners, inner sep=10pt, inner ysep=20pt]
% \tikzstyle{fancytitle} =[fill=red, text=black]
% \node [mybox] (box){%
%      \begin{minipage}[r]{\textwidth} 
%                                      {#1}
%                                  \end{minipage}};
%  \node[fancytitle, right=10pt] at (box.north west) {Ejemplo.
% \thechapter.\theejemplo};
%  \node[fancytitle, rounded corners] at (box.east) {$\clubsuit$};              
 
%    
%       \end{tikzpicture} 
%                         \hspace*{10pt}         }{
%  
%                 \vskip 5pt
%  }%\greenline \vskip 0.1pt}
% 
% %%%%%%%%%%%%%%%%%%%%%%%%%%
%\newcounter{ejemplo}
\newenvironment{ejemgraf}[2]{\vskip 5pt
\stepcounter{ejemplo}
\begin{minipage}[t][0.8\height]{0.35\textwidth}
\bf{Ejemplo. \theejemplo\ } {#1}

\end{minipage}\vline \vline \vline \hfill
\begin{minipage}[t][0.9\height][t]{0.50\textwidth}
 {#2}
\end{minipage}
 }
 %\counterwithin{ejemplo}{section} 
  
%%%%%%%%%%%%%%%%%%%%%%%%%%%%%%%%%%%%%%%%%%%%%%%%%%%%%%%%%%%%%%%%%%%%%%%%%%%%%%%%%%%
%$$$$$$$$$$$$$$$$$$$$$$$ Comandos $$$$$$$$$$$$$$$$$$$$$$$$$$$$$$$$$$$$$$$$$$
%%%%%%%%%%%%%%%%%%%%%%%%%%%%%%%%%%%%%% solucion %%%%%%%%%%%%%%%%%%%%%%%%%%%%%%%%%%%%%%%%%%
\newcommand{\solucion}
 {\vskip 5pt
  {
 \colorbox{red!30}{\color{white}{Soluci\'on: \,}}
 }
}
%%%%%%%%%%%%%%%%%%%%%%%%%%%%%%%%%%%%%%%%% problemas %%%%%%%%%%%%%%%%%%%%%%%%%%%
\newcommand{\problema}[1]
{
%\noindent
\section{Problemas}
\begin{multicols}{2}
\begin{enumerate}
 #1
\end{enumerate}
\end{multicols}
{\setlength{\parindent}{0mm}\color{black}\rule{\linewidth}{1mm}}
}         
%%%%%%%%%%%%%%%%%%%%%%%%%%%% Arco %%%%%%%%%%%%%%%%%%%%%%%%%%%%%%%%%%%%%%%%
\newcommand{\arco}[2][30pt]{\overset{\tikz \draw[rotate=60,line width=1pt]
 (#1,0pt) arc (0:60:#1);}{#2}}
%%%%%%%%%%%%%%%%%%%%%%%%%%%%%%%%%%%%%%%%%%%%%%%%%Nota y solución%%%%%%%%%%%%%%%%%%%%%%%%%%%%%%%%%%%%%%%%%%%%%%%%%%%%%%%%%%%%%%%%
%\newcommand{\solucion}{\colorbox{black}{\color{white}{Soluciön:}}}
\newcommand{\nota}{\colorbox{red!20!white}{\color{black}{Nota:}}}
\newcommand{\degre}{\ensuremath{^\circ}}
%%%%%%%%%%%%%%%%%%%%%%%%%%% Cierra Introducción de tipos  @ %%%%%%%%%%%%%%%%%%%%%%%%%%%%%%%%
\makeatother
%%%%%%%%%%%%%%%%%%%%%%%%%%%Regla y transportador %%%%%%%%%%%%%%%%%%%%%%%%%%
\newcommand{\regla}[1]{
\begin{center}
\begin{tikzpicture}[scale=#1,font=\fontsize{7}{7}\selectfont]
% La regla de plastico
\draw[thick,red] (-0.25,0.0) rectangle (25.25,2.5);
% Marcas de milímetros
\foreach \x in{0.0, 0.1,...,25.0}
	\draw[blue,shift={(\x,0)}] (0,0.5pt) -- (0,5pt);
% Marcas de centímetros	
\foreach \x in{0, 1, ..., 25}
	\draw[thick,shift={(\x,0)}] (0,-0.5pt) -- (0,7.5pt) node[above]{$\x$};
% Para que escriban su nombre los niños:
%\node[blue,right]	at (0,1.75) {\textbf{Nombre:} \rule{7.5cm}{0.5pt}};
\end{tikzpicture}
\end{center}
}
%%%%%%%%%%%%%%%%%%%%%%%%%%%%%%%%%%%%%%%%%%%%%%%%%%%%%%%%%%%%%%%%%%%%%%%%%%
\newcommand{\transportador}[1]{
\begin{center}
\begin{tikzpicture}[scale=#1,font=\fontsize{7}{7}\selectfont]
\draw[thick] (5,0) arc (0:180:5);
\draw[thick] (5,0) -- (5,-5pt) -- (-5,-5pt) -- (-5,0);
%
\draw[red,fill=red] (0,0) circle (3.5pt);
% Grados
\foreach \ang in {0,1,...,180}
	\draw[rotate=\ang] (4.9,0) -- (5.0,0);
% Multiplos de 5
\foreach \mult in {0, 5,..., 180}
	\draw[thick,rotate=\mult] (4.8,0) -- (5.0,0);
% Multiplos de 10 (Solamente nodos)
\foreach \nodo in {0, 10, ...,180}
	\node[red] at (\nodo:4.5) {\nodo};
% Multiplos de 10 [los últimos]
\foreach \ult in {0, 10, ...,180}
	\draw[very thick,rotate=\ult,red]	(4.8,0) -- (5.0,0);
\end{tikzpicture}
\end{center}
}
%%%%%%%%%%%%%%%%%%%%%%%%%%%%%%%%%%%%%%%%%%%%%%%%%%%%%%%%%%%%%%%%%%%%%%%%
\newcommand{\instruccion}[1]{
	\vspace{\fill}
	\begin{center}
	\cooltooltip
	[1 1 0]
	{Instrucciones:}
	{Imprima en hojas tamaño carta sin lineas. \\
	Utilice sólo regla sin graduar y compás.	
	}
	{http://www.uninorte.edu.co}{Catálogo Web}
	{#1\strut}
	\end{center}
}
%%%%%%%%%%%%%%%%%%%%%%%%%%%%%triangulo    %%%%%%%%%%%%%%%%%%%%%%%%%%%%%%
\newcommand{\triangulo}{\tikz \draw[scale=0.5] (0,0) -- ++(60:1)
-- ++(-60:1) -- cycle ;
                 }
%%%%%%%%%%%%%%%%%%%%%%%%%%%%%%%%%%%%%%%%%%%%%%%%%%%%%%%%%%%%%%%%%%%%%%%%%%
%%%%%%%%%%%%%%%%%%%%%%%%%%%%%%%%%5Probar %%%%%%%%%%%%%%%%%%%%%
\newenvironment{probar}[1]
 {\vskip 5pt
  
 \colorbox{red!30}{\color{white}{Prueba}}\vskip 5pt

#1 
 }
{\hfill  \raggedright{ \rule{0.5em}{0.5em} }\vskip 5pt }
%%%%%%%%%%%%%%%%%%%%%%%%%%%%%%%%%%%%%%%%%%%%%%%%%%%%%%%%%%%%%%
%%%%%%%%%%%%%%%%%%%%%%%%%%%%%%%%%%%%%b Cuerpo del Documento %%%%%%%%%%%%%%%
%$$$$$$$$$$$$$$$$$$$$$$$$$$$$$$$$$$$$$$$$$$$$$$$$$$$$$$$$$$$$$$$$$$$$$$$$
%%%%%%%%%%%%%%%%%%%%%%%%%%%%%%%%%%%%%%%%%%%%%%%%%%%%%%%%%%%%%%%%%%%%%%%%%
% makeidx
\makeindex
\begin{document}
\frontmatter
\title{Geometr\'ia con regla y compás}
\author{UNIVERSIDAD DEL NORTE
\and \'{A}REA DE CIENCIAS B\'{A}SICAS
\and MATEM\'{A}TICAS Y ESTAD\'ISTICA
\and Esp. ANTALCIDES OLIVO}
\maketitle
\chapterimage{./Geometria2.jpg}
\tableofcontents
\mainmatter
\thispagestyle{plain}
\setcounter{chapter}{1}
\setcounter{page}{1}
%\begin{quote}
\bigskip
%\end{quote}
\chapterimage{./Geometria2.jpg}
\chapter{Postulados y el razonamiento en geometr\'ia }
\label{chap:2}
\vspace{40pt}
\section*{Introducci\'on}
En este cap\'itulo seguiremos con la construci\'on axiomática de la geometr\'ia
y paralelamente presentaremos construcciones con regla y c\'ompas, que luego al
final de este cap\'itulo empezaremos a demostrar.\\
Del párrafo anterior se nos presenta una duda, ¿Qué es demostrar? \\
Podemos considerar la demostración de una proposición $q$ como una cadena finita
de conclusiones que se realizan mediante reglas lógicas y que se forman a partir
de proposiciones verdaderas o supuestamente verdaderas y las cuales nos conducen
a una proposición $p$.\\
La proposición o relación resultante de otras mediante el proceso de la
demostración se llama \texttt{teorema} y este proceso podemos dividirlo
tres partes (figura\ref{demos} ):
\begin{lista}
\item  Conocer la proposición que se trata de demostrar. En esta parte es
importante diferenciar muy claramente la información que nos dan (la
hipótesis) de lo que nos solicitan que demostremos (la tesis).
\item Los fundamentos empleados como base de la demostración. Estos fundamentos
   están constituidos por los términos primitivos, las definiciones, los
postulados y las proposiciones o teoremas ya demostrados.
\item  El procedimiento usado para lograr que la proposición quede demostrada
(elegir el método adecuado).
\vspace{-20pt} 
\end{lista}
\begin{figura}{
\definecolor{zzccff}{rgb}{0.6,0.8,1}
\definecolor{cqcqcq}{rgb}{0.75,0.75,0.75}
\begin{tikzpicture}[scale=1.2,line cap=round,font=\fontsize{7}{7}\selectfont,line
join=round,>=triangle
45,x=1.0cm,y=1.0cm]
%\draw [color=cqcqcq,dash pattern=on 1pt off 1pt, xstep=2.0cm,ystep=2.0cm]
%(-0.27,1.66) grid (12.29,5.02);
\clip(-0.27,1.66) rectangle (12.29,5.02);
\fill[color=zzccff,fill=zzccff,fill opacity=0.1] (0,4) -- (4,4) -- (4,2) --
(0,2) -- cycle;
\fill[color=zzccff,fill=zzccff,fill opacity=0.1] (4.5,3.5) -- (6.5,3.5) --
(6.5,4) -- (7.5,3) -- (6.5,2) -- (6.5,2.5) -- (4.5,2.5) -- cycle;
\fill[color=zzccff,fill=zzccff,fill opacity=0.1] (8,4) -- (12,4) -- (12,2) --
(8,2) -- cycle;
\draw [color=zzccff] (0,4)-- (4,4);
\draw [color=zzccff] (4,4)-- (4,2);
\draw [color=zzccff] (4,2)-- (0,2);
\draw [color=zzccff] (0,2)-- (0,4);
\draw [color=zzccff] (4.5,3.5)-- (6.5,3.5);
\draw [color=zzccff] (6.5,3.5)-- (6.5,4);
\draw [color=zzccff] (6.5,4)-- (7.5,3);
\draw [color=zzccff] (7.5,3)-- (6.5,2);
\draw [color=zzccff] (6.5,2)-- (6.5,2.5);
\draw [color=zzccff] (6.5,2.5)-- (4.5,2.5);
\draw [color=zzccff] (4.5,2.5)-- (4.5,3.5);
\draw [color=zzccff] (8,4)-- (12,4);
\draw [color=zzccff] (12,4)-- (12,2);
\draw [color=zzccff] (12,2)-- (8,2);
\draw [color=zzccff] (8,2)-- (8,4);
\draw (0.61,3.69) node[anchor=north west] {\parbox{3.78 cm}{Términos no
definidos \\ Definiciones 
\\  Postulados \\  Otros  teoremas}};
\draw (1.16,4.54) node[anchor=north west] {Hipótesis};
\draw (4.48,4.52) node[anchor=north west] {Razonamieto lógico};
\draw (9.47,4.57) node[anchor=north west] {Tesis};
\draw (9.31,3.09) node[anchor=north west] {Conclusión};
\draw (4.7,3.36) node[anchor=north west] {\parbox{2.71 cm}{Esquema de 
\\  razonamiento}};
\end{tikzpicture}}{Esquema de una demostración}{demos}
 Para realizar una demostración además de tener las reglas lógicas y el
encadenamiento de proposiciones, también hay que determinar un método o
razona\-miento como expondremos a continuación.
\end{figura}
\vspace{-20pt}
\section{Razonamiento inductivo}
El razonamiento inductivo es el proceso mediante el cual se obtienen
conclusiones a partir de nuestras propias observaciones o a partir de ejemplos
particulares, es decir al observar que una acci\'on o propiedad se repite se
concluye en general que esa acci\'on o propiedad siempre es cierta
\begin{ideas}{Es la conclusi\'on que se obtiene a partir de un
\vspace{-20pt}
proceso inductivo}{Conjetura}
\end{ideas}

\begin{ejemplo}{Suponga que una persona mide los lados de cuatro  tri\'angulos
como muestra la figura(\ref{dest})}
\begin{figura}{
\definecolor{qqqqff}{rgb}{0,0,1}
\begin{tikzpicture}[scale=5.0, font=\fontsize{7}{8}\selectfont,line
cap=round,line join=round,>=triangle 45,x=1.0cm,y=1.0cm]
\clip(0.01,-0.13) rectangle (0.97,2.28);
\draw[line width=1.2pt] (0.05,1.19)-- (0.23,1.93);
\draw[line width=1.2pt] (0.23,1.93)-- (0.36,1.17);
\draw[line width=1.2pt] (0.36,1.17)-- (0.05,1.19);
\draw[line width=1.2pt] (0.11,1.67) node[anchor=north west] {0.77};
\draw[line width=1.2pt] (0.31,1.63) node[anchor=north west] {0.77};
\draw[line width=1.2pt] (0.2,1.13) node[anchor=north west] {0.31};
\draw[line width=1.2pt] (0.44,1.86)-- (0.61,1.12);
\draw[line width=1.2pt] (0.61,1.12)-- (0.91,1.15);
\draw[line width=1.2pt] (0.91,1.15)-- (0.44,1.86);
\draw[line width=1.2pt] (0.5,1.45) node[anchor=north west] {0.76};
\draw[line width=1.2pt] (0.74,1.06) node[anchor=north west] {0.3};
\draw[line width=1.2pt] (0.66,1.65) node[anchor=north west] {0.85};
\draw[line width=1.2pt] (0.06,0.23)-- (0.35,0.77);
\draw[line width=1.2pt] (0.35,0.77)-- (0.4,0.11);
\draw[line width=1.2pt] (0.06,0.23)-- (0.4,0.11);
\draw[line width=1.2pt] (0.38,0.51) node[anchor=north west] {0.67};
\draw[line width=1.2pt] (0.25,0.11) node[anchor=north west] {0.35};
\draw[line width=1.2pt] (0.21,0.68) node[anchor=north west] {0.61};
\draw[line width=1.2pt] (0.54,0.74)-- (0.54,0.11);
\draw[line width=1.2pt] (0.54,0.11)-- (0.81,0.13);
\draw[line width=1.2pt] (0.81,0.13)-- (0.54,0.74);
\draw[line width=1.2pt] (0.67,0.54) node[anchor=north west] {0.35};
\draw[line width=1.2pt] (0.51,0.47) node[anchor=north west] {0.62};
\draw[line width=1.2pt] (0.66,0.1) node[anchor=north west] {0.27};
\fill [color=qqqqff] (0.05,1.19) circle (0.5pt);
\draw[color=qqqqff] (0.04,1.18) node {$A$};
\fill [color=qqqqff] (0.23,1.93) circle (0.5pt);
\draw[color=qqqqff] (0.23,2) node {$B$};
\fill [color=qqqqff] (0.36,1.17) circle (0.5pt);
\draw[color=qqqqff] (0.37,1.14) node {$C$};
\fill [color=qqqqff] (0.44,1.86) circle (0.5pt);
\draw[color=qqqqff] (0.44,1.95) node {$D$};
\fill [color=qqqqff] (0.61,1.12) circle (0.5pt);
\draw[color=qqqqff] (0.61,1.06) node {$E$};
\fill [color=qqqqff] (0.91,1.15) circle (0.5pt);
\draw[color=qqqqff] (0.92,1.13) node {$F$};
\fill [color=qqqqff] (0.06,0.23) circle (0.5pt);
\draw[color=qqqqff] (0.05,0.25) node {$G$};
\fill [color=qqqqff] (0.35,0.77) circle (0.5pt);
\draw[color=qqqqff] (0.36,0.83) node {$H$};
\fill [color=qqqqff] (0.4,0.11) circle (0.5pt);
\draw[color=qqqqff] (0.41,0.1) node {$I$};
\fill [color=qqqqff] (0.54,0.74) circle (0.5pt);
\draw[color=qqqqff] (0.53,0.81) node {$J$};
\fill [color=qqqqff] (0.54,0.11) circle (0.5pt);
\draw[color=qqqqff] (0.53,0.08) node {$K$};
\fill [color=qqqqff] (0.81,0.13) circle (0.5pt);
\draw[color=qqqqff] (0.82,0.15) node {$L$};
\end{tikzpicture}}{Desigualdad triangular}{dest}
\end{figura}
De los cuatro tri\'angulos podemos concluir que la suma de las longitudes de dos
lados siempre es menor que la longitud del tercer lado.
 \end{ejemplo}
\section{M\'etodo del contraejemplo}
Hay ocasiones donde despu\'es de un razonamiento inductivo obtenemos una
conjetura que no se cumple para todos los casos, es decir obtenemos una
generalizaci\'on falsa, entonces para indicar que esa generalizaci\'on es falsa
buscamos un ejemplo donde no se cumpla la acci\'on o la propiedad.
\begin{ideas}{Es el m\'etodo que se usa para demostrar que una
generalizaci\'on es falsa, utilizando un ejemplo que la
contradiga.}{Contraejmplo}
\end{ideas}
\begin{ejemplo}{Si un cuadril\'atero tiene sus diagonales perpendiculares,
entonces es un rombo.}
\begin{figura}{
\definecolor{qqwuqq}{rgb}{0,0.39,0}
\definecolor{uququq}{rgb}{0.25,0.25,0.25}
\definecolor{xdxdff}{rgb}{0.49,0.49,1}
\definecolor{qqqqff}{rgb}{0,0,1}
\begin{tikzpicture}[scale=0.7,font=\fontsize{8}{9}\selectfont, cap=round,line
join=round,>=triangle 45,x=1.0cm,y=1.0cm,line width=1.2pt]
\clip(-3.28,-2.66) rectangle (1.6,11.45);
\draw[color=qqwuqq,fill=qqwuqq,fill opacity=0.1] (-0.33,4.17) -- (-0.33,4.34) --
(-0.51,4.34) -- (-0.5,4.17) -- cycle;
\draw (-0.53,9.97)-- (-0.48,-1.04);
\draw (-2.96,4.16)-- (-0.53,9.97);
\draw (-0.53,9.97)-- (1.05,4.18);
\draw (1.05,4.18)-- (-0.48,-1.04);
\draw (-0.48,-1.04)-- (-2.96,4.16);
\draw (-1.92,8.12) node[anchor=north west] {6.29};
\draw (0.22,8.34) node[anchor=north west] {6};
\draw (-2.96,4.16)-- (1.05,4.18);
\fill [color=qqqqff] (-0.53,9.97) circle (1.5pt);
\draw[color=qqqqff] (-0.52,10.58) node {$A$};
\fill [color=qqqqff] (-0.48,-1.04) circle (1.5pt);
\draw[color=qqqqff] (-0.46,-1.47) node {$B$};
\fill [color=qqqqff] (1.05,4.18) circle (1.5pt);
\draw[color=qqqqff] (1.12,4.65) node {$C$};
\fill [color=xdxdff] (-2.96,4.16) circle (1.5pt);
\draw[color=xdxdff] (-2.96,3.23) node {$D$};
\fill [color=uququq] (-0.5,4.17) circle (1.5pt);
\draw[color=qqwuqq] (-0.12,4.94) node {$\alpha = 90$};
\end{tikzpicture}}{Cometa}{com}
 \end{figura}
En el cuadril\'atero podemos observar que $AD=6.29$ y $AC=6$ y por definici\'on
de rombo el cuadril\'atero $ABCD$ no puede ser un rombo porque no tiene sus
lados congruentes.
\end{ejemplo}
\section{Razonamiento deductivo}
El método deductivo consiste en partir de un número reducido de información
(hipótesis) y mediante un proceso lógico deducir otros conocimientos o
proposiciones nuevas.
Para profundizar y entender este método explicaremos a continuación cuales son
los procesos lógicos.
\section{Tipos de proposiciones lógicas}
En este curso casi siempre trabajaremos con proposiciones compuestas, es decir
una proposici\'on que esta formada por un conjunto de proposiciones, unidas por
operadores l\'ogicos llamados conectores:\\ Esos conectores son
\begin{lista}
\item $\sim$ , es una proposici\'on simple a la cual se le cambia el valor de
verdad. La proposici\'on se llama nagaci\'on.
\setlength{\parindent}{-25pt}
\begin{ejemplo}{La negación de la proposición \textquotedblleft Los ángulos
$\angle\alpha$ y $\angle\beta$ son rectos\textquotedblright\, es:\\
\textquotedblleft El ángulo $\angle\alpha$ no es recto o el ángulo
$\angle\beta$ no es recto\textquotedblright.}
\end{ejemplo}
 \item $\vee$ , a la proposici\'on compuesta formada por este conectivo que se
lee "o``, se llama disyunci\'on.
\begin{ejemplo}{
 Un cuadril\'atero es un polígono con cuatro lados o cuatro v\'ertices.}
\end{ejemplo}

\item $\wedge$ , la proposici\'on resultante se llama conjunci\'on y el conector
se lee " y ''
\begin{ejemplo}{Un polígono regular tiene los ángulos interiores y sus lados
congruentes.}
 \end{ejemplo}
\item $ \mbox{si}\cdots , \mbox{entonces } \cdots$, la proposic\'on resultante
se llama condicional.
\begin{ejemplo}{Dos rectas que no se intersecan son paralelas o alabeadas}
 Est\'a proposición es de la forma $p \rightarrow \left( q \vee r\right) $
\end{ejemplo}
\item $\cdots \mbox{si y s\'olo si} \cdots$, a la proposic\'on resultante se le
llama bicondicional.
\end{lista}
Vamos a hablar un poco de las dos \'ultimas proposiciones:
La condicional es una proposici\'on que se denota de la forma $p\longrightarrow
q$, donde $p$ y $q$ son proposiciones que llamaremos, hip\'otesis y conclusi\'on
respectivamente.
Por ejemplo
\begin{ejemplo}{ Si un tri\'angulo tiene un \'angulo recto, entonces es un
tri\'angulo rect\'angulo.}
En este caso la hip\'otesis es : el tri\'angulo es rect\'angulo y la
conclusi\'on es: el tri\'angulo es rect\'angulo.
\end{ejemplo}
La bicondicional se denota $p\longleftrightarrow q$, y se puede descomponer en
la conjunci\'on de dos condicionales, as\'i: $ p\longrightarrow
q \wedge q\longrightarrow p$.
\begin{ejemplo}{Dos rectas son paralelas si y sólo si sus ángulos alternos
internos formados por una transversal son congruentes}
en este ejemplo $p$\ es: Dos rectas son paralelas.\\
$q$\ es: Los ángulos formados por una transversal son congruentes.   
\end{ejemplo}

\nota\ Para saber el valor de verdad  proposición compuesta se utilizan las definiciones
 organizadas en la siguiente tabla:
\[%
\begin{tabular}
[c]{llllll}%
$ p$ &$q$ & $p\wedge q$ & $p\vee q$ & $p\rightarrow q$ &
$p\leftrightarrow q$\\
\multicolumn{1}{c}{V}&\multicolumn{1}{c}{V} & \multicolumn{1}{c}{V} & \multicolumn{1}{c}{V} &
\multicolumn{1}{c}{V} & \multicolumn{1}{c}{V}\\
\multicolumn{1}{c}{V}&\multicolumn{1}{c}{F} & \multicolumn{1}{c}{F} & \multicolumn{1}{c}{V} &
\multicolumn{1}{c}{F} & \multicolumn{1}{c}{F}\\
\multicolumn{1}{c}{F}&\multicolumn{1}{c}{V} & \multicolumn{1}{c}{F} & \multicolumn{1}{c}{V} &
\multicolumn{1}{c}{V} & \multicolumn{1}{c}{F}\\
\multicolumn{1}{c}{F}&\multicolumn{1}{c}{F} & \multicolumn{1}{c}{F} & \multicolumn{1}{c}{F} &
\multicolumn{1}{c}{V} & \multicolumn{1}{c}{V}%
\end{tabular}
\]
Con la condicional y la bicondicional se pueden construir dos tautologías de la
siguiente
manera:
\begin{ideas}{Es una condicional que siempre es verdadera}{Implicaci\'on}
 Por ejemplo:
\end{ideas}
\begin{ejemplo}{ Si un cuadril\'atero tiene dos lados opuestos paralelos,
entonces estos lados son congruentes.}
La figura(\ref{tra}) llamada trapecio tiene los lados $\overline{AB}$ y
$\overline{CD}$ paralelos y no son congruentes. Por tanto esta condicional no es
una implicaci\'on.
\begin{figura}{\definecolor{xdxdff}{rgb}{0.49,0.49,1}
\definecolor{qqqqff}{rgb}{0,0,1}
\begin{tikzpicture}[scale=0.8,font=\fontsize{9}{8}\selectfont,line
cap=round,line join=round,>=triangle 45,x=1.0cm,y=1.0cm,line width=1.2pt]
\clip(-1.07,-1.25) rectangle (5.06,2.06);
\draw (-0.91,-0.76)-- (-0.21,1.43);
\draw (-0.21,1.43)-- (3.48,1.39);
\draw (3.48,1.39)-- (4.77,-0.81);
\draw (4.77,-0.81)-- (-0.91,-0.76);
\draw (1.46,1.50) node[anchor=north west] {3.69};
\draw (1.7,-0.650) node[anchor=north west] {5.69};
\fill [color=qqqqff] (-0.21,1.43) circle (1.5pt);
\draw[color=qqqqff] (-0.40,1.55) node {$A$};
\fill [color=qqqqff] (3.48,1.39) circle (1.5pt);
\draw[color=qqqqff] (3.55,1.7) node {$B$};
\fill [color=xdxdff] (-0.91,-0.76) circle (1.5pt);
\draw[color=xdxdff] (-0.9,-1.1) node {$D$};
\fill [color=xdxdff] (4.77,-0.81) circle (1.5pt);
\draw[color=xdxdff] (4.85,-1.1) node {$E$};
\end{tikzpicture}}{Trapecio}{tra}
\end{figura}
La condicional se denota $p \Longrightarrow q$ y se lee $p$ implica $q$
\end{ejemplo}
\begin{ideas}{Es una bicondicional que siempre es cierta.}{Doble implicaci\'on}
Por ejemplo todas las definiciones son bicondicionales.
\nota\ La doble implicaci\'on se denota  $p \Longleftrightarrow q$ y se dice que
que $p$ y $q$ son proposiciones lógicamente equivalentes
\end{ideas}
\begin{ejemplo}{Analice el valor de verdad de la proposición compuesta%
\[
(p\rightarrow q)\leftrightarrow(\sim q\rightarrow\sim p)
\]
}
\solucion\ 
%
\[%
\begin{tabular}
[c]{ccccccc}%
$p$ & $q$ & $p\rightarrow q$ & $\sim q$ & $\sim p$ & $\sim q\rightarrow\sim p$
& $(p\rightarrow q)\leftrightarrow(\sim q\rightarrow\sim p)$\\
V & V & V & F & F & V & V\\
V & F & F & V & F & F & V\\
F & V & V & F & V & V & V\\
F & F & V & V & V & V & V
\end{tabular}
\ \
\]
Es decir las proposiciones: $p\rightarrow q$ y $\sim
q\rightarrow\sim p$\ son equivalentes. 
 \end{ejemplo}

\begin{ejemplo}{Presentaremos algunos ejemplos de implicaciones y equivalencia
lógicas\label{ej2_11} } 
\begin{enumerate}
\item $\left[  p\wedge\left(  p\rightarrow q\right)  \right]  \Longrightarrow
q$

\item $\left[  \left(  p\rightarrow q\right)  \wedge\sim q\right]
\Longrightarrow\sim p$

\item $\left[  \left(  p\rightarrow q\right)  \wedge\left(  q\rightarrow
r\right)  \right]  \Longrightarrow\left(  p\rightarrow r\right)  $

\item $\left(  p\wedge q\right)  \Longrightarrow p$
\item $\sim\left(  p\wedge q\right)  \Longleftrightarrow\left(  \sim p\vee\sim
q\right)  $

\item $\sim\left(  p\vee q\right)  \Longleftrightarrow\left(  \sim p\wedge\sim
q\right)  $

\item $\left(  p\rightarrow q\right)  \Longleftrightarrow\left(  \sim
q\rightarrow\sim p\right)  $

\item $\left[  p\rightarrow\left(  q\vee r\right)  \right]
\Longleftrightarrow\left[  \left(  p\wedge\sim q\right)  \rightarrow r\right]
$

\item $\left[  \left(  p\rightarrow q\right)  \wedge\left(  q\rightarrow
p\right)  \right]  \Longleftrightarrow\left(  p\leftrightarrow q\right)  $

\item $\left(  p\rightarrow q\right)  \Longleftrightarrow\left(  \sim p\vee
q\right)  $

\item $\sim\left(  p\rightarrow q\right)  \Longleftrightarrow\left(
p\wedge\sim q\right)  $
\item $p \rightarrow \left( p \vee q\right)$
\end{enumerate}
\end{ejemplo} 
\subsection{Otras formas de expresar una implicación}
En algunos casos las condicionales, vienen expresadas de tal forma que cada
proposición simple se puede representar como un conjunto, por ejemplo
\begin{equation}
 \mbox{Los polígonos de cuatro lados son cuadriláteros.}
\label{ec1}
\end{equation} 
En este caso llamaremos $P_x=\{x:x\ \mbox{es un polígono}\}$\ y
$Q_x=\{x:x\ \mbox{es un cuadrilátero}\}$ \footnote{A las proposiciones que
dependen de una variable y de un conjunto referencia, como es el caso de $P_x$\
y $Q_x$ se llaman proposiciones abiertas. } y representamos la proposición
(\ref{ec1}) como
$P_x \rightarrow Q_x$, pero a nosotros nos interesan son las implicaciones, por
tanto veamos cuando está proposición es verdadera. Para ello tomaremos otro
conjunto $U_x=\{x:x\ \mbox{es una figura plana}\}$, a este conjunto lo
llamaremos universal o de referencia.
Ahora realizaremos un diagrama de Venn mostrado en la figura (\ref{impli}).
\begin{figura}{
\definecolor{qqqqff}{rgb}{0,0,1}
\definecolor{ffttww}{rgb}{1,0.2,0.4}
\definecolor{ttttff}{rgb}{0.2,0.2,1}
\definecolor{zzttqq}{rgb}{0.6,0.2,0}
\definecolor{cqcqcq}{rgb}{0.75,0.75,0.75}
\begin{tikzpicture}[line cap=round,font=\fontsize{6}{6}\selectfont,line
join=round,>=triangle 45,x=1.0cm,y=1.0cm]
%\draw [color=cqcqcq,dash pattern=on 1pt off 1pt, xstep=1.0cm,ystep=1.0cm]
%(1,0.42) grid (9.53,3.71);
\clip(1,0.42) rectangle (9.53,3.71);
\fill[color=zzttqq,fill=zzttqq,fill opacity=0.1] (1.5,3.5) -- (1.5,1) -- (5,1)
-- (5,3.5) -- cycle;
\fill[color=zzttqq,fill=zzttqq,fill opacity=0.1] (5.5,3.5) -- (5.5,1) -- (9,1)
-- (9,3.5) -- cycle;
\draw [color=zzttqq] (1.5,3.5)-- (1.5,1);
\draw [color=zzttqq] (1.5,1)-- (5,1);
\draw [color=zzttqq] (5,1)-- (5,3.5);
\draw [color=zzttqq] (5,3.5)-- (1.5,3.5);
\draw [color=zzttqq] (5.5,3.5)-- (5.5,1);
\draw [color=zzttqq] (5.5,1)-- (9,1);
\draw [color=zzttqq] (9,1)-- (9,3.5);
\draw [color=zzttqq] (9,3.5)-- (5.5,3.5);
\draw [rotate around={-20.14:(3.26,2.23)},line
width=2pt,color=ttttff,fill=ttttff,fill opacity=0.1] (3.26,2.23) ellipse (1.58cm
and 0.87cm);
\draw [rotate around={-17.08:(7.14,2.21)},line
width=2pt,color=ttttff,fill=ttttff,fill opacity=0.1] (7.14,2.21) ellipse (1.48cm
and 0.88cm);
\draw [line width=2pt,color=ffttww,fill=ffttww,fill opacity=0.1] (3.22,2.23)
circle (0.64cm);
\draw [line width=2pt,color=ffttww,fill=ffttww,fill opacity=0.1] (7.27,2.07)
circle (0.63cm);
\draw (4.00,3.41) node[anchor=north west] {$\mathbf{U_x}$};
\draw (2.07,3.07) node[anchor=north west] {$\mathbf{P_x}$};
\draw (3.12,2.79) node[anchor=north west] {$\mathbf{Q_x}$};
\draw (4.0,2.26) node[anchor=north west] {$\mathbf{x_1}$};
\draw (5.99,2.94) node[anchor=north west] {$\mathbf{P_x}$};
\draw (7.63,3.47) node[anchor=north west] {$\mathbf{U_x}$};
\draw (7.18,2.11) node[anchor=north west] {$\mathbf{x_2}$};
\draw (6.96,2.64) node[anchor=north west] {$\mathbf{Q_x}$};
\draw (2.81,0.84) node[anchor=north west] {Diagrama 1};
\draw (6.76,0.83) node[anchor=north west] {Diagrama 2};
\fill [color=qqqqff] (7.22,1.86) circle (1.5pt);
\fill [color=qqqqff] (4.16,1.86) circle (1.5pt);
\end{tikzpicture}}{Proposiciones abiertas}{impli}
Si tomamos un polígono, el cual representaremos con el símbolo $x_1$,\
observamos en el diagrama 1 que $x_2$\ no está en el conjunto de los
cuadriláteros, por tanto
en este caso $P_x \rightarrow Q_x$\ es falsa, mientras en el diagrama 2
tomaremos un polígono representado por $x_2$\ el cual está en $Q_x$,\ por tanto 
$P_x \rightarrow Q_x$\ es verdadera, de lo que podemos concluir que para que
$P_x \rightarrow Q_x$\ sea una tautología debe cumplirse la relación.
\[ Q_x \subseteq P_x\]
Para indicar que $P_x \rightarrow Q_x$\ es una tautología utilizamos unos
operadores lógicos de existencia, los cuales son:
\begin{lista}
 \item Existe algún: Se representa $\exists x$\ y se lee existe algún $x.$
\item Para todo: se representa $\forall x$\ y se lee para todo $x.$
\item Existe un único: Se representa $\exists\,! x$\ y se lee existe uno y sólo
un $x.$
\item Ningún: Se representa $\sim\exists x$\ y se lee no existe ningún $X.$
\end{lista}
En nuestro caso la proposición quedaría:
\[\forall x (p_x \Longrightarrow q_x) \]
De aquí en adelante el conjunto que representa a la proposición se representará
con letras mayúsculas y las proposiciones con letras minúsculas con la variable
como subíndice:
\end{figura}
Analicemos nuevamente el diagrama 2 de la figura (\ref{impli})
Como, $x_2$\ está en el conjunto, entonces podemos decir que $P$ es
condición suficiente para que $x_2$\ esté en el conjunto $Q$. En este
sentido, se dice que $p_{x}$ es condición suficiente para $q_{x}$. Además es
claro que para que un elemento $x_2$ esté en $P$\ se necesita que $x_2$ esté en
en $Q$. De aquí que se diga que $q_{x}$ es condición necesaria para $p_{x}.$
También se observa que un elemento $x_2$ está en el conjunto $Q$ si está en el
conjunto $P$. Este análisis precedente sugiere otras maneras de expresar la
implicación
\[
p_{x}\Longrightarrow q_{x}%
\]
éstas son:

\begin{enumerate}
\item $p_{x}$ implica a $q_{x}$

\item $p_{x}$ es condición suficiente para $q_{x}$

\item $q_{x}$ es condición necesaria para $p_{x}$

\item $q_{x},$ si $p_{x}$
\end{enumerate}
\subsection{Derivadas de un condicional}

Asociado al condicional $p_{x}\rightarrow q_{x}$ hay otros tres condicionales
que se consideran en matemáticas. éstos son:

\begin{enumerate}
\item \textbf{La recíproca%
}, cuya estructura es $q_{x}\rightarrow p_{x}$

\item \textbf{La contraria%
\index{Condicional!Contraria de un}%
}, cuya estructura es $\sim p_{x}\rightarrow\sim q_{x}$

\item \textbf{La contrarrecíproca%
}, cuya estructura es $\sim q_{x}\rightarrow\sim p_{x}$
\end{enumerate}

\begin{ejemplo}{
Determine el valor de verdad de la proposición: \textquotedblleft Si dos
rectas son perpendiculares, entonces se intersecan\textquotedblright. Además
escriba la recíproca, la contraria y la contrarrecíproca del condicional con
sus respectivos valores de verdad.}
\end{ejemplo}

\solucion\
De la definición de rectas perpendiculares se deduce que el
condicional dado es verdadero. La recíproca del condicional es:
\textquotedblleft Si dos rectas de intersecan, entonces son
perpendiculares\textquotedblright, la cual es falsa, ya que en el
si\-guien\-te ejemplo se tienen dos rectas que se intersecan y no son
perpendiculares.
\begin{figura}{
\definecolor{qqqqff}{rgb}{0,0,1}
\begin{tikzpicture}[scale=0.5,font=\fontsize{6}{6}\selectfont,line
cap=round,line join=round,>=triangle 45,x=1.0cm,y=1.0cm,line width=1.2pt]
\clip(-1.68,-5.92) rectangle (6.84,7.36);
\draw [domain=-1.00:5.84,<->] plot(\x,{(-8--4*\x)/4});
\draw [domain=-1.00:5.84,<->] plot(\x,{(--22.12-7.12*\x)/-0.38});
\fill [color=qqqqff] (1,-1) circle (1.5pt);
\draw[color=qqqqff] (0.72,-0.66) node {$A$};
\fill [color=qqqqff] (5,3) circle (1.5pt);
\draw[color=qqqqff] (4.88,3.18) node {$B$};
%\draw[color=black] (-2.7,-4.92) node {$a$};
\fill [color=qqqqff] (3.38,5.12) circle (1.5pt);
\draw[color=qqqqff] (3.54,5.38) node {$C$};
\fill [color=qqqqff] (3,-2) circle (1.5pt);
\draw[color=qqqqff] (3.16,-1.74) node {$D$};
%\draw[color=black] (3.68,6.14) node {$b$};
\end{tikzpicture}}{Reciproco}{recc}
\end{figura}

La contraria del condicional dado es: \textquotedblleft Si dos rectas no son
perpendiculares, entonces no se intersecan\textquotedblright, la cual es
falsa. De un contraejemplo.\medskip
La contrarrecíproca del condicional es: \textquotedblleft Si dos rectas no se
intersecan, entonces no son perpendiculares\textquotedblright, ¿cuál es su
valor de verdad?, ¿por qué.?
\section{Esquemas de razonamiento}
Si no podemos encontrar un ejemplo que contradiga una conjetura, no quiere decir
que esa generalizaci\'on sea cierta, el camino a seguir es usar un esquema de
razonamiento que nos asegure que la proposici\'on siempre es verdadera.
\subsection{Prueba indirecta}
Es un razomiento de la forma:
\begin{center}
 \begin{tabular}{c}
$p\longrightarrow q$\\
$\sim q$ \\
\hline \\
$\sim q$
 \end{tabular}
\end{center}
 Como la condicional  debe ser una implicaci\'on, entonces tenemos que para
que ella sea una tautología, solo existen dos posibilidades, que las dos 
proposiciones $p$ y $q$ tengan el mismo valor de verdad. Por tanto si $q$ es
falsa se deduce que $p$ tambi\'en es falsa.
\begin{ejemplo}{\[%
\begin{tabular}
[t]{rl}%
$p\rightarrow q:$ & Si un triángulo tiene tres ángulos congruentes,
entonces es equilátero.\medskip\\
$p:$ & El triángulo $\triangle ABC$ tiene tres ángulos congruentes\medskip
\\\hline
 \\ $q:$ & El triangulo $\triangle ABC$ es equilátero.
\end{tabular}
\ \
\]}
 
\end{ejemplo}

\subsection{Modus ponendus ponens}
Este es un razonamiento de la forma:
\begin{center}
 \begin{tabular}{c}
$p\longrightarrow q$\\
$\sim p$ \\
\hline \\
$\sim p$
 \end{tabular}
\end{center}
La argumentaci\'on es la misma de la prueba indirectaes decir para que
$p\longleftrightarrow q$ sea una implicaci\'on si $p$ es verdadera, se tiene
que $q$ tambi\'en lo es.
\begin{ejemplo}{\[%
\begin{tabular}
[t]{rl}%
$p\rightarrow q:$ & Si dos rectas son paralelas, entonces no tienen puntos
en común.\medskip\\
$\sim q:$ & Las rectas $\overleftrightarrow{l_{1}}$ y $\overleftrightarrow
{l_{2}}$ tienen un punto en común.\medskip \medskip \\\hline
\medskip  \\
$\sim p:$ & Las rectas $\overleftrightarrow{l_{1}}$ y $\overleftrightarrow
{l_{2}}$ no son paralelas.
\end{tabular}
\
\]}
 \end{ejemplo}

\subsection{Regla de la cadena}
Este razonamiento es el m\'as usado en geometr\'ia consiste en construir una
cadena de implicaciones partiendo de la hip\'otesis hasta obtener la
conclusi\'on y es de la forma:
\begin{center}
 \begin{tabular}{c}
$p\longrightarrow r$\\
$r\longrightarrow q$\\
\hline \\
$p\longrightarrow q$.\\
 \end{tabular}
\end{center}
\begin{ejemplo}{\[%
\begin{tabular}
[c]{lll}%
$p\rightarrow q:$ & $\text{Si dos rectas son perpendiculares, entonces se
intersecan.}$\medskip & \\
$q\rightarrow r:$ & $\text{Si dos rectas se intersecan, entonces no son
paralelas.}$\medskip & \\\hline \\
$p\rightarrow r:$ & $\text{Si dos rectas son perpendiculares, entonces no
son paralelas.}$ &
\end{tabular}
\ \
\]}
 \end{ejemplo}
Otra forma de interpretar este razonamiento es: $$\left( p\longrightarrow r
\wedge r\longrightarrow q\right)  \longrightarrow \left(
p\longrightarrow q\right), $$ mirándola de esta forma el razonamiento es
equivalente si la conjunción es cierta entonces la conclusi\'on tambi\'en lo es
decir $p\longrightarrow q$ es una tautología.
\begin{ejemplo}{Sea $\overleftrightarrow{ED}$ una mediatriz del
segmento $\overline{AB}$ en el $\triangulo{ABC}$, si el punto $F$  es
la intersección de lado $AB$ y la mediatriz, entonces
$\overline{AF}\cong \overline{FB}.$ }
\end{ejemplo}
\begin{figura}{\definecolor{xdxdff}{rgb}{0.49,0.49,1}
\definecolor{uququq}{rgb}{0.25,0.25,0.25}
\definecolor{qqqqff}{rgb}{0,0,1}
\begin{tikzpicture}[scale=1.9,font=\fontsize{9}{8}\selectfont,line
cap=round,line join=round,>=triangle 45,x=1.0cm,y=1.0cm,line width=1.2pt]
\clip(-0.59,-1.42) rectangle (4.72,2.48);
\draw (0.37,-0.25)-- (3.55,-0.27);
\draw[<->] (1.96,-1.42) -- (1.96,2.48);
\draw (0.37,-0.25)-- (3.08,1.35);
\draw (3.08,1.35)-- (3.55,-0.27);
\fill [color=qqqqff] (0.37,-0.25) circle (1.5pt);
\draw[color=qqqqff] (0.22,-0.19) node {$A$};
\fill [color=qqqqff] (3.55,-0.27) circle (1.5pt);
\draw[color=qqqqff] (3.71,-0.17) node {$B$};
\fill [color=qqqqff] (3.08,1.35) circle (1.5pt);
\draw[color=qqqqff] (3.18,1.51) node {$C$};
\draw[color=black] (2.12,3.96) node {$b$};
\draw[color=black] (1.86,0.47) node {$c$};
\draw[color=black] (3.16,0.58) node {$d$};
\fill [color=uququq] (1.96,-0.26) circle (1.5pt);
\draw[color=uququq] (2.11,-0.38) node {$F$};
\fill [color=xdxdff] (1.97,1.91) circle (1.5pt);
\draw[color=xdxdff] (2.07,2.07) node {$E$};
\fill [color=xdxdff] (1.96,-1.14) circle (1.5pt);
\draw[color=xdxdff] (2.06,-0.97) node {$D$};
\end{tikzpicture}}{Mediatriz}{Pmedio}
\end{figura}
\solucion Para demostrar que \'esta proposici\'on es una implicaci\'on vamos a
utilizar el m\'etodo de razonamiento deductivo, de la si\-gui\-ente manera:\\
\begin{figura}{\definecolor{zzttqq}{rgb}{0.6,0.2,0}
\definecolor{cqcqcq}{rgb}{0.75,0.75,0.75}
\begin{tikzpicture}[scale=2.0,font=\fontsize{6}{6}\selectfont,line
cap=round,line join=round,>=triangle 45,x=1.0cm,y=1.0cm,line width=1.2pt]
\clip(-1.11,-1.66) rectangle (5.96,2.71);
\fill[color=zzttqq,fill=zzttqq,fill opacity=0.1] (-1,2) -- (-1,1) -- (1,1) --
(1,2) -- cycle;
\fill[color=zzttqq,fill=zzttqq,fill opacity=0.1] (3,2) -- (3,1) -- (5.3,1) --
(5.31,2) -- cycle;
\fill[color=zzttqq,fill=zzttqq,fill opacity=0.1] (3,0) -- (5.39,0.01) --
(5.39,-1.02) -- (3,-1) -- cycle;
\draw [color=zzttqq] (-1,2)-- (-1,1);
\draw [color=zzttqq] (-1,1)-- (1,1);
\draw [color=zzttqq] (1,1)-- (1,2);
\draw [color=zzttqq] (1,2)-- (-1,2);
\draw [color=zzttqq,->] (1,1)-- (3,0);
\draw (-0.95,1.65) node[anchor=north west] {$\overline{ED}$ es mediatriz de
$\overline{AB}$};
\draw [->] (1,1.48) -- (2.99,1.5);
\draw [color=zzttqq] (3,2)-- (3,1);
\draw [color=zzttqq] (3,1)-- (5.3,1);
\draw [color=zzttqq] (5.3,1)-- (5.31,2.01);
\draw [color=zzttqq] (5.31,2.01)-- (3,2);
\draw (3.05,1.61) node[anchor=north west] {$F$ es el punto medio de
$\overline{AB}$};
\draw [->] (4.21,1) -- (4.22,0.03);
\draw [color=zzttqq] (3,0)-- (5.39,0.01);
\draw [color=zzttqq] (5.39,-1.02)-- (3,-1);
\draw (3.9,-0.4) node[anchor=north west] {$\overline{AF}\cong \overline{FB}$};
\draw (-0.15,2.33) node[anchor=north west] {Dato};
\draw (3.3,2.3) node[anchor=north west] {Definición de mediatrz};
\draw (3.26,-1.15) node[anchor=north west] {Definición de punto medio};
\end{tikzpicture}}{Soluci\'on}{solej}
\end{figura}
La estructura para radactar una demostraci\'on que usaremos  es la
sigui\-ente.
\begin{prueba}{$\overline{ED}\ \mbox{es la mediatriz de} \overline{AB}$}{2.&
$F\ \mbox{es punto medio de} \overline{AB}$ & Definici\'on de punto
mediatriz\\
3. & $AD\cong DB$ & Definici\'on de punto medio\\
}
\end{prueba}
Es decir en este ejemplo  usamos la regla de la cadena.\\
\subsection{Ley modus tollendo-ponens }
Este razonamiento es de la forma:
\begin{center}
 \begin{tabular}{c}
$p\vee q$\\
$\sim p$\\
\hline \\
$q$.\\
 \end{tabular}
\end{center}
\subsection{Ley del silogismo disyuntivo}
Es un razonamiento con la siguiente estructura
\begin{center}
 \begin{tabular}{c}
$p\vee q$\\
$p\longrightarrow r$\\
$q\longrightarrow s$\\
\hline \\
$r\vee s$.\\
 \end{tabular}
\end{center}
\nota\ Existen tres reglas básicas de validez que se aplican continuamente en
una demostración.\\
Regla 1: las definiciones, los postulados y los teoremas demostrados pueden
aparecer en cualquier paso de la demostración.\\
Regla 2: las proposiciones equivalentes se pueden sustituir entre sí en
cualquier parte de una demostración.\\
Regla 3: una proposición verdadera se puede introducir en cualquier punto de la
demostración.\\
\section{Postulados b\'asicos}
\begin{postulado}{El espacio existe y contiene por lo menos 4 puntos no
coplanares. }{Existencia de los puntos}
\end{postulado}
\begin{postulado}{Dos puntos están contenidos en una y sólo una linea
recta.}{Existencia de la recta}
\end{postulado}
\begin{postulado}{Tres puntos no colineales están en uno y sólo un
plano.}{Existencia del plano}
\end{postulado}
\begin{postulado}{Si dos planos se intersecan, se intersecan
exacta en una recta.}{Intersección de los planos}
\end{postulado}
\begin{postulado}{Si dos puntos están en un plano, la recta que los
contiene, también está en el plano.}{Del Plano y la recta}
\end{postulado}
\begin{postulado}{Una recta divide al plano en tres regiones convexas, la
recta y dos semiplanos.}{Separación del plano.}
\end{postulado}
\begin{postulado}{Un plano divide al espacio en tres regiones convexas, el
plano y dos semi-espacios.}{Separación del espacio}
\end{postulado}
\begin{postulado}{Dada una recta y un punto no alineado, existe una y sola
una recta paralela que contenga al punto.}{Existencia de la recta paralela}
\end{postulado}
\begin{postulado}{Dada una recta y un punto no alineado, existe una y sólo
una recta perpendicular que contiene al punto}{Existencia de la recta
perpendicular}
\end{postulado}
\begin{axioma}{
Si $D$ est\'a en el interior del $\angle BAC,$ entonces $m\angle BAC=m\angle
BAD+m\angle DAC.$}{Adici\'on de \'angulos}
\end{axioma}
\begin{figura}{
\definecolor{qqqqff}{rgb}{0.33,0.33,0.33}
\begin{tikzpicture}[font=\fontsize{7}{7}\selectfont,scale=0.8,line
width=1.2pt,line cap=round,line join=round,>=triangle 45,x=1.0cm,y=1.0cm]
\clip(-2.58,-3.14) rectangle (5.42,4.42);
\draw (-1.46,-1.06)-- (0.64,1.9);
\draw (-1.46,-1.06)-- (3.26,-1.86);
\draw (-1.46,-1.06)-- (1.12,-0.24);
\draw [->] (-1.46,-1.06) -- (2.04,3.9);
\draw [->] (-1.46,-1.06) -- (3.5,0.5);
\draw [->] (-1.46,-1.06) -- (4.7,-2.12);
\fill [color=qqqqff] (-1.46,-1.06) circle (1.5pt);
\draw[color=qqqqff] (-1.76,-1.1) node {$A$};
\fill [color=qqqqff] (0.64,1.9) circle (1.5pt);
\draw[color=qqqqff] (0.26,2.12) node {$B$};
\fill [color=qqqqff] (3.26,-1.86) circle (1.5pt);
\draw[color=qqqqff] (3.2,-2.12) node {$C$};
\fill [color=qqqqff] (1.12,-0.24) circle (1.5pt);
\draw[color=qqqqff] (0.9,0.1) node {$D$};
\end{tikzpicture}}{Suma de \'angulos}{sac}
\end{figura}
\section{Nuevas definiciones}
En \'esta secci\'on presentaremos m\'as definiciones para practicar el
razonamiento deductivo.
\begin{definicion}{Dos \'angulos son opuestos por el v\'ertice o verticales
si tienen un v\'ertice com\'un y sus lados son rayos opuestos}{\'Angulos
verticales}
\begin{figura}{
\definecolor{xdxdff}{rgb}{0.49,0.49,1}
\definecolor{qqqqff}{rgb}{0,0,1}
\begin{tikzpicture}[scale=0.9,line cap=round,line join=round,>=triangle
45,x=1.0cm,y=1.0cm,font=\fontsize{7}{7}\selectfont]
\clip(-1.76,-6.54) rectangle (9.78,3.1);
\draw [->] (3.48,-1.64) -- (6.86,2.52);
\draw [->] (6.02,1.49) -- (0.6,-5.3);
\draw [->] (3.48,-1.64) -- (-0.6,1.52);
\draw [->] (0.85,0.4) -- (6.82,-4.16);
\fill [color=qqqqff] (3.48,-1.64) circle (1.5pt);
\draw[color=qqqqff] (3.56,-1.9) node {$A$};
\fill [color=xdxdff] (6.02,1.49) circle (1.5pt);
\draw[color=xdxdff] (5.66,1.8) node {$C$};
\fill [color=xdxdff] (1.74,-3.87) circle (1.5pt);
\draw[color=xdxdff] (1.5,-3.68) node {$B$};
\fill [color=qqqqff] (-0.6,1.52) circle (1.5pt);
\fill [color=xdxdff] (0.85,0.4) circle (1.5pt);
\draw[color=xdxdff] (1.02,0.66) node {$E$};
\fill [color=xdxdff] (5.56,-3.2) circle (1.5pt);
\draw[color=xdxdff] (5.72,-2.94) node {$D$};
\end{tikzpicture}}{\'Angulos opuestos por el v\'artice}{opv}
En la figura(\ref{opv} tenemos que los \'angulos $EAC$ y $BAD$ tienen el mismo
v\'ertice y adem\'as sus lados respectivos $AE$, $AD$ y $AC$, $AB$ son rayos
opuestos, por tanto los \'angulos son verticales.)
\end{figura}
\end{definicion}
\begin{definicion}{Dos \'angulos forman un par lineal si uno de sus lados
son rayos opuestos y adem\'as tienen un v\'ertice y un lado com\'un.
}{Par lineal}
\begin{figura}{
\definecolor{xdxdff}{rgb}{0.49,0.49,1}
\definecolor{qqqqff}{rgb}{0,0,1}
\begin{tikzpicture}[font=\fontsize{7}{7}\selectfont,scale=0.8,line
width=1.2pt,line cap=round,line
join=round,>=triangle
45,x=1.0cm,y=1.0cm]
\clip(-1.42,-3.26) rectangle (10.82,3.4);
\draw [->] (1.16,-1.98) -- (9.96,-2.14);
\draw [->] (4.95,-2.05) -- (6.66,2.18);
\draw [->] (4.95,-2.05) -- (-0.58,-1.98);
\fill [color=qqqqff] (1.16,-1.98) circle (1.5pt);
\draw[color=qqqqff] (1.2,-2.24) node {$A$};
\fill [color=xdxdff] (4.95,-2.05) circle (1.5pt);
\draw[color=xdxdff] (4.92,-2.34) node {$C$};
\fill [color=xdxdff] (7.92,-2.1) circle (1.5pt);
\draw[color=xdxdff] (8.06,-1.84) node {$F$};
\fill [color=xdxdff] (6.1,0.79) circle (1.5pt);
\draw[color=xdxdff] (5.66,0.86) node {$G$};
\end{tikzpicture}}{Par lineal}{parl}
 \end{figura}
\end{definicion}
\begin{teorema}{Los \'angulos que forman un par lineal son suplementarios\label{pro2}}{Teorema del par lineal}
 La demostraci\'on queda de ejercicio.
\end{teorema}
Resuelva las siguientes interrogantes
\begin{enumerate}
\item ?`Es cierto que si dos \'angulos son suplementarios entonces forman un par
lineal? Justifique su respuesta.

\item Si dos \'angulos suplementarios tienen medidas iguales, ?`cu\'al es la medida
de cada \'angulo? Argumente su respuesta.

\item Dos veces la medida de un \'angulo es 30 menos que cinco veces la medida
de su suplemento. Hallar la medida de cada \'angulo.

\item Pruebe que si dos \'angulos forman un par lineal y son congruentes
entonces miden 90 cada uno.
\end{enumerate}
%%%%%%%%%%%%%%%%%%%%%%%%%%%%%DESDE  aqui%%%%%%%%%%%%%%%%%
%%%%%&&&&&&&&&&&&&&&&&&&&&&&&&&&&&&&&&&&&&&&&&&&&&&&&
%%%%%%%%%%%%%&&&&&&&&&&&&&&&&&&&&&&&&&&&&&&&&&&&&&&&
%%%%%%%%%%%%%%%%%%%&&&&&&&&&&&&&&&&&&&&&&&&&&&&&&&&&&&&&
%%%%%%%%%%%%%%%%%%%%&&&&&&&&&&&&&&&&&&&&&&&&&&
\section{Conjeturas y teoremas}
\begin{conjetura}{Los suplementos de \'angulos congruentes son congruentes.}{De
los suplementos}
\end{conjetura}
\begin{prueba}{$\angle A\cong\angle B$ }{
2. & $\angle C$ es el suplemento de $\angle A$ & Dado\\
3. & $\angle D$ es el suplemento de $\angle B$ & Dado\\
4. & $m\angle A+m\angle C=180$ & Definici\'on de angulos suplementarios\\
5. & $m\angle B+m\angle D=180$ & Definici\'on de angulos
suplementarios\\
6. & $m\angle A=m\angle B$ & Definici\'on de congruencia de
\'angulos\\
7. & $m\angle C=180-m\angle A$ & Propiedades de la
igualdad\\
8. & $m\angle D=180-m\angle B$ & Propiedades de la
igualdad\\
9. & $m\angle C=180-m\angle B$ & Sustituci\'on de $\left(
6\right)  $ en $\left(  7\right)  $\\
10. & $m\angle C=m\angle D$ & Sustituci\'on de $\left(
8\right)  $ en $\left(  9\right)  $\\
11. & $\angle C\cong\angle D$ & Definici\'on de congruencia\\}
\end{prueba}%
\begin{teorema}{
Las propiedades reflexiva, sim\'etrica y transitiva son validas en la
congruencia de \'angulos y segmentos\label{pro3}
}{Equivalencia}
\end{teorema}
Probaremos la propiedad transitiva de la congruencia de segmentos, esto es, si
$\overline{AB}\cong\overline{CD}$ y $\overline{CD}\cong\overline{EF}$,
entonces $\overline{AB}\cong\overline{EF}$.
\begin{probar}{Si $\overline{AB}\cong\overline{CD}$ y $\overline{CD}\cong\overline{EF}$, por la
definici\'on de congruencia de segmentos se sigue que $AB=CD$ y $CD=EF$. Ahora al aplicar las
propiedades de la relaci\'on de igualdad, se sabe que \'esta cumple la transitivad. En
consecuencia, $AB=EF$, lo cual implica que $\overline{AB}\cong\overline{EF}$.}
\end{probar}
\begin{axioma}{Si dos \'angulos forman un par lineal, entonces son
suplementarios.}{Par lineal}
\end{axioma}
% \begin{teorema}{
% Todo \'angulo es congruente consigo mismo.}{Propiedad reflexiva}
% \end{teorema}
% La congruencia de \'angulos cumple las propiedades sim\'etrica y transitiva,
% estas conjeturas se dejan de ejercicio.
\begin{teorema}{Dos \'angulos rectos cualesquiera son congruentes.}{\'Angulos
rectos}
\end{teorema}
\begin{teorema}{Si dos \'angulos son a la vez congruentes y suplementarios,
entonces cada uno de ellos es un \'angulo recto.}{\'Angulo recto}
\end{teorema}
\begin{figura}{
\definecolor{qqwuqq}{rgb}{0.13,0.13,0.13}
\definecolor{uququq}{rgb}{0.25,0.25,0.25}
\definecolor{xdxdff}{rgb}{0.66,0.66,0.66}
\definecolor{qqqqff}{rgb}{0.33,0.33,0.33}
\begin{tikzpicture}[line cap=round,line join=round,>=triangle 45,x=1.0cm,y=1.0cm]
\clip(-3.56,-3.04) rectangle (5.82,3.22);
\draw[color=qqwuqq,fill=qqwuqq,fill opacity=0.1] (1.39,0.88) -- (1.78,1.04) -- (1.61,1.43) -- (1.22,1.27) -- cycle; 
\draw [->] (3.64,2.3) -- (-2.58,-0.36);
\draw [->] (-0.94,0.32) -- (5.42,3.08);
\draw [->] (1.22,1.27) -- (2.72,-2.28);
\fill [color=qqqqff] (-0.94,0.32) circle (1.5pt);
\draw[color=qqqqff] (-0.9,0.02) node {$A$};
\fill [color=qqqqff] (3.64,2.3) circle (1.5pt);
\draw[color=qqqqff] (3.56,1.72) node {$B$};
\fill [color=xdxdff] (1.22,1.27) circle (1.5pt);
\draw[color=xdxdff] (1,1.78) node {$C$};
\fill [color=uququq] (2.26,-1.15) circle (1.5pt);
\draw[color=uququq] (2.42,-0.9) node {$D$};
\draw[color=qqwuqq] (2.66,0.98) node {$\alpha = 90\textrm{\degre}$};
\end{tikzpicture}
}{Angulo recto}{angrec}
\begin{probar}{En la figura \ref{angrec} tenemos que $\angle BCD \cong \angle ACD $ y adem\'as cumplen
que $m\angle BCD + m\angle ACD =180 $ y como los \'angulos son congruentes entonces se tiene
que $m\angle BCD = m\angle ACD $ por tanto se tiene que $2m\angle BCD =180 $, entonces 
$m\angle BCD =90=m\angle ACD$}
\end{probar}

\end{figura}
\begin{teorema}{Los complementos de \'angulos congruentes son
congruentes.}{Complementos}
\end{teorema}

\begin{teorema}{Si dos rectas que se cortan forman un \'angulo recto, entonces
forman cuatro \'angulos rectos.}{\'Angulo recto}
\end{teorema}
\begin{teorema}{Si los \'angulos de un par lineal tienen la misma
medida, entonces cada uno de ellos es un \'angulo recto.}{\'Angulo Recto}
\end{teorema}
\begin{teorema}{En un plano, por un punto de una recta dada pasa una y s\'olo
una perpendicular a la recta dada.}{Unicidad de la recta perpendicular}
\end{teorema}
\begin{teorema}{
Un segmento coplanario tiene exactamente una mediatriz.}{Mediatriz}
\end{teorema}
\subsubsection{Clasificaci\'on de los tri\'angulos}
\begin{definicion}{El tri\'angulo es un pol\'igono de tres lados.}{Tri\'angulo}
\end{definicion}
Los tri\'angulos pueden clasificarse por sus \'angulos y por sus lados de las
si\-guien\-tes formas:
\begin{definicion}{Un tri\'angulo es \textbf{acut\'angulo} si todos sus
\'angulos interiores son agudos.}{Tri\'angulo acut\'angulo}
\end{definicion}
\begin{definicion}{Si un tri\'angulo tiene un \'angulo interior recto, entonces
se llama \textbf{tri\'angulo rect\'angulo.}}{Tri\'angulo rect\'angulo}
\end{definicion}
\begin{definicion}{Si un tri\'angulo tiene un \'angulo interior obtuso, entonces
se llama \textbf{tri\'angulo obtus\'angulo}.}{Tri\'angulo obtus\'angulo}
\end{definicion}
\begin{definicion}{Un tri\'angulo es \textbf{equi\'angulo} si todos sus
\'angulos interiores son congruentes entre s\'i.}{Equi\'angulo}
\end{definicion}
\begin{definicion}{Un tri\'angulo es \textbf{equil\'atero} si todos sus lados
son congruentes entre s\'i.}{Equil\'atero}
\end{definicion}
\begin{definicion}{Un tri\'angulo es \textbf{is\'osceles} si al menos dos de sus
lados son congruentes entre s\'i.}{Is\'osceles}
\end{definicion}
\begin{definicion}{Un tri\'angulo \textbf{escaleno}es aquel que no tiene lados
congruentes entre s\'i.}{Escaleno}
\end{definicion}
\subsubsection{Elementos notables de un tri\'angulo}
\begin{definicion}{Una \textbf{altura} de un tri\'angulo es un segmento
contenido en la recta que pasa por  un v\'ertice del tri\'angulo y es
perpendicular la recta que contiene al lado opuesto y que tiene como
extremos el v\'ertice y el punto de intersecci\'on de las rectas }{Altura}
\end{definicion}
\begin{definicion}{El punto donde se intersecan las alturas de un tri\'angulo se
denomina \textbf{ortocentro.}}{Ortocentro}
\begin{figura}{
\definecolor{qqwuqq}{rgb}{0.13,0.13,0.13}
\definecolor{uququq}{rgb}{0.25,0.25,0.25}
\definecolor{zzttqq}{rgb}{0.27,0.27,0.27}
\definecolor{qqqqff}{rgb}{0.33,0.33,0.33}
\begin{tikzpicture}[line cap=round,line join=round,>=triangle
45,x=1.0cm,y=1.0cm]
\clip(-2.2,-1.94) rectangle (6.22,5.76);
\draw[color=qqwuqq,fill=qqwuqq,fill opacity=0.1] (1.76,3.85) -- (1.42,3.59) --
(1.68,3.25) -- (2.02,3.51) -- cycle;
\draw[color=qqwuqq,fill=qqwuqq,fill opacity=0.1] (-0.08,2.42) -- (0.12,2.8) --
(-0.26,2.99) -- (-0.46,2.62) -- cycle;
\draw[color=qqwuqq,fill=qqwuqq,fill opacity=0.1] (0.11,1.19) -- (-0.31,1.27) --
(-0.39,0.85) -- (0.03,0.77) -- cycle;
\draw (0.84,5.08)-- (-1.3,1.02);
\draw (-1.3,1.02)-- (4.74,-0.12);
\draw (4.74,-0.12)-- (0.84,5.08);
\draw [dotted] (1.99,1.88) circle (3.4cm);
\draw [domain=-2.2:6.22] plot(\x,{(--0.72--6.04*\x)/1.14});
\draw [domain=-2.2:6.22] plot(\x,{(-10.37-3.9*\x)/-5.2});
\draw [domain=-2.2:6.22] plot(\x,{(--9.66-2.14*\x)/4.06});
\draw [dotted] (1.15,2.05) circle (1.7cm);
\fill [color=qqqqff] (0.84,5.08) circle (1.5pt);
\draw[color=qqqqff] (0.64,5.32) node {$A$};
\fill [color=zzttqq] (-1.3,1.02) circle (1.5pt);
\draw[color=zzttqq] (-1.5,1.18) node {$B$};
\fill [color=zzttqq] (4.74,-0.12) circle (1.5pt);
\draw[color=zzttqq] (4.9,-0.36) node {$C$};
\draw[color=black] (1.26,6.14) node {$a$};
\draw[color=black] (-4.12,-0.76) node {$b$};
\draw[color=black] (-4.14,4.36) node {$c$};
\fill [color=uququq] (0.3,2.22) circle (1.5pt);
\draw[color=uququq] (0.66,2.32) node {$D$};
\fill [color=uququq] (2.02,3.51) circle (1.5pt);
\draw[color=uququq] (2.04,3.94) node {$I$};
\fill [color=uququq] (-0.46,2.62) circle (1.5pt);
\draw[color=uququq] (-0.6,3.12) node {$O$};
\fill [color=uququq] (0.03,0.77) circle (1.5pt);
\draw[color=uququq] (-0.26,0.58) node {$T$};
\end{tikzpicture}}{Ortocentro}{ort}
\end{figura}
En la figura (\ref{ort}) El punto $D$ es el ortocentro.
\end{definicion}
\begin{definicion}{El punto de intersecci\'on de las mediatrices de los lados de
un tri\'angulo se denomina \textbf{circuncentro.}}{Circuncentro}
\begin{figura}{
\definecolor{qqwuqq}{rgb}{0.13,0.13,0.13}
\definecolor{uququq}{rgb}{0.25,0.25,0.25}
\definecolor{cccccc}{rgb}{0.8,0.8,0.8}
\definecolor{zzttqq}{rgb}{0.27,0.27,0.27}
\definecolor{qqqqff}{rgb}{0.33,0.33,0.33}
\begin{tikzpicture}[line cap=round,line join=round,>=triangle
45,x=1.0cm,y=1.0cm]
\clip(-2.6,-1.54) rectangle (6.64,5.94);
\fill[dotted,color=cccccc,fill=cccccc,fill opacity=0.1] (1.14,5.06) -- (-1,1) --
(4.74,-0.12) -- cycle;
\draw[color=qqwuqq,fill=qqwuqq,fill opacity=0.1] (0.45,2.83) -- (0.64,3.21) --
(0.27,3.41) -- (0.07,3.03) -- cycle;
\draw[color=qqwuqq,fill=qqwuqq,fill opacity=0.1] (2.7,2.82) -- (2.35,2.58) --
(2.59,2.23) -- (2.94,2.47) -- cycle;
\draw[color=qqwuqq,fill=qqwuqq,fill opacity=0.1] (2.29,0.36) -- (2.37,0.78) --
(1.95,0.86) -- (1.87,0.44) -- cycle;
\draw [dotted,color=cccccc] (1.14,5.06)-- (-1,1);
\draw [dotted,color=cccccc] (-1,1)-- (4.74,-0.12);
\draw [dotted,color=cccccc] (4.74,-0.12)-- (1.14,5.06);
\draw (1.14,5.06)-- (-1,1);
\draw (-1,1)-- (4.74,-0.12);
\draw (4.74,-0.12)-- (1.14,5.06);
\draw [domain=-2.6:6.64] plot(\x,{(-10.24--5.74*\x)/1.12});
\draw [domain=-2.6:6.64] plot(\x,{(-2.21-3.6*\x)/-5.18});
\draw [domain=-2.6:6.64] plot(\x,{(--12.45-2.14*\x)/4.06});
\draw (0.22,3.88)-- (0.92,3.92);
\draw (-0.88,1.78)-- (-0.08,1.8);
\draw(2.16,1.93) circle (3.29cm);
\draw [dash pattern=on 2pt off 2pt] (2.16,1.93)-- (1.14,5.06);
\draw [dash pattern=on 2pt off 2pt] (2.16,1.93)-- (-1,1);
\draw [dash pattern=on 2pt off 2pt] (2.16,1.93)-- (4.74,-0.12);
\draw [dotted] (1.36,2.01) circle (1.65cm);
\fill [color=qqqqff] (1.14,5.06) circle (1.5pt);
\draw[color=qqqqff] (1.3,5.32) node {$A$};
\fill [color=zzttqq] (-1,1) circle (1.5pt);
\draw[color=zzttqq] (-1.2,1.16) node {$B$};
\fill [color=zzttqq] (4.74,-0.12) circle (1.5pt);
\draw[color=zzttqq] (4.9,-0.36) node {$C$};
\draw[color=black] (3.2,6.14) node {$d$};
\draw[color=black] (-3.52,-2.16) node {$e$};
\draw[color=black] (-4.16,5.06) node {$f$};
\fill [color=uququq] (2.16,1.93) circle (1.5pt);
\draw[color=uququq] (2.7,2.02) node {$D$};
\fill [color=uququq] (2.94,2.47) circle (1.5pt);
\draw[color=uququq] (3.34,2.56) node {$J$};
\fill [color=uququq] (1.87,0.44) circle (1.5pt);
\draw[color=uququq] (2.1,0.04) node {$H$};
\fill [color=uququq] (0.07,3.03) circle (1.5pt);
\draw[color=uququq] (-0.32,2.96) node {$K$};
\end{tikzpicture}}{Circuncentro}{circ}
 En la figura(\ref{circ}) el punto de intersecci\'on de las mediatrices es $D$
y es el centro de la circunferencia que contiene los v\'ertices del
tri\'angulos.
\end{figura}
\begin{definicion}{El punto donde se intersecan las bisectrices de los \'angulos
interiores de un tri\'angulo se denomina\textbf{ incentro.}}{Incentro}
\begin{figura}{
\definecolor{uququq}{rgb}{0.25,0.25,0.25}
\definecolor{zzttqq}{rgb}{0.27,0.27,0.27}
\definecolor{qqqqff}{rgb}{0.33,0.33,0.33}
\begin{tikzpicture}[line cap=round,line join=round,>=triangle
45,x=1.0cm,y=1.0cm]
\clip(-2.5,-1.82) rectangle (6.42,5.58);
\draw (0.84,5.08)-- (-1.3,1.02);
\draw (-1.3,1.02)-- (4.74,-0.12);
\draw (4.74,-0.12)-- (0.84,5.08);
\draw [dotted] (1.99,1.88) circle (3.4cm);
\draw [domain=-2.5:6.42] plot(\x,{(--1.24-1*\x)/0.08});
\draw [domain=-2.5:6.42] plot(\x,{(-2.4--0.53*\x)/-0.85});
\draw [domain=-2.5:6.42] plot(\x,{(--1.48--0.43*\x)/0.9});
\draw [dotted] (1.08,2.14) circle (1.59cm);
\draw [shift={(0.84,5.08)},color=zzttqq,fill=zzttqq,fill opacity=0.1]  (0,0) --
plot[domain=4.23:5.36,variable=\t]({1*1.04*cos(\t r)+0*1.04*sin(\t
r)},{0*1.04*cos(\t r)+1*1.04*sin(\t r)}) -- cycle ;
\draw (0.64,4.34)-- (0.56,3.78);
\draw (1.06,4.44)-- (1.34,3.92);
\fill [color=qqqqff] (0.84,5.08) circle (1.5pt);
\draw[color=qqqqff] (0.64,5.32) node {$A$};
\fill [color=zzttqq] (-1.3,1.02) circle (1.5pt);
\draw[color=zzttqq] (-1.5,1.18) node {$B$};
\fill [color=zzttqq] (4.74,-0.12) circle (1.5pt);
\draw[color=zzttqq] (4.9,-0.36) node {$C$};
\draw[color=black] (0.98,6.14) node {$a$};
\draw[color=black] (-4.12,5.22) node {$b$};
\draw[color=black] (-4.14,-0.02) node {$c$};
\fill [color=uququq] (1.07,2.16) circle (1.5pt);
\draw[color=uququq] (1.34,2.68) node {$D$};
\fill [color=uququq] (2.51,2.86) circle (1.5pt);
\draw[color=uququq] (2.6,3.12) node {$I$};
\fill [color=uququq] (1.07,2.16) circle (1.5pt);
%\draw[color=uququq] (1.24,2.42) node {$O$};
\fill [color=uququq] (1.2,0.55) circle (1.5pt);
\draw[color=uququq] (1.34,0.8) node {$T$};
\fill [color=uququq] (-0.26,2.99) circle (1.5pt);
\draw[color=uququq] (-0.7,2.74) node {$U$};
\end{tikzpicture}}{Incentro}{incen}
\end{figura}
El punto de la figura(\ref{incen}) es el punto $D$.
\end{definicion}
\end{definicion}
\begin{definicion}{Una \textbf{mediana} de un tri\'angulo es un segmento que une
un v\'ertice del tri\'angulo con el punto medio del lado opuesto}{Mediana}
\end{definicion}
%
\begin{definicion}{El \textbf{baricentro o centroide }es el punto de
intersecci\'on de las medianas de un tri\'angulo.}{Centroide}
\begin{figura}{
\definecolor{uququq}{rgb}{0.25,0.25,0.25}
\definecolor{zzttqq}{rgb}{0.27,0.27,0.27}
\definecolor{qqqqff}{rgb}{0.33,0.33,0.33}
\begin{tikzpicture}[line cap=round,line join=round,>=triangle
45,x=1.0cm,y=1.0cm]
\clip(-2.48,-2) rectangle (6.8,5.72);
\draw (0.84,5.08)-- (-1.3,1.02);
\draw (-1.3,1.02)-- (4.74,-0.12);
\draw (4.74,-0.12)-- (0.84,5.08);
\draw (0.84,5.08)-- (1.72,0.45);
\draw (-0.23,3.05)-- (4.74,-0.12);
\draw (2.79,2.48)-- (-1.3,1.02);
\draw [dotted] (1.99,1.88) circle (3.4cm);
\draw [dotted] (1.15,2.05) circle (1.7cm);
\draw (0.04,4.22)-- (0.58,3.96);
\draw (-0.96,2.38)-- (-0.4,2.14);
\fill [color=qqqqff] (0.84,5.08) circle (1.5pt);
\draw[color=qqqqff] (1,5.34) node {$A$};
\fill [color=zzttqq] (-1.3,1.02) circle (1.5pt);
\draw[color=zzttqq] (-1.5,1.18) node {$B$};
\fill [color=zzttqq] (4.74,-0.12) circle (1.5pt);
\draw[color=zzttqq] (4.9,-0.36) node {$C$};
\fill [color=uququq] (1.72,0.45) circle (1.5pt);
\draw[color=uququq] (1.6,0.14) node {$T$};
\fill [color=uququq] (2.79,2.48) circle (1.5pt);
\draw[color=uququq] (2.9,2.74) node {$I$};
\fill [color=uququq] (-0.23,3.05) circle (1.5pt);
\draw[color=uququq] (-0.56,3.58) node {$O$};
\end{tikzpicture}}{Centroide}{centri}
 \end{figura}
\end{definicion}

%%%%%%%%%%%%%%%%%%%%%%%%%%%%%Hasta aqui%%%%%%%%%%%%%%%%%
%%%%%&&&&&&&&&&&&&&&&&&&&&&&&&&&&&&&&&&&&&&&&&&&&&&&&
%%%%%%%%%%%%%&&&&&&&&&&&&&&&&&&&&&&&&&&&&&&&&&&&&&&&
%%%%%%%%%%%%%%%%%%%&&&&&&&&&&&&&&&&&&&&&&&&&&&&&&&&&&&&&
%%%%%%%%%%%%%%%%%%%%&&&&&&&&&&&&&&&&&&&&&&&&&&

\subsection{Relaciones de \'angulos}

\begin{conjetura}{Si dos \'angulos forman un par lineal, entonces son
suplementarios }{Par lineal}
 \end{conjetura}

 \begin{prueba}{Los \'angulos $\measuredangle ABC$ y $\measuredangle DBC$\
 forman un par lineal}{
 2.& $m \measuredangle ABD = 180$& Definici\'on del ángulo llano.\\
3.& $ m\measuredangle ABC = x$ y $ m\measuredangle DBC = y$ & Postulado del
 transportador\\
4.& $ m\measuredangle ABC + m\measuredangle DBC =m \measuredangle ABD = 180$ &
 Construcci\'on\\
5.& Los \'angulos $\measuredangle ABC$ y $\measuredangle DBC$ son
 suplementarios& De 2), 3), 4) y definici\'on de \'angulos suplementarios \\
}
\end{prueba}
\begin{teorema}{Si dos \'angulos son congruentes sus  suplementos son
congruentes}{Suplementos}
\begin{figura}{
\definecolor{uququq}{rgb}{0.25,0.25,0.25}
\definecolor{qqwuqq}{rgb}{0.13,0.13,0.13}
\definecolor{xdxdff}{rgb}{0.66,0.66,0.66}
\definecolor{qqqqff}{rgb}{0.33,0.33,0.33}
\begin{tikzpicture}[scale=0.7,font=\fontsize{7}{6}\selectfont,line
cap=round,line join=round,>=triangle 45,x=1.0cm,y=1.0cm,line width=1.2pt]
\clip(-3.54,-1.32) rectangle (12.54,5.02);
\draw [shift={(0.94,0.55)},color=qqwuqq,fill=qqwuqq,fill opacity=0.1] (0,0) --
(122.89:0.6) arc (122.89:179.47:0.6) -- cycle;
\draw [shift={(7.24,2.94)},color=qqwuqq,fill=qqwuqq,fill opacity=0.1] (0,0) --
(-178.41:0.6) arc (-178.41:-121.83:0.6) -- cycle;
\draw (-2.48,0.58)-- (4,0.52);
\draw (-0.84,3.3)-- (0.94,0.55);
\draw (4.36,2.86)-- (11.24,3.06);
\draw (7.24,2.94)-- (5.72,0.49);
\draw (0.24,0.94)-- (0.68,0.68);
\draw (6.96,2.8)-- (6.56,2.48);
\fill [color=qqqqff] (-2.48,0.58) circle (1.5pt);
\draw[color=qqqqff] (-2.32,0.84) node {$A$};
\fill [color=qqqqff] (4,0.52) circle (1.5pt);
\draw[color=qqqqff] (4.16,0.78) node {$B$};
\fill [color=qqqqff] (-0.84,3.3) circle (1.5pt);
\draw[color=qqqqff] (-0.68,3.56) node {$C$};
\fill [color=xdxdff] (0.94,0.55) circle (1.5pt);
\draw[color=xdxdff] (0.92,0.32) node {$D$};
\fill [color=qqqqff] (4.36,2.86) circle (1.5pt);
\draw[color=qqqqff] (4.52,3.12) node {$E$};
\fill [color=qqqqff] (7.24,2.94) circle (1.5pt);
\draw[color=qqqqff] (7.38,3.2) node {$F$};
\fill [color=uququq] (5.72,0.49) circle (1.5pt);
\draw[color=uququq] (5.34,0.52) node {$H$};
\fill [color=qqqqff] (11.24,3.06) circle (1.5pt);
\draw[color=qqqqff] (11.4,3.32) node {$G$};
\end{tikzpicture}}{Complementos}{acc}
En la figura (\ref{acc}) los \'angulos $\angle CDB$ y $\angle HFG$ son
congruentes. 
\end{figura}

\end{teorema}
\begin{teorema}{Si dos \'angulos son opuestos por el v\'ertice, entonces
son congruentes.}{Opuestos por el v\'ertice}
\begin{figura}{
\definecolor{qqwuqq}{rgb}{0.13,0.13,0.13}
\definecolor{uququq}{rgb}{0.25,0.25,0.25}
\definecolor{qqqqff}{rgb}{0.33,0.33,0.33}
\begin{tikzpicture}[scale=0.7,font=\fontsize{7}{6}\selectfont,line
cap=round,line join=round,>=triangle 45,x=1.0cm,y=1.0cm,line width=1.2pt]
\clip(-1.7,-2.08) rectangle (5.8,3.64);
\draw [shift={(2.44,1.03)},color=qqwuqq,fill=qqwuqq,fill opacity=0.1] (0,0) --
(139.24:0.6) arc (139.24:228.64:0.6) -- cycle;
\draw [shift={(2.44,1.03)},color=qqwuqq,fill=qqwuqq,fill opacity=0.1] (0,0) --
(-40.76:0.6) arc (-40.76:48.64:0.6) -- cycle;
\draw (3.68,2.44)-- (0.88,-0.74);
\draw (1.16,2.14)-- (4.64,-0.86);
\fill [color=qqqqff] (3.68,2.44) circle (1.5pt);
\draw[color=qqqqff] (3.84,2.7) node {$A$};
\fill [color=qqqqff] (0.88,-0.74) circle (1.5pt);
\draw[color=qqqqff] (0.68,-0.96) node {$B$};
\fill [color=qqqqff] (1.16,2.14) circle (1.5pt);
\draw[color=qqqqff] (1.32,2.4) node {$C$};
\fill [color=qqqqff] (4.64,-0.86) circle (1.5pt);
\draw[color=qqqqff] (4.8,-0.6) node {$D$};
\fill [color=uququq] (2.44,1.03) circle (1.5pt);
\draw[color=uququq] (2.42,0.7) node {$E$};
\end{tikzpicture}}{Opuestos por el v\'ertice}{opv}
En la figura(opv) se observa claramente que si aplicamos la definici\'on de
suma de \'angulos y la de par lineal y luego aplicamos el teorema de los
\'angulos suplementarios nos que demostrado el teorema.
\end{figura}

\end{teorema}
\begin{teorema}{Si dos \'angulos son congruentes sus  complementos son
congruentes}{Complementos}
\begin{figura}{
\definecolor{qqwuqq}{rgb}{0.13,0.13,0.13}
\definecolor{cccccc}{rgb}{0.8,0.8,0.8}
\definecolor{uququq}{rgb}{0.25,0.25,0.25}
\definecolor{qqqqff}{rgb}{0.33,0.33,0.33}
\begin{tikzpicture}[scale=0.7,font=\fontsize{7}{6}\selectfont,line
cap=round,line join=round,>=triangle 45,x=1.0cm,y=1.0cm,line width=1.2pt]
\clip(-2.88,-8.62) rectangle (10.36,-1.94);
\draw[color=cccccc,fill=cccccc,fill opacity=0.1] (1.8,-3.37) -- (1.79,-3.8) --
(2.21,-3.8) -- (2.22,-3.38) -- cycle;
\draw [shift={(2.22,-3.38)},color=qqwuqq,fill=qqwuqq,fill opacity=0.1] (0,0) --
(179.22:0.6) arc (179.22:216.98:0.6) -- cycle;
\draw [shift={(8.48,-3.18)},color=qqwuqq,fill=qqwuqq,fill opacity=0.1] (0,0) --
(180:0.6) arc (180:217.77:0.6) -- cycle;
\draw[color=cccccc,fill=cccccc,fill opacity=0.1] (8.06,-3.18) -- (8.06,-3.6) --
(8.48,-3.6) -- (8.48,-3.18) -- cycle;
\draw (-2.16,-3.32)-- (2.22,-3.38);
\draw (-1.1,-5.88)-- (2.22,-3.38);
\draw (2.22,-3.38)-- (2.16,-7.76);
\draw (4.76,-3.18)-- (8.48,-3.18);
\draw (8.48,-3.18)-- (5.54,-5.46);
\draw (8.48,-3.18)-- (8.48,-6.9);
\fill [color=qqqqff] (-2.16,-3.32) circle (1.5pt);
\draw[color=qqqqff] (-2,-3.06) node {$A$};
\fill [color=qqqqff] (2.22,-3.38) circle (1.5pt);
\draw[color=qqqqff] (2.38,-3.12) node {$B$};
\fill [color=qqqqff] (-1.1,-5.88) circle (1.5pt);
\draw[color=qqqqff] (-1.36,-5.94) node {$C$};
\fill [color=uququq] (2.16,-7.76) circle (1.5pt);
\draw[color=uququq] (2.32,-7.5) node {$D$};
\fill [color=qqqqff] (4.76,-3.18) circle (1.5pt);
\draw[color=qqqqff] (4.92,-2.92) node {$E$};
\fill [color=qqqqff] (8.48,-3.18) circle (1.5pt);
\draw[color=qqqqff] (8.62,-2.92) node {$F$};
\fill [color=uququq] (5.54,-5.46) circle (1.5pt);
\draw[color=uququq] (5.2,-5.12) node {$G$};
\fill [color=uququq] (8.48,-6.9) circle (1.5pt);
\draw[color=uququq] (8.78,-6.86) node {$H$};
\end{tikzpicture}}{\'Angulos complementarios}{acc}
 En la figura(\ref{acc}) observamos que si los \'angulos $\angle ABC $ y
$\angle EFG$ son congruentes, los \'angulos $\angle CBD $ y $\angle
HFG$ deben ser congruentes. Utilizando las definiciones de \'angulos
complementarios y suma de \'angulos.
\end{figura}

\end{teorema}
\begin{conjetura}{La medida del \'angulo exterior de un tri\'angulo es mayor
que la medida de cualquiera de los \'angulos interiores no contiguos}{Teorema
del \'angulo exterior}
Replantearemos la conjetura del \'angulo exterior de la siguiente forma:\\
Dado el \triangulo $ABC$ como se muestra en la figura(\ref{tae}) con
\'angulo exterior, $\angle 1$. demuestre que $m\angle 1 > m\angle 2.$
\begin{figura}{
\definecolor{uququq}{rgb}{0.25,0.25,0.25}
\definecolor{qqqqff}{rgb}{0.33,0.33,0.33}
\begin{tikzpicture}[scale=1.2,font=\fontsize{7}{6}\selectfont, line
cap=round,line join=round,>=triangle 45,x=1.0cm,y=1.0cm,line width=1.2pt]
\clip(-1.41,-2.27) rectangle (2.87,1.38);
\draw (-0.88,-0.37)-- (-0.1,1.09);
\draw (-0.1,1.09)-- (1.33,-0.41);
\draw (-0.88,-0.37)-- (2.48,-0.43);
\draw [shift={(1.33,-0.41)}] plot[domain=-0.01:2.33,variable=\t]({1*0.28*cos(\t
r)+0*0.28*sin(\t r)},{0*0.28*cos(\t r)+1*0.28*sin(\t r)});
\draw [shift={(1.33,-0.41)}] plot[domain=3.12:6.27,variable=\t]({1*0.44*cos(\t
r)+0*0.44*sin(\t r)},{0*0.44*cos(\t r)+1*0.44*sin(\t r)});
\draw [line width=1.2pt,dash pattern=on 1pt off 1pt] (-0.1,1.09)-- (0.57,-1.86);
\draw [line width=1.2pt,dash pattern=on 1pt off 1pt] (0.57,-1.86)--
(1.33,-0.41);
\draw (-0.1,1.09)-- (2.19,-1.3);
\draw [shift={(-0.88,-0.37)}] plot[domain=-0.02:1.08,variable=\t]({1*0.31*cos(\t
r)+0*0.31*sin(\t r)},{0*0.31*cos(\t r)+1*0.31*sin(\t r)});
\draw (1.46,0.05) node[anchor=north west] {1};
\draw (-0.56,-0.04) node[anchor=north west] {2};
\draw (1.32,-0.89) node[anchor=north west] {3};
\draw (-0.01,0.23)-- (0.19,0.25);
\draw (0.23,-0.94)-- (0.46,-0.88);
\fill [color=qqqqff] (-0.88,-0.37) circle (1.5pt);
\draw[color=qqqqff] (-1,-0.42) node {$A$};
\fill [color=qqqqff] (-0.1,1.09) circle (1.5pt);
\draw[color=qqqqff] (-0.03,1.19) node {$B$};
\fill [color=qqqqff] (1.33,-0.41) circle (1.5pt);
\draw[color=qqqqff] (1.09,-0.29) node {$C$};
\fill [color=uququq] (0.23,-0.39) circle (1.5pt);
\draw[color=uququq] (0.14,-0.47) node {$M$};
\fill [color=qqqqff] (0.57,-1.86) circle (1.5pt);
\draw[color=qqqqff] (0.54,-1.97) node {$D$};
\fill [color=qqqqff] (2.19,-1.3) circle (1.5pt);
\draw[color=qqqqff] (2.25,-1.19) node {$E$};
\end{tikzpicture}}{Teorema del \'angulo exterior}{tae}
\begin{prueba}{$\angle 1$ es un \'angulo exterior del $\triangulo ABC.$}{
2.& Sea $M$ el punto medio de $\overline{AC}$ & Construcci\'on
auxiliar.\\
3.& Sea $\overline{BD}$ un segmento tal que $M$ sea su punto medio. &
Construcci\'on auxiliar.\\
4.& $\overline{AM}\cong \overline{MC}$ y $\overline{BM}\cong \overline{MD}$ &
Definici\'on de punto medio.\\
5. & $\angle BMA \cong \angle DMC $ & \'Angulos opuestos por el v\'ertice.\\
6. & $\triangulo ABM \cong \triangulo CMD$ & LAL de 4) y 5).\\
7. & $\angle 2 \cong \angle MCD$ & PCTCC\\
8.& $m\angle 2 = m\angle MCD$ & Definici\'on de \'angulos congruentes.\\
9. & $m\angle MCD + m\angle DCE = m\angle 3$ & Definici\'on de suma de
\'angulos.\\
10.& $m\angle 2 + m\angle DCE = m\angle 3$ & Principio de sustituci\'on.\\
11. & $m\angle 3 > m\angle 2 $ & Definici\'on de mayor que.\\
12. & $\angle 3 \cong \angle 1$ & Opuestos por el v\'ertice.\\
13. & $m\angle 3 = m\angle 1$ & Definici\'on de congruencia de \'angulos\\
14. & $m\angle 1 > m\angle 2$ & Principio de sustituci\'on de 11) y 13).\\
}
\end{prueba}
\end{figura}

\end{conjetura}
\nota: Observe que en esta prueba se utiliza la principio de sustituci\'on del
\'algebra y el principio de adici\'on de premisa de acuerdo con el ítem 4 del
ejemplo 2.10, en los pasos 2) y 3).
\begin{definicion}{Se dice que una recta  es transversal o secante si interseca
a otras dos en puntos diferentes }{Recta transversal}
\end{definicion}
\subsection{\'Angulos formados por una transversal}
\begin{lista}
\item Los \'angulos alternos internos son dos \'angulos interiores con
diferentes v\'ertices y en lados diferentes de la secante.
\item Los \'angulos conjugados internos son dos \'angulos internos con
diferentes v\'ertices y del mismo lados de la secante.
\item Loa \'angulos correspondientes son dos \'angulos con diferentes
v\'ertices del mismo lado de la secante pero, uno exterior y el otro interior.
\end{lista}
\begin{figura}{
\definecolor{uququq}{rgb}{0.25,0.25,0.25}
\begin{tikzpicture}[scale=1.2,font=\fontsize{7}{6}\selectfont,line
cap=round,line join=round,>=triangle 45,x=1.0cm,y=1.0cm,line width=0.8pt]
\clip(-1,-6.74) rectangle (11.34,1.52);
\draw (-0.48,-4.98)-- (10.9,-2.74);
\draw (0.2,-0.16)-- (10.2,-1.66);
\draw (6.54,1.48)-- (1.56,-6.54);
\draw [shift={(5.07,-0.89)}] plot[domain=-0.15:1.02,variable=\t]({1*0.77*cos(\t
r)+0*0.77*sin(\t r)},{0*0.77*cos(\t r)+1*0.77*sin(\t r)});
\draw [shift={(5.07,-0.89)}] plot[domain=2.99:4.16,variable=\t]({1*1.1*cos(\t
r)+0*1.1*sin(\t r)},{0*1.1*cos(\t r)+1*1.1*sin(\t r)});
\draw [shift={(5.07,-0.89)}] plot[domain=1.02:2.99,variable=\t]({1*0.75*cos(\t
r)+0*0.75*sin(\t r)},{0*0.75*cos(\t r)+1*0.75*sin(\t r)});
\draw [shift={(5.07,-0.89)}] plot[domain=4.16:6.13,variable=\t]({1*0.53*cos(\t
r)+0*0.53*sin(\t r)},{0*0.53*cos(\t r)+1*0.53*sin(\t r)});
\draw [shift={(2.95,-4.31)}] plot[domain=1.02:3.34,variable=\t]({1*1.09*cos(\t
r)+0*1.09*sin(\t r)},{0*1.09*cos(\t r)+1*1.09*sin(\t r)});
\draw [shift={(2.95,-4.31)}] plot[domain=3.34:4.16,variable=\t]({1*1.01*cos(\t
r)+0*1.01*sin(\t r)},{0*1.01*cos(\t r)+1*1.01*sin(\t r)});
\draw [shift={(2.95,-4.31)}] plot[domain=0.19:1.02,variable=\t]({1*0.89*cos(\t
r)+0*0.89*sin(\t r)},{0*0.89*cos(\t r)+1*0.89*sin(\t r)});
\draw [shift={(2.95,-4.31)}] plot[domain=-2.13:0.19,variable=\t]({1*0.46*cos(\t
r)+0*0.46*sin(\t r)},{0*0.46*cos(\t r)+1*0.46*sin(\t r)});
\draw (5.96,-0.14) node[anchor=north west] {1};
\draw (4.68,0.26) node[anchor=north west] {2};
\draw (3.72,-1.14) node[anchor=north west] {3};
\draw (3.84,-3.42) node[anchor=north west] {5};
\draw (5.28,-1.46) node[anchor=north west] {4};
\draw (2.06,-3.02) node[anchor=north west] {6};
\draw (1.88,-4.96) node[anchor=north west] {7};
\draw (3.28,-4.76) node[anchor=north west] {8};
\fill [color=uququq] (5.07,-0.89) circle (1.5pt);
\draw[color=uququq] (4.88,-0.5) node {$G$};
\fill [color=uququq] (2.95,-4.31) circle (1.5pt);
\draw[color=uququq] (2.7,-3.98) node {$O$};
\end{tikzpicture}}{Recta secante}{afs}
Por ejemplo los ángulos 4 y 6 son alternos internos, los \'angulos 3 y 6 son
conjugados internos, y los \'angulos 1 y 5 son correspondientes.
\end{figura}


\subsection{\'Angulos especiales entre rectas paralelas}

\begin{conjetura}{ Dos rectas cortadas por una recta transversal son
paralelas si y s\'olo si sus \'angulos  correspondientes son
congruentes}{\'Angulos correspondientes (AC)}
Para realizar la prueba del reciproco de la conjetura usaremos el m\'etodo
reducci\'on al absurdo o de contradicci\'on, de la siguiente forma: \\
Suponemos que existen dos rectas $l_1$ y $l_2$ que no son paralelas y son
cortadas por una secante $l_3$ y adem\'as que $\angle 1 \cong \angle 2$\\
Para la prueba observa la figura (\ref{acep})
\begin{prueba}{$l_1 \nparallel l_2$}{
2. & $\angle 1 \cong \angle 2$ & Dato.\\
3. & Como $l_1 \nparallel l_2$, entonces se cortan en el punto $C$ &
Definici\'on de rectas paralelas\\
4. & Entre los puntos $A,B$ y $C$ se forma un tri\'angulo & Defnici\'on de
pol\'igono.\\
5. & El $\angle 2$ es un \'angulo exterior del $\triangulo ABC$ & Definici\'on
del \'angulo exterior.\\
6. & $m\angle 2 > m\angle 1.$ & Teorema del \'angulo exterior.\\
7. & $m\angle 2 = m\angle 1.$ contradicci\'on de 6). & Definici\'on de \'angulos
congruentes de 2).\\
8. & $l_1 \parallel l_2$ & Conclusi\'on de la prueba indirecta\\
}
\end{prueba}
Con lo anterior demostramos que:\\
Si las rectas $l_1 , l_2$ y $l_3$ forman \'angulos correspondientes $\angle 1
\cong \angle 2$, entonces las rectas $l_1 \parallel l_2$. \\
Nos queda probar que sia las rectas $l_1 \parallel l_2$ entonces se tiene que
$\angle 1
\cong \angle 2.$\\
Ahora para probar la expresi\'on directa de la conjetura usaremos el mismo
m\'etodo de la siguiente forma:\\
Suponemos que $\angle 1 \ncong \angle 2$ y que $l_1 \parallel l_2$.
\end{conjetura}
\begin{figura}{}{ACEP}{acep}
\centering
\definecolor{uququq}{rgb}{0.25,0.25,0.25}
\begin{tikzpicture}[scale=0.7,font=\fontsize{7}{6}\selectfont,line
cap=round,line join=round,>=triangle 45,x=1.0cm,y=1.0cm,line width=0.8pt]
\clip(-0.74,-6.92) rectangle (11.66,2.04);
\draw (-0.48,-4.98)-- (10.9,-2.74);
\draw (0.2,-0.16)-- (10.26,-4.64);
\draw (6.54,1.48)-- (1.56,-6.54);
\draw [shift={(4.37,-2.02)}] plot[domain=-0.42:1.02,variable=\t]({1*0.84*cos(\t
r)+0*0.84*sin(\t r)},{0*0.84*cos(\t r)+1*0.84*sin(\t r)});
\draw [shift={(2.95,-4.31)}] plot[domain=0.19:1.02,variable=\t]({1*0.89*cos(\t
r)+0*0.89*sin(\t r)},{0*0.89*cos(\t r)+1*0.89*sin(\t r)});
\draw (3.8,-3.36) node[anchor=north west] {1};
\draw (5.32,-1.38) node[anchor=north west] {2};
\draw (10.5,-4.34) node[anchor=north west] {$l_1$};
\draw (11.2,-2.34) node[anchor=north west] {$l_2$};
\draw (6.42,1.02) node[anchor=north west] {$l_3$};
\fill [color=uququq] (4.37,-2.02) circle (1.5pt);
\draw[color=uququq] (4.18,-1.62) node {$A$};
\fill [color=uququq] (2.95,-4.31) circle (1.5pt);
\draw[color=uququq] (2.7,-3.98) node {$B$};
\fill [color=uququq] (7.5,-3.41) circle (1.5pt);
\draw[color=uququq] (7.66,-3.14) node {$C$};
\end{tikzpicture}
\end{figura}

\begin{figura}{
\definecolor{uququq}{rgb}{0.25,0.25,0.25}
\begin{tikzpicture}[scale=0.7,font=\fontsize{7}{6}\selectfont,line
cap=round,line join=round,>=triangle 45,x=1.0cm,y=1.0cm,line width=0.8pt]
\clip(-0.72,-6.26) rectangle (11.86,3.38);
\draw (-0.48,-4.98)-- (11.02,-2.66);
\draw (0.2,-0.16)-- (10.54,1.94);
\draw (7.24,2.38)-- (2.26,-5.64);
\draw [shift={(6.45,1.11)}] plot[domain=3.34:4.16,variable=\t]({1*1.2*cos(\t
r)+0*1.2*sin(\t r)},{0*1.2*cos(\t r)+1*1.2*sin(\t r)});
\draw [shift={(3.12,-4.25)}] plot[domain=0.2:1.02,variable=\t]({1*0.76*cos(\t
r)+0*0.76*sin(\t r)},{0*0.76*cos(\t r)+1*0.76*sin(\t r)});
\draw (4.1,-3.34) node[anchor=north west] {1};
\draw (5.68,0.76) node[anchor=north west] {2};
\draw (10.5,1.9) node[anchor=north west] {$l_1$};
\draw (11.28,-2.2) node[anchor=north west] {$l_2$};
\draw (2.82,-4.92) node[anchor=north west] {$l_3$};
\draw [dash pattern=on 3pt off 3pt] (2.14,2.72)-- (11.26,-0.68);
\draw (11.44,-0.36) node[anchor=north west] {$l_4$};
\draw (7.78,1.62)-- (7.1,1.3);
\draw (3.46,-4.04)-- (3.9,-3.68);
\draw (7.66,1.96) node[anchor=north west] {3};
\draw [shift={(6.45,1.11)}] plot[domain=-0.35:1.02,variable=\t]({1*1.13*cos(\t
r)+0*1.13*sin(\t r)},{0*1.13*cos(\t r)+1*1.13*sin(\t r)});
\fill [color=uququq] (6.45,1.11) circle (1.5pt);
\draw[color=uququq] (6.28,1.5) node {$A$};
\fill [color=uququq] (3.12,-4.25) circle (1.5pt);
\draw[color=uququq] (2.88,-3.94) node {$B$};
\end{tikzpicture}}{ACEP}{acep2}
\end{figura}
%\setlength{\extrarowheight}{1.5pt}
\begin{prueba}{\rr $\angle 1 \ncong \angle 2$}{
2. & $ l_1 \parallel l_2$ & Dato.\\ 
3. & \rr \multirow{2}{8cm}[2ex]{ Trazamos una recta $l_4$ de tal
 forma que pasa $A$ y con la secante
 forma un $\angle 3$ congruente con el $\angle1$}
& Construcci\'on auxiliar \\
4. & $l_4 \parallel l_2$ y $A$ est\'a en $l_4$ & \'Angulos
correspondientes congruentes.\\
5. & Hay dos rectas paralelas a la recta $l_2$ & de 2) y 4).\\
6. & $\angle 1 \cong \angle 2$ & Conclusi\'on de la prueba indirecta\\
}
\end{prueba}


\begin{conjetura}{Dos rectas son  paralelas si y son por una recta transversal
sus \'angulos alternos internos son congruentes}{\'Angulos
alternos internos (AAI)}
 \end{conjetura}


\begin{conjetura}{Dos rectas son  paralelas si y s\'olo si  al ser
interceptadas por una recta transversal sus \'angulos alternos externos
son congruentes}{\'Angulos
alternos externos (AAE)}
 \end{conjetura}

\begin{conjetura}{Dos rectas son  paralelas si y s\'olo si  al ser
interceptadas por una recta transversal sus \'angulos conjugados
internos son suplementarios}{\'Angulos
conjugados internos (ACI)}
 \end{conjetura}
\begin{teorema}{Si dos rectas son paralelas y una secante es perpendicular a una
de las paralelas, entonces tambi\'en es perpendicular a la otra.}{recta
perpendicular}
\end{teorema}
\begin{construccion}{Sea $\overleftrightarrow{AB}$\ una recta paralela
que pasa por por un punto $C$.}{Recta paralela}
Para resolver el problema vamos a seguir los siguientes pasos:
\begin{lista}
 \item Se construye una recta $\overleftrightarrow{AB}$ \ que pase por el punto
$C$.
\item Se construye un segmento $AC$.
\item Se traza un arco centrado en $A$ que pase por $C$, radio $f$.
\item Se traza un arco centrado en $C$ con el mismo radio $f$.
\item En el punto de intersecci\'on del arco centrado en $A$ y la recta
$\overleftrightarrow{AB}$  \ se traza que pase por $C$, radio $e$.
\item En el punto de intersecci\'on de la circunferencia centrada en $C$ y
radio $f$ se traza una circunferencia de radio $e$.
\item Se traza una recta entre el punto $C$ y el punto de intersecci\'on de la
circunferencia y el arco centrado en $C$.
\end{lista}

\end{construccion}
\begin{figura}{
\definecolor{qqwuqq}{rgb}{0,0.39,0}
\definecolor{xdxdff}{rgb}{0.49,0.49,1}
\definecolor{ffttww}{rgb}{1,0.2,0.4}
\definecolor{ffzzqq}{rgb}{1,0.6,0}
\definecolor{ttttff}{rgb}{0.2,0.2,1}
\definecolor{uququq}{rgb}{0.25,0.25,0.25}
\definecolor{cccccc}{rgb}{0.8,0.8,0.8}
\definecolor{qqqqff}{rgb}{0,0,1}
\begin{tikzpicture}[scale=0.8,font=\fontsize{7}{6}\selectfont,line
cap=round,line join=round,>=triangle 45,x=1.0cm,y=1.0cm]
\clip(-3.55,-2.62) rectangle (6.25,6.56);
\draw [shift={(1,2)},color=qqwuqq,fill=qqwuqq,fill opacity=0.1] (0,0) --
(0.09:0.35) arc (0.09:63.43:0.35) -- cycle;
\draw [shift={(0,0)},color=qqwuqq,fill=qqwuqq,fill opacity=0.1] (0,0) --
(0:0.35) arc (0:63.43:0.35) -- cycle;
\draw [line width=1.6pt] (1,2)-- (0,0);
\draw (0,0)-- (4,0);
\draw [line width=1.6pt,dash pattern=on 1pt off
1pt,color=cccccc,fill=cccccc,fill opacity=0.25] (0,0) circle (2.24cm);
\draw [line width=1.6pt,dash pattern=on 3pt off 3pt,color=ttttff] (1,2)--
(2.24,0);
\draw [line width=1.6pt,dash pattern=on 3pt off 3pt,color=ffzzqq] (0,0)--
(2.24,0);
\draw [line width=1.6pt,dash pattern=on 3pt off
3pt,color=ffttww,fill=ffttww,fill opacity=0.05] (2.24,0) circle (2.35cm);
\draw [line width=1.2pt,dash pattern=on 1pt off
1pt,color=cccccc,fill=cccccc,fill opacity=0.15] (1,2) circle (2.24cm);
\draw [line width=1.6pt,dash pattern=on 3pt off 3pt,color=ffzzqq] (0,0)-- (2,4);
\draw [line width=1.6pt,dash pattern=on 1pt off
1pt,color=ffttww,fill=ffttww,fill opacity=0.05] (2,4) circle (2.35cm);
\draw [line width=1.6pt,dash pattern=on 3pt off 3pt,color=ttttff] (2,4)--
(3.24,2);
\draw [line width=1.6pt,dash pattern=on 3pt off 3pt,color=ffzzqq] (1,2)--
(-0.34,3.79);
\draw [line width=1.6pt,dash pattern=on 3pt off 3pt,color=ffzzqq] (1,2)-- (6,2);
\draw [line width=1.6pt,dash pattern=on 3pt off 3pt] (-3,0)-- (6,0);
\fill [color=qqqqff] (0,0) circle (1.5pt);
\draw[color=qqqqff] (0.04,-0.13) node {$A$};
\fill [color=qqqqff] (4,0) circle (1.5pt);
\draw[color=qqqqff] (4.1,0.15) node {$B$};
\fill [color=qqqqff] (1,2) circle (1.5pt);
\draw[color=qqqqff] (0.98,1.73) node {$C$};
\draw[color=cccccc] (-1.09,1.76) node {$c$};
\fill [color=uququq] (2.24,0) circle (1.5pt);
\draw[color=uququq] (2.33,-0.11) node {$D$};
\draw[color=ttttff] (1.49,0.99) node {$e$};
\draw[color=ffzzqq] (0.69,0.74) node {$f$};
\fill [color=xdxdff] (2,4) circle (1.5pt);
\draw[color=xdxdff] (1.95,4.23) node {$E$};
\fill [color=uququq] (-0.34,3.79) circle (1.5pt);
\fill [color=xdxdff] (3.24,2) circle (1.5pt);
\draw[color=xdxdff] (3.3,1.81) node {$H$};
\end{tikzpicture}}{Recta paralela}{parll}
\end{figura}

\begin{conjetura}{La suma de los \'angulos interiores de un tri\'angulo es
180}{SAIT}
 \end{conjetura}
Dado el $\triangle ABC$, ver la siguiente figura(\ref{sait}), se probar\'a que
$m\angle1+m\angle2+m\angle3=180.$%
\begin{figura}{
\definecolor{zzttqq}{rgb}{0.27,0.27,0.27}
\definecolor{qqqqff}{rgb}{0.33,0.33,0.33}
\begin{tikzpicture}[line cap=round,line join=round,>=triangle
45,x=1.0cm,y=1.0cm]
\clip(-2.1,-0.82) rectangle (5.34,6.04);
\draw (0.84,5.08)-- (-1.3,1.02);
\draw (-1.3,1.02)-- (4.74,-0.12);
\draw (4.74,-0.12)-- (0.84,5.08);
\draw [domain=-2.1:5.34] plot(\x,{(--31.64-1.14*\x)/6.04});
\draw [domain=-2.1:5.34] plot(\x,{(--4.68-1.14*\x)/6.04});
\draw [->] (0.84,5.08) -- (3.92,4.5);
\draw [->] (-1.3,1.02) -- (1.73,0.45);
\draw [shift={(0.84,5.08)},color=zzttqq,fill=zzttqq,fill opacity=0.1]  (0,0) --
plot[domain=5.36:6.1,variable=\t]({1*0.91*cos(\t r)+0*0.91*sin(\t
r)},{0*0.91*cos(\t r)+1*0.91*sin(\t r)}) -- cycle ;
\draw [shift={(0.84,5.08)},color=zzttqq,fill=zzttqq,fill opacity=0.1]  (0,0) --
plot[domain=4.23:5.36,variable=\t]({1*0.87*cos(\t r)+0*0.87*sin(\t
r)},{0*0.87*cos(\t r)+1*0.87*sin(\t r)}) -- cycle ;
\draw [shift={(0.84,5.08)},color=zzttqq,fill=zzttqq,fill opacity=0.1]  (0,0) --
plot[domain=2.96:4.23,variable=\t]({1*0.98*cos(\t r)+0*0.98*sin(\t
r)},{0*0.98*cos(\t r)+1*0.98*sin(\t r)}) -- cycle ;
\draw [shift={(-1.3,1.02)},color=zzttqq,fill=zzttqq,fill opacity=0.1]  (0,0) --
plot[domain=-0.19:1.09,variable=\t]({1*1.12*cos(\t r)+0*1.12*sin(\t
r)},{0*1.12*cos(\t r)+1*1.12*sin(\t r)}) -- cycle ;
\draw [shift={(4.74,-0.12)},color=zzttqq,fill=zzttqq,fill opacity=0.1]  (0,0) --
 plot[domain=2.21:2.96,variable=\t]({1*1.43*cos(\t r)+0*1.43*sin(\t
r)},{0*1.43*cos(\t r)+1*1.43*sin(\t r)}) -- cycle ;
\draw (-0.08,1.86) node[anchor=north west] {1};
\draw (3.16,1.06) node[anchor=north west] {2};
\draw (0.86,4.08) node[anchor=north west] {3};
\draw (1.84,4.6) node[anchor=north west] {4};
\draw (-0.4,4.94) node[anchor=north west] {5};
\draw (3.66,4.44) node[anchor=north west] {$l_1$};
\fill [color=qqqqff] (0.84,5.08) circle (1.5pt);
\draw[color=qqqqff] (0.94,5.44) node {$C$};
\fill [color=zzttqq] (-1.3,1.02) circle (1.5pt);
\draw[color=zzttqq] (-1.32,0.82) node {$A$};
\fill [color=zzttqq] (4.74,-0.12) circle (1.5pt);
\draw[color=zzttqq] (4.98,0.16) node {$B$};
\draw[color=black] (-4.12,5.84) node {$a$};
\end{tikzpicture}}{SAIT}{sait}
\end{figura}
\begin{prueba}{Sea un tri\'angulo $ABC$}{
2. & $\overleftrightarrow{l}\Vert\overleftrightarrow{AB}$ y pasa por $C$ &
Construcci\'on auxiliar\\
3. & $\angle1\cong\angle5$ y $\angle2\cong\angle4$ &
\multicolumn{1}{|l}{Teorema de paralelismo}\\
4. & $m\angle3+m\angle4+m\angle5=180$ & \multicolumn{1}{|l}{Postulado de
adici\'on de \'angulos y}\\
&  & \multicolumn{1}{|l}{postulado de par lineal}\\
5. & $m\angle1+m\angle2+m\angle3=180$ & \multicolumn{1}{|l}{Sustituci\'on de 2)
en 3)}%
}
\end{prueba}

 \subsection{Rectas paralelas en un plano}
 \begin{conjetura}{En un plano, si dos rectas son paralelas a una tercera recta,
entonces son paralelas entre s\'i.}{Transitividad entre Paralelas}
\end{conjetura}
 La prueba se deja como ejercicio.
\begin{teorema}{Dos rectas en un plano son paralelas, si ambas son
perpendiculares a una misma recta.}{Transitividad entre paralelas}
\end{teorema}
% La prueba se deja como ejercicio.\vspace{-0.1cm}
%
\begin{teorema}{Por un punto exterior a una recta dada pasa una y s\'olo una
perpendicular a la recta dada.}{Unicidad de la perpendicular}
\end{teorema}
%
% La prueba se deja como ejercicio.\vspace{-0.1cm}
%
\begin{teorema}{Ning\'un tri\'angulo tiene dos \'angulos rectos.}{}
\end{teorema}
%
% La prueba se deja como ejercicio.
%
\begin{ejemplo}{
\begin{minipage}[c]{7cm}
 Dado: $m\angle2+m\angle3+m\angle5=180,$\\
 $\angle4\cong\angle5.$\\
Pruebe: $\overline{AB}\parallel\overline{CD}.$%
\end{minipage}\hfill
\begin{minipage}[c]{7cm}
\begin{figura}{
\definecolor{zzttqq}{rgb}{0.27,0.27,0.27}
\definecolor{qqqqff}{rgb}{0.33,0.33,0.33}
\begin{tikzpicture}[scale=0.75,font=\fontsize{7}{6}\selectfont,line
cap=round,line join=round,>=triangle 45,x=1.0cm,y=1.0cm,line width=0.8pt]
\clip(-1.56,-1.18) rectangle (5.56,7.44);
\draw (0.84,5.08)-- (-1.3,1.02);
\draw (-1.3,1.02)-- (4.74,-0.12);
\draw (4.74,-0.12)-- (0.84,5.08);
\draw [domain=-1.56:5.56] plot(\x,{(--4.68-1.14*\x)/6.04});
\draw [->] (0.84,5.08) -- (3.92,4.5);
\draw [->] (-1.3,1.02) -- (1.73,0.45);
\draw [shift={(0.84,5.08)},color=zzttqq,fill=zzttqq,fill opacity=0.1]  (0,0) --
plot[domain=5.36:6.1,variable=\t]({1*0.91*cos(\t r)+0*0.91*sin(\t
r)},{0*0.91*cos(\t r)+1*0.91*sin(\t r)}) -- cycle ;
\draw [shift={(0.84,5.08)},color=zzttqq,fill=zzttqq,fill opacity=0.1]  (0,0) --
plot[domain=4.23:5.36,variable=\t]({1*0.87*cos(\t r)+0*0.87*sin(\t
r)},{0*0.87*cos(\t r)+1*0.87*sin(\t r)}) -- cycle ;
\draw [shift={(-1.3,1.02)},color=zzttqq,fill=zzttqq,fill opacity=0.1]  (0,0) --
plot[domain=-0.19:1.09,variable=\t]({1*1.12*cos(\t r)+0*1.12*sin(\t
r)},{0*1.12*cos(\t r)+1*1.12*sin(\t r)}) -- cycle ;
\draw [shift={(4.74,-0.12)},color=zzttqq,fill=zzttqq,fill opacity=0.1]  (0,0) --
 plot[domain=2.21:2.96,variable=\t]({1*1.43*cos(\t r)+0*1.43*sin(\t
r)},{0*1.43*cos(\t r)+1*1.43*sin(\t r)}) -- cycle ;
\draw (1.65,5.56) node {1};
\draw (1.84,4.66) node {2};
\draw (0.86,4.01) node {3};
\draw (0,1.64) node {4};
\draw (3.06,0.92) node {5};
%\draw (3.66,4.44) node[anchor=north west] {$l_1$};
\draw [->] (-1.3,1.02) -- (1.78,6.88);
\draw [shift={(0.84,5.08)},color=zzttqq,fill=zzttqq,fill opacity=0.1]  (0,0) --
plot[domain=-0.15:1.09,variable=\t]({1*0.79*cos(\t r)+0*0.79*sin(\t
r)},{0*0.79*cos(\t r)+1*0.79*sin(\t r)}) -- cycle ;
\fill [color=qqqqff] (0.84,5.08) circle (1.5pt);
\draw[color=qqqqff] (0.62,5.32) node {$C$};
\fill [color=zzttqq] (-1.3,1.02) circle (1.5pt);
\draw[color=zzttqq] (-1.32,0.82) node {$A$};
\fill [color=zzttqq] (4.74,-0.12) circle (1.5pt);
\draw[color=zzttqq] (4.98,0.16) node {$B$};
\fill [color=qqqqff] (2.92,4.69) circle (1.5pt);
\draw[color=qqqqff] (2.96,5.22) node {$D$};
\end{tikzpicture}}{Ejemplo}{ej1}
 \end{figura}

\end{minipage}

}
 
\end{ejemplo}
\begin{sol}{$m\angle2+m\angle3+m\angle5=180$}{
2. & $\angle4\cong\angle5$ & Dado\\
3. & $m\angle3+m\angle4+m\angle5=180$ & Teorema suma de \'angulos\\
&  & interiores de un tri\'angulo\\
4. & $m\angle2+m\angle3=m\angle ACD$ & Postulado de adici\'on de \'angulos\\
5. & $m\angle ACD+m\angle1=180$ & Postulado del par lineal\\
6. & $m\angle2+m\angle3+m\angle1=180$ & Sustituci\'on de 4) en 5)\\
7. & $m\angle2+m\angle3+m\angle5=$ & Igualaci\'on\\
& \multicolumn{1}{c|}{$m\angle2+m\angle3+m\angle1$} & \\
8. & $m\angle5=m\angle1$ & Propiedad cancelativa\\
9. & $m\angle4=m\angle1$ & Propiedad transitiva\\
10. & $\overline{AB}\parallel\overline{CD}$ & Teorema de paralelas
}
\end{sol}

% \begin{solsinpunto}%
% \[%
% \begin{tabular}
% [c]{cl|l}\hline
% \multicolumn{2}{l|}{Afirmaciones} & Razones\\\hline\hline
% \vspace{-0.3cm} &  & \\
% 1. & $m\angle2+m\angle3+m\angle5=180$ & Dado\\
% \vspace{-0.3cm} &  & \\
% 2. & $\angle4\cong\angle5$ & Dado\\
% \vspace{-0.3cm} &  & \\
% 3. & $m\angle3+m\angle4+m\angle5=180$ & Teorema suma de �ngulos\\
% &  & interiores de un tri�ngulo\\
% \vspace{-0.3cm} &  & \\
% 4. & $m\angle2+m\angle3=m\angle ACD$ & Postulado de adici�n de �ngulos\\
% \vspace{-0.3cm} &  & \\
% 5. & $m\angle ACD+m\angle1=180$ & Postulado del par lineal\\
% &  & \\
% 6. & $m\angle2+m\angle3+m\angle1=180$ & Sustituci�n de 4) en 5)\\
% &  & \\
% 7. & $m\angle2+m\angle3+m\angle5=$ & Igualaci�n\\
% & \multicolumn{1}{r|}{$m\angle2+m\angle3+m\angle1$} & \\
% \vspace{-0.3cm} &  & \\
% 8. & $m\angle5=m\angle1$ & Propiedad cancelativa\\
% \vspace{-0.3cm} &  & \\
% 9. & $m\angle4=m\angle1$ & Propiedad transitiva\\
% \vspace{-0.3cm} &  & \\
% 10. & $\overline{AB}\parallel\overline{CD}$ & Teorema de paralelas
% \end{tabular}
% \ \ \
% \]
% 
% \end{solsinpunto}
% 
% \begin{example}
% \textcolor{white}{.}\vspace{-0.5cm}
% \end{example}

% %
%
% \begin{tabular}
% [t]{ll}%
% Dado: & $\medskip\overleftrightarrow{AB}\parallel\overleftrightarrow{CD},$\\
% & $\medskip\overleftrightarrow{BC}\parallel\overleftrightarrow{DE}.$\\
% Pru�bese: & $m\angle1+m\angle4=180.$%
% \end{tabular}%
% %TCIMACRO{\FRAME{itbpF}{1.9683in}{1.7936in}{1.0032in}{}{}{grafico22.wmf}%
% %{\special{ language "Scientific Word";  type "GRAPHIC";
% %maintain-aspect-ratio TRUE;  display "USEDEF";  valid_file "F";
% %width 1.9683in;  height 1.7936in;  depth 1.0032in;  original-width 3.173in;
% %original-height 2.8893in;  cropleft "0";  croptop "1";  cropright "1";
% %cropbottom "0";
% %filename 'FIGURASS/CAP2/Grafico22.wmf';file-properties "XNPEU";}}}%
% %BeginExpansion
% \raisebox{-1.0032in}{\includegraphics[
% natheight=2.889300in,
% natwidth=3.173000in,
% height=1.7936in,
% width=1.9683in
% ]%
% {FIGURASS/CAP2/Grafico22.wmf}%
% }%
% %EndExpansion
%
%
% \begin{solsinpunto}%
% \[%
% \begin{tabular}
% [c]{cl|c}\hline
% \multicolumn{2}{c|}{Afirmaciones} & Razones\\\hline\hline
% \vspace{-0.3cm} &  & \\
% 1. & $\overleftrightarrow{AB}\parallel\overleftrightarrow{CD}$ &
% \multicolumn{1}{|l}{Dado}\\
% \vspace{-0.3cm} &  & \\
% 2. & $\overleftrightarrow{BC}\parallel\overleftrightarrow{DE}$ &
% \multicolumn{1}{|l}{Dado}\\
% \vspace{-0.3cm} &  & \\
% 3. & $m\angle1=m\angle3$ & \multicolumn{1}{|l}{Teorema �ngulos
% correspondientes}\\
% \vspace{-0.3cm} &  & \\
% 4. & $m\angle2=m\angle4$ & \multicolumn{1}{|l}{Teorema �ngulos
% correspondientes}\\
% \vspace{-0.3cm} &  & \\
% 5. & $m\angle2+m\angle3=180$ & \multicolumn{1}{|l}{Postulado del par lineal}\\
% \vspace{-0.3cm} &  & \\
% 6. & $m\angle4+m\angle1=180$ & \multicolumn{1}{|l}{Sustituci�n de 3) y 4) en
% 5)}%
% \end{tabular}
% \ \ \
% \]
%
% \end{solsinpunto}
%
% \begin{example}
% En la figura, $\overline{RS}\parallel\overline{PQ},$ $m\angle R=55$ y $m\angle
% P=40.$ Hallar la medida de $\angle1,$ $\angle2,$ $\angle3,$ $\angle4$ y
% $\angle5.$%
% \[%
% %TCIMACRO{\FRAME{itbpF}{2.0237in}{2.0124in}{0in}{}{}{grafico23.wmf}%
% %{\special{ language "Scientific Word";  type "GRAPHIC";
% %maintain-aspect-ratio TRUE;  display "USEDEF";  valid_file "F";
% %width 2.0237in;  height 2.0124in;  depth 0in;  original-width 3.0312in;
% %original-height 3.0139in;  cropleft "0";  croptop "1";  cropright "1";
% %cropbottom "0";
% %filename 'FIGURASS/CAP2/Grafico23.wmf';file-properties "XNPEU";}}}%
% %BeginExpansion
% {\includegraphics[
% natheight=3.013900in,
% natwidth=3.031200in,
% height=2.0124in,
% width=2.0237in
% ]%
% {FIGURASS/CAP2/Grafico23.wmf}%
% }%
% %EndExpansion
% \]
%
% \end{example}
%
% \begin{sol}
% Como $\overline{RS}\parallel\overline{PQ},$ entonces los �ngulos alternos
% internos $\angle1$ y $\angle TPQ\;$son congruentes. Luego $m\angle1=40.$
%
% La suma de las medidas de los �ngulos interiores de un tri�ngulo es 180,
% entonces
% \[
% m\angle2+40+55=180,
% \]
% por lo tanto $m\angle2=85.$ Ahora, como $\angle2$ y $\angle4$ son opuestos por
% el v�rtice, entonces son congruentes y $m\angle4=85.$
%
% Los �ngulos $\angle2$ y $\angle3$ forman un par lineal, entonces
% \begin{align*}
% 85+m\angle3  &  =180\\
% m\angle3  &  =180-85\\
% m\angle3  &  =95
% \end{align*}
% El $\angle5$ y el $\angle TRS$ son alternos internos entre paralelas. Entonces
% son con\-gru\-en\-tes y por consiguiente $m\angle5=55.$
% \end{sol}
%
% \begin{exercise}
% \textcolor{white}{.}\hfil
%
% \end{exercise}
%
% \begin{enumerate}
% \item En la figura, $\angle x\cong\angle y$ y $\angle a\cong\angle b.$
% Demostrar que
%$\overleftrightarrow{l_{1}}\parallel\overleftrightarrow{l_{3}}.$%
% %TCIMACRO{\FRAME{dtbpF}{2.4085in}{2.1854in}{0pt}{}{}{grafico24.wmf}%
% %{\special{ language "Scientific Word";  type "GRAPHIC";
% %maintain-aspect-ratio TRUE;  display "USEDEF";  valid_file "F";
% %width 2.4085in;  height 2.1854in;  depth 0pt;  original-width 3in;
% %original-height 2.7198in;  cropleft "0";  croptop "1";  cropright "1";
% %cropbottom "0";
% %filename 'FIGURASS/CAP2/Grafico24.wmf';file-properties "XNPEU";}}}%
% %BeginExpansion
% \begin{center}
% \includegraphics[
% natheight=2.719800in,
% natwidth=3.000000in,
% height=2.1854in,
% width=2.4085in
% ]%
% {FIGURASS/CAP2/Grafico24.wmf}%
% \end{center}
% %EndExpansion
%
%
% \item Demostrar: En un plano si los lados de un �ngulo son paralelos a los de
% otro �ngulo, entonces los �ngulos o bien son congruentes o son suplementarios.
%
% \item \textbf{Teorema del �ngulo exterior%
% \index{Teorema!del �ngulo exterior}%
% }: Demuestre que la medida de un �ngulo exterior de un tri�ngulo es igual a la
% suma de las medidas de los �ngulos interiores no contiguos al �ngulo exterior.
%
% \item Demuestre: Si dos rectas paralelas son cortadas por una secante,
% entonces las bisectrices de los �ngulos correspondientes son paralelas.
%
% \item Demuestre: Si dos rectas paralelas son cortadas por una secante,
% entonces las rectas que bisecan los �ngulos internos del mismo lado de la
% secante (conjugados internos) son perpendiculares.
%
% \item
% \begin{tabular}
% [t]{ll}
% & \\
% \medskip Dado: & $\overline{AB}\parallel\overline{XY},$ $\overline
% {BC}\parallel\overline{YZ},$\\
% \medskip & $\overline{AC}\parallel\overline{XZ}.$\\
% Pru�bese: & $m\angle1=m\angle Y+m\angle Z$%
% \end{tabular}%
% %TCIMACRO{\FRAME{itbpF}{1.8239in}{1.7288in}{1.0032in}{}{}{grafico25.wmf}%
% %{\special{ language "Scientific Word";  type "GRAPHIC";
% %maintain-aspect-ratio TRUE;  display "USEDEF";  valid_file "F";
% %width 1.8239in;  height 1.7288in;  depth 1.0032in;  original-width 1.7858in;
% %original-height 1.6916in;  cropleft "0";  croptop "1";  cropright "1";
% %cropbottom "0";
% %filename 'FIGURASS/CAP2/Grafico25.wmf';file-properties "XNPEU";}}}%
% %BeginExpansion
% \raisebox{-1.0032in}{\includegraphics[
% natheight=1.691600in,
% natwidth=1.785800in,
% height=1.7288in,
% width=1.8239in
% ]%
% {FIGURASS/CAP2/Grafico25.wmf}%
% }%
% %EndExpansion
%
%
% \item Con base en los datos de la figura, �es $\overline{PQ}\parallel
% \overline{AB}$? �$\overline{AC}\parallel\overline{QR}$? �$\overline
% {PS}\parallel\overline{BC}$? Justifique sus respuestas.%
% %TCIMACRO{\FRAME{dtbpF}{2.0755in}{1.8611in}{0pt}{}{}{grafico26.wmf}%
% %{\special{ language "Scientific Word";  type "GRAPHIC";
% %maintain-aspect-ratio TRUE;  display "USEDEF";  valid_file "F";
% %width 2.0755in;  height 1.8611in;  depth 0pt;  original-width 2.6576in;
% %original-height 2.38in;  cropleft "0";  croptop "1";  cropright "1";
% %cropbottom "0";
% %filename 'FIGURASS/CAP2/Grafico26.wmf';file-properties "XNPEU";}}}%
% %BeginExpansion
% \begin{center}
% \includegraphics[
% natheight=2.380000in,
% natwidth=2.657600in,
% height=1.8611in,
% width=2.0755in
% ]%
% {FIGURASS/CAP2/Grafico26.wmf}%
% \end{center}
% %EndExpansion
%
%
% \item En la figura, $\overline{AB}\parallel\overline{DE},$ $m\angle1=120$ y
% $\angle A\cong\angle B.$ Hallar la medida del $\angle D.$ �Es el $\triangle
% CDE$ equi�ngulo?%
% %TCIMACRO{\FRAME{dhF}{1.772in}{1.6751in}{0pt}{}{}{grafico27.wmf}%
% %{\special{ language "Scientific Word";  type "GRAPHIC";
% %maintain-aspect-ratio TRUE;  display "USEDEF";  valid_file "F";
% %width 1.772in;  height 1.6751in;  depth 0pt;  original-width 2.8219in;
% %original-height 2.6697in;  cropleft "0";  croptop "1";  cropright "1";
% %cropbottom "0";
% %filename 'FIGURASS/CAP2/Grafico27.wmf';file-properties "XNPEU";}}}%
% %BeginExpansion
% \begin{center}
% \includegraphics[
% natheight=2.669700in,
% natwidth=2.821900in,
% height=1.6751in,
% width=1.772in
% ]%
% {FIGURASS/CAP2/Grafico27.wmf}%
% \end{center}
%EndExpansion

%\end{enumerate}


\begin{definicion}{Un \'angulo es un \'angulo  exterior de un
pol\'{\i}gono si y solamente si forma un par lineal con uno de los \'angulos
del pol\'igono. }{\'Angulo exterior Un \'angulo}
 \end{definicion}
\begin{figura}{
\definecolor{zzttqq}{rgb}{0.27,0.27,0.27}
\definecolor{xdxdff}{rgb}{0.66,0.66,0.66}
\definecolor{qqqqff}{rgb}{0.33,0.33,0.33}
\begin{tikzpicture}[scale=0.7,line cap=round,line join=round,>=triangle
45,x=1.0cm,y=1.0cm]
\clip(-3.73,-7.07) rectangle (11.29,5.73);
\draw (3.57,-1.34)-- (0.15,5.3);
\draw (1.53,2.59)-- (-1.05,-3.17);
\draw (-0.17,-1.23)-- (9.81,-4.18);
\draw (6.46,-3.19)-- (1.29,0.07);
\draw [shift={(6.46,-3.19)},color=zzttqq,fill=zzttqq,fill opacity=0.1]  (0,0) --
 plot[domain=-0.28:2.58,variable=\t]({1*1.16*cos(\t r)+0*1.16*sin(\t
r)},{0*1.16*cos(\t r)+1*1.16*sin(\t r)}) -- cycle ;
\draw [shift={(1.53,2.59)},color=zzttqq,fill=zzttqq,fill opacity=0.1]  (0,0) --
plot[domain=2.03:4.29,variable=\t]({1*1.08*cos(\t r)+0*1.08*sin(\t
r)},{0*1.08*cos(\t r)+1*1.08*sin(\t r)}) -- cycle ;
\draw [shift={(-0.17,-1.23)},color=zzttqq,fill=zzttqq,fill opacity=0.1]  (0,0)
--  plot[domain=4.28:6,variable=\t]({1*1.03*cos(\t r)+0*1.03*sin(\t
r)},{0*1.03*cos(\t r)+1*1.03*sin(\t r)}) -- cycle ;
\draw [shift={(3.57,-1.34)},color=zzttqq,fill=zzttqq,fill opacity=0.1]  (0,0) --
 plot[domain=2.05:2.6,variable=\t]({1*1.72*cos(\t r)+0*1.72*sin(\t
r)},{0*1.72*cos(\t r)+1*1.72*sin(\t r)}) -- cycle ;
\fill [color=qqqqff] (1.53,2.59) circle (1.5pt);
\draw[color=qqqqff] (1.74,2.94) node {$A$};
\fill [color=qqqqff] (-0.17,-1.23) circle (1.5pt);
\draw[color=qqqqff] (-1.23,-1.18) node {$B$};
\fill [color=qqqqff] (6.46,-3.19) circle (1.5pt);
\draw[color=qqqqff] (6.01,-3.75) node {$C$};
\fill [color=qqqqff] (3.57,-1.34) circle (1.5pt);
\draw[color=qqqqff] (3.78,-0.99) node {$D$};
\fill [color=xdxdff] (7.58,-3.52) circle (1.5pt);
\draw[color=xdxdff] (7.76,-3.17) node {$L$};
\fill [color=xdxdff] (1.04,3.56) circle (1.5pt);
\draw[color=xdxdff] (1.26,3.89) node {$N$};
\fill [color=xdxdff] (-0.6,-2.17) circle (1.5pt);
\draw[color=xdxdff] (-0.54,-2.69) node {$P$};
\fill [color=xdxdff] (2.78,0.19) circle (1.5pt);
\draw[color=xdxdff] (2.99,0.55) node {$R$};
\end{tikzpicture}}{\'Angulos exteriores}{angex}
\end{figura}

\section{Congruencia de triágulos}
\subsection{Correspondencia entre triángulos}
\begin{figura}{
\definecolor{zzttqq}{rgb}{0.27,0.27,0.27}
\definecolor{qqqqff}{rgb}{0.33,0.33,0.33}
\begin{tikzpicture}[scale=0.4,font=\fontsize{7}{6}\selectfont,line
cap=round,line join=round,>=triangle 45,x=1.0cm,y=1.0cm]
\clip(-4.13,-7.89) rectangle (16.39,4.53);
\fill[color=zzttqq,fill=zzttqq,fill opacity=0.1] (-3.07,-3.91) -- (-0.41,0.79)
-- (3.41,-4.18) -- cycle;
\fill[color=zzttqq,fill=zzttqq,fill opacity=0.1] (6.97,-3.94) -- (10.58,0.76) --
(13.9,-4.57) -- cycle;
\draw [color=zzttqq] (-3.07,-3.91)-- (-0.41,0.79);
\draw [color=zzttqq] (-0.41,0.79)-- (3.41,-4.18);
\draw [color=zzttqq] (3.41,-4.18)-- (-3.07,-3.91);
\draw [color=zzttqq] (6.97,-3.94)-- (10.58,0.76);
\draw [color=zzttqq] (10.58,0.76)-- (13.9,-4.57);
\draw [color=zzttqq] (13.9,-4.57)-- (6.97,-3.94);
\draw[color=qqqqff,->] (4.73,2.82) -- (4.82,2.82) -- (4.91,2.81)
-- (5,2.8) -- (5.09,2.79) -- (5.18,2.77) -- (5.27,2.76) -- (5.37,2.75) --
(5.46,2.73) -- (5.55,2.72) -- (5.65,2.7) -- (5.74,2.68) -- (5.84,2.66) --
(5.93,2.64) -- (6.03,2.62) -- (6.12,2.6) -- (6.22,2.57) -- (6.32,2.55) --
(6.42,2.53) -- (6.51,2.5) -- (6.61,2.47) -- (6.71,2.44) -- (6.81,2.41) --
(6.91,2.38) -- (7.01,2.35) -- (7.11,2.32) -- (7.21,2.29) -- (7.32,2.25) --
(7.42,2.22) -- (7.52,2.18) -- (7.63,2.14) -- (7.73,2.1) -- (7.83,2.07) --
(7.94,2.02) -- (8.04,1.98) -- (8.15,1.94) -- (8.26,1.9) -- (8.36,1.85) --
(8.47,1.81) -- (8.58,1.76) -- (8.69,1.71) -- (8.79,1.66) -- (8.9,1.61) --
(9.01,1.56) -- (9.12,1.51) -- (9.23,1.46) -- (9.35,1.41) -- (9.46,1.35) --
(9.57,1.3) -- (9.68,1.24) -- (9.79,1.18) -- (9.91,1.12) -- (10.02,1.06) --
(10.14,1) -- (10.25,0.94) -- (10.37,0.88) -- (10.48,0.81) -- (10.54,0.78) --
(10.57,0.77) -- (10.58,0.76) -- (10.58,0.76) -- (10.58,0.76) -- (10.58,0.76) --
(10.58,0.76) -- (10.58,0.76) -- (10.58,0.76) -- (10.58,0.76) --
(10.58,0.76)(-0.41,0.79) -- (-0.41,0.79) -- (-0.41,0.79) -- (-0.41,0.79) --
(-0.41,0.79) -- (-0.41,0.79) -- (-0.41,0.79) -- (-0.41,0.79) -- (-0.41,0.79) --
(-0.41,0.79) -- (-0.41,0.79) -- (-0.41,0.79) -- (-0.41,0.79) -- (-0.41,0.79) --
(-0.41,0.79) -- (-0.41,0.79) -- (-0.41,0.79) -- (-0.4,0.8) -- (-0.39,0.81) --
(-0.37,0.83) -- (-0.34,0.87) -- (-0.28,0.94) -- (-0.22,1) -- (-0.17,1.06) --
(-0.11,1.12) -- (-0.05,1.18) -- (0.01,1.24) -- (0.07,1.29) -- (0.12,1.35) --
(0.18,1.4) -- (0.24,1.46) -- (0.31,1.51) -- (0.37,1.56) -- (0.43,1.61) --
(0.49,1.66) -- (0.55,1.71) -- (0.62,1.76) -- (0.68,1.8) -- (0.75,1.85) --
(0.81,1.89) -- (0.87,1.94) -- (0.94,1.98) -- (1.01,2.02) -- (1.07,2.06) --
(1.14,2.1) -- (1.21,2.14) -- (1.28,2.18) -- (1.34,2.21) -- (1.41,2.25) --
(1.48,2.28) -- (1.55,2.32) -- (1.62,2.35) -- (1.69,2.38) -- (1.77,2.41) --
(1.84,2.44) -- (1.91,2.47) -- (1.98,2.5) -- (2.06,2.52) -- (2.13,2.55) --
(2.2,2.57) -- (2.28,2.6) -- (2.35,2.62) -- (2.43,2.64) -- (2.51,2.66) --
(2.58,2.68) -- (2.66,2.7) -- (2.74,2.71) -- (2.82,2.73) -- (2.89,2.75) --
(2.97,2.76) -- (3.05,2.77) -- (3.13,2.79) -- (3.21,2.8) -- (3.29,2.81) --
(3.38,2.82) -- (3.46,2.82) -- (3.54,2.83) -- (3.62,2.84) -- (3.71,2.84) --
(3.79,2.85) -- (3.87,2.85) -- (3.96,2.85) -- (4.04,2.85) -- (4.13,2.85) --
(4.22,2.85) -- (4.3,2.85) -- (4.39,2.85) -- (4.48,2.84) -- (4.57,2.84) --
(4.66,2.83) -- (4.73,2.82);
\draw[color=qqqqff,->] (5.4,-3.01) -- (5.43,-3.02) --
(5.46,-3.02) -- (5.49,-3.02) -- (5.51,-3.03) -- (5.54,-3.03) -- (5.57,-3.03) --
(5.6,-3.04) -- (5.62,-3.04) -- (5.65,-3.05) -- (5.68,-3.06) -- (5.71,-3.06) --
(5.73,-3.07) -- (5.76,-3.08) -- (5.79,-3.09) -- (5.82,-3.1) -- (5.85,-3.11) --
(5.87,-3.12) -- (5.9,-3.13) -- (5.93,-3.14) -- (5.96,-3.15) -- (5.98,-3.16) --
(6.01,-3.18) -- (6.04,-3.19) -- (6.07,-3.2) -- (6.09,-3.22) -- (6.12,-3.23) --
(6.15,-3.25) -- (6.18,-3.26) -- (6.2,-3.28) -- (6.23,-3.3) -- (6.26,-3.31) --
(6.29,-3.33) -- (6.31,-3.35) -- (6.34,-3.37) -- (6.37,-3.39) -- (6.4,-3.41) --
(6.42,-3.43) -- (6.45,-3.45) -- (6.48,-3.47) -- (6.51,-3.49) -- (6.53,-3.52) --
(6.56,-3.54) -- (6.59,-3.56) -- (6.62,-3.59) -- (6.64,-3.61) -- (6.67,-3.63) --
(6.7,-3.66) -- (6.73,-3.69) -- (6.75,-3.71) -- (6.78,-3.74) -- (6.81,-3.77) --
(6.84,-3.8) -- (6.86,-3.82) -- (6.89,-3.85) -- (6.92,-3.88) -- (6.95,-3.91) --
(6.96,-3.93) -- (6.97,-3.94) -- (6.97,-3.94) -- (6.97,-3.94) -- (6.97,-3.94) --
(6.97,-3.94) -- (6.97,-3.94) -- (6.97,-3.94) -- (6.97,-3.94) -- (6.97,-3.94) --
(6.97,-3.94)(3.41,-4.18) -- (3.41,-4.18) -- (3.41,-4.18) -- (3.41,-4.18) --
(3.41,-4.18) -- (3.41,-4.18) -- (3.41,-4.18) -- (3.41,-4.18) -- (3.41,-4.18) --
(3.41,-4.18) -- (3.41,-4.18) -- (3.41,-4.18) -- (3.41,-4.18) -- (3.41,-4.18) --
(3.41,-4.18) -- (3.41,-4.17) -- (3.41,-4.17) -- (3.42,-4.17) -- (3.42,-4.16) --
(3.43,-4.15) -- (3.45,-4.13) -- (3.48,-4.1) -- (3.51,-4.06) -- (3.53,-4.03) --
(3.56,-4) -- (3.59,-3.97) -- (3.62,-3.94) -- (3.65,-3.91) -- (3.67,-3.88) --
(3.7,-3.85) -- (3.73,-3.82) -- (3.76,-3.79) -- (3.79,-3.76) -- (3.81,-3.73) --
(3.84,-3.71) -- (3.87,-3.68) -- (3.9,-3.66) -- (3.93,-3.63) -- (3.96,-3.61) --
(3.98,-3.58) -- (4.01,-3.56) -- (4.04,-3.53) -- (4.07,-3.51) -- (4.1,-3.49) --
(4.12,-3.47) -- (4.15,-3.45) -- (4.18,-3.42) -- (4.21,-3.4) -- (4.24,-3.38) --
(4.26,-3.37) -- (4.29,-3.35) -- (4.32,-3.33) -- (4.35,-3.31) -- (4.38,-3.29) --
(4.4,-3.28) -- (4.43,-3.26) -- (4.46,-3.25) -- (4.49,-3.23) -- (4.52,-3.22) --
(4.54,-3.2) -- (4.57,-3.19) -- (4.6,-3.17) -- (4.63,-3.16) -- (4.66,-3.15) --
(4.68,-3.14) -- (4.71,-3.13) -- (4.74,-3.12) -- (4.77,-3.11) -- (4.79,-3.1) --
(4.82,-3.09) -- (4.85,-3.08) -- (4.88,-3.07) -- (4.91,-3.06) -- (4.93,-3.06) --
(4.96,-3.05) -- (4.99,-3.04) -- (5.02,-3.04) -- (5.05,-3.03) -- (5.07,-3.03) --
(5.1,-3.02) -- (5.13,-3.02) -- (5.16,-3.02) -- (5.18,-3.02) -- (5.21,-3.01) --
(5.24,-3.01) -- (5.27,-3.01) -- (5.3,-3.01) -- (5.32,-3.01) -- (5.35,-3.01) --
(5.38,-3.01) -- (5.4,-3.01);
\draw[<-,color=qqqqff,->] (7.32,-5.85) -- (7.45,-5.84) --
(7.57,-5.84) -- (7.7,-5.84) -- (7.83,-5.84) -- (7.96,-5.83) -- (8.08,-5.83) --
(8.21,-5.82) -- (8.34,-5.82) -- (8.46,-5.81) -- (8.59,-5.8) -- (8.71,-5.79) --
(8.83,-5.79) -- (8.96,-5.78) -- (9.08,-5.76) -- (9.2,-5.75) -- (9.32,-5.74) --
(9.44,-5.73) -- (9.57,-5.71) -- (9.69,-5.7) -- (9.81,-5.68) -- (9.93,-5.67) --
(10.04,-5.65) -- (10.16,-5.63) -- (10.28,-5.61) -- (10.4,-5.59) -- (10.52,-5.57)
-- (10.63,-5.55) -- (10.75,-5.53) -- (10.86,-5.51) -- (10.98,-5.49) --
(11.09,-5.46) -- (11.21,-5.44) -- (11.32,-5.41) -- (11.44,-5.39) --
(11.55,-5.36) -- (11.66,-5.33) -- (11.77,-5.3) -- (11.89,-5.27) -- (12,-5.24) --
(12.11,-5.21) -- (12.22,-5.18) -- (12.33,-5.15) -- (12.44,-5.11) --
(12.54,-5.08) -- (12.65,-5.04) -- (12.76,-5.01) -- (12.87,-4.97) --
(12.97,-4.94) -- (13.08,-4.9) -- (13.19,-4.86) -- (13.29,-4.82) -- (13.4,-4.78)
-- (13.5,-4.74) -- (13.61,-4.7) -- (13.71,-4.65) -- (13.81,-4.61) --
(13.86,-4.59) -- (13.89,-4.58) -- (13.9,-4.57) -- (13.9,-4.57) -- (13.9,-4.57)
-- (13.9,-4.57) -- (13.9,-4.57) -- (13.9,-4.57) -- (13.9,-4.57) -- (13.9,-4.57)
-- (13.9,-4.57)(-3.07,-3.91) -- (-3.07,-3.91) -- (-3.07,-3.91) -- (-3.07,-3.91)
-- (-3.07,-3.91) -- (-3.07,-3.91) -- (-3.07,-3.91) -- (-3.07,-3.91) --
(-3.07,-3.91) -- (-3.07,-3.91) -- (-3.07,-3.91) -- (-3.07,-3.91) --
(-3.07,-3.91) -- (-3.06,-3.91) -- (-3.06,-3.91) -- (-3.06,-3.91) --
(-3.05,-3.92) -- (-3.04,-3.92) -- (-3.01,-3.93) -- (-2.96,-3.95) --
(-2.85,-3.98) -- (-2.77,-4.01) -- (-2.69,-4.04) -- (-2.61,-4.06) --
(-2.53,-4.09) -- (-2.44,-4.12) -- (-2.36,-4.14) -- (-2.28,-4.17) -- (-2.2,-4.19)
-- (-2.12,-4.22) -- (-2.04,-4.24) -- (-1.96,-4.27) -- (-1.89,-4.29) --
(-1.81,-4.32) -- (-1.73,-4.34) -- (-1.65,-4.36) -- (-1.57,-4.39) --
(-1.49,-4.41) -- (-1.41,-4.44) -- (-1.33,-4.46) -- (-1.25,-4.48) -- (-1.17,-4.5)
-- (-1.1,-4.53) -- (-1.02,-4.55) -- (-0.94,-4.57) -- (-0.86,-4.59) --
(-0.78,-4.62) -- (-0.7,-4.64) -- (-0.63,-4.66) -- (-0.55,-4.68) -- (-0.47,-4.7)
-- (-0.39,-4.72) -- (-0.32,-4.74) -- (-0.24,-4.76) -- (-0.16,-4.78) --
(-0.09,-4.8) -- (-0.01,-4.82) -- (0.07,-4.84) -- (0.14,-4.86) -- (0.22,-4.88) --
(0.3,-4.9) -- (0.37,-4.92) -- (0.45,-4.94) -- (0.53,-4.96) -- (0.6,-4.98) --
(0.68,-5) -- (0.75,-5.01) -- (0.83,-5.03) -- (0.9,-5.05) -- (0.98,-5.07) --
(1.05,-5.08) -- (1.13,-5.1) -- (1.2,-5.12) -- (1.28,-5.13) -- (1.35,-5.15) --
(1.43,-5.17) -- (1.5,-5.18) -- (1.58,-5.2) -- (1.65,-5.22) -- (1.73,-5.23) --
(1.8,-5.25) -- (1.87,-5.26) -- (1.95,-5.28) -- (2.02,-5.29) -- (2.09,-5.31) --
(2.17,-5.32) -- (2.24,-5.33) -- (2.31,-5.35) -- (2.39,-5.36) -- (2.46,-5.38) --
(2.53,-5.39) -- (2.61,-5.4) -- (2.68,-5.42) -- (2.75,-5.43) -- (2.82,-5.44) --
(2.89,-5.45) -- (2.97,-5.47) -- (3.04,-5.48) -- (3.11,-5.49) -- (3.18,-5.5) --
(3.25,-5.51) -- (3.33,-5.52) -- (3.4,-5.53) -- (3.47,-5.55) -- (3.54,-5.56) --
(3.61,-5.57) -- (3.68,-5.58) -- (3.75,-5.59) -- (3.82,-5.6) -- (3.89,-5.61) --
(3.96,-5.62) -- (4.03,-5.63) -- (4.1,-5.63) -- (4.17,-5.64) -- (4.24,-5.65) --
(4.31,-5.66) -- (4.38,-5.67) -- (4.45,-5.68) -- (4.52,-5.69) -- (4.59,-5.69) --
(4.66,-5.7) -- (4.73,-5.71) -- (4.8,-5.72) -- (4.87,-5.72) -- (4.94,-5.73) --
(5.01,-5.74) -- (5.08,-5.74) -- (5.14,-5.75) -- (5.21,-5.75) -- (5.28,-5.76) --
(5.35,-5.77) -- (5.42,-5.77) -- (5.48,-5.78) -- (5.55,-5.78) -- (5.62,-5.79) --
(5.69,-5.79) -- (5.75,-5.8) -- (5.82,-5.8) -- (5.89,-5.8) -- (5.96,-5.81) --
(6.02,-5.81) -- (6.09,-5.82) -- (6.16,-5.82) -- (6.22,-5.82) -- (6.29,-5.82) --
(6.42,-5.83) -- (6.55,-5.83) -- (6.69,-5.84) -- (6.82,-5.84) -- (6.95,-5.84) --
(7.08,-5.85) -- (7.21,-5.85) -- (7.32,-5.85);
\fill [color=qqqqff] (-3.07,-3.91) circle (1.5pt);
\draw[color=qqqqff] (-3.36,-3.91) node {$A$};
\fill [color=qqqqff] (-0.41,0.79) circle (1.5pt);
\draw[color=qqqqff] (-0.44,1.21) node {$B$};
\fill [color=qqqqff] (3.41,-4.18) circle (1.5pt);
\draw[color=qqqqff] (3.81,-4.26) node {$C$};
\fill [color=qqqqff] (6.97,-3.94) circle (1.5pt);
\draw[color=qqqqff] (6.49,-3.96) node {$D$};
\fill [color=qqqqff] (10.58,0.76) circle (1.5pt);
\draw[color=qqqqff] (10.79,1.11) node {$E$};
\fill [color=qqqqff] (13.9,-4.57) circle (1.5pt);
\draw[color=qqqqff] (14.08,-4.23) node {$F$};
\end{tikzpicture}}{Correspondencia entre tri\'angulos}{corres}
\end{figura}
Dados dos tri\'angulos como se muestra en la figura(\ref{corres}) siempre
podemos establecer una correspondencia entre los v\'ertices, como por ejemplo:\\
Podemos establecer $A\longleftrightarrow F , C \longleftrightarrow D$ y
$B\longleftrightarrow E$.
\begin{definicion}{La correspondencia establecida entre los v\'ertices
de de dos tri\'angulos determina tres parejas de elementos del tri\'angulo
llamados partes correspondientes. }{Partes correspondientes}
 La correspondencia establecida en la figura \ref{corres} determina las
siguientes partes correspondientes.\\
$\angle A \longleftrightarrow \angle F, \angle B \longleftrightarrow \angle E,
\angle C \longleftrightarrow \angle D, \overline{AB}\longleftrightarrow
\overline{FE}, \overline{BC}\longleftrightarrow
\overline{ED}$ y $\overline{AC}\longleftrightarrow
\overline{DF}.$
\end{definicion}
\begin{definicion}{Dos tri\'angulos son congruentes si y s\'olo si existe una
correspondencia entre los tr\'angulos de tal forma que sus partes
correspodientes, sean congruentes.
}{Tri\'angulos congruentes}

\end{definicion}
\subsection{Postulados de congruencia}
\begin{postulado}{Si dos lados y el \'angulo comprendido de un tri\'angulo
son respectivamente congruentes con dos lados y un \'angulo comprendido de
otro tri\'angulo, entonces los dos tri\'angulos son congruentes.}{LAL}
\end{postulado}
\subsection{Postulados de congruencia}
\begin{postulado}{Si dos \'angulos y el lado del tri\'angulo que es com\'un,
son respectivamente congruentes con dos \'angulos y el lado comprendido de
otro tri\'angulo, entonces los tri\'angulos son congruentes.}{ALA}
\end{postulado}
\subsection{Postulados de congruencia}
\begin{postulado}{Si Los tres lados de un tri\'angulo son respectivamente
congruentes con tres lados de otro tri\'angulos, los tri\'angulos son
congruentes.}{LAL}
\end{postulado}
\begin{teorema}{
En un tri\'angulo is\'oseles, la mediana al lado de la base forma dos
tri\'angulos congruentes. punto medio 
}{Tri\'angulo is\'osseles}
\end{teorema}

\begin{teorema}{
Si en un tri\'angulo , dos \'angulos y un lado opuesto a uno de los \'angulos
son congruentes con dos \'angulos y el lado correspondiente de un segundo
tri\'angulo, entonces los tri\'angulos son congruentes.
}{LAA}
\end{teorema}

\begin{teorema}{
Si la hip\'otenusa y un \'angulo agudo de un tri\'angulo rect\'angulo son
congruentes con la hipotenusa y un \'angulo agudo de otro tri\'angulo
recta\'angulo, entonces los tri\'angulos son congruentes,
}{Hipotenusa y \'angulo}
\end{teorema}

\begin{teorema}{si la hipotenusa y un cateto de un tri\'angulo rect\'angulo son
congruentes con la hipotenusa y un cateto de otro tri\'angulo, entonces los
tri\'angulos son congruentes.
}{hipotenusa y el cateto}
\end{teorema}
\begin{teorema}{
Si un punto equidista de los extremos de segmento, entonces pertenece a la
mediatriz del segmento.
}{Mediatriz}
 \end{teorema}
\begin{teorema}{
El circuncentro equidista de los v\'ertices del tri\'angulo.
}{circuncentro}
\end{teorema}
\begin{teorema}{
El incentro de un tri\'angulo equidista de los lados del tri\'angulo
}{Incentro}
\end{teorema}
\begin{teorema}{
El centroide de un tri\'angulo se encuentra a dos tercios de la longitud de
cada mediana. 
}{Centroide}
 
\end{teorema}
\begin{teorema}{
Si las medidas de dos \'angulos de un tri\'angulo son diferentes, entonces la
longitud del lado opuesto al \'angulo mayor, es mayor que la longitud del
lado opuesto al \'angulo menor.
}{desigualdad triangular}
\end{teorema}
\begin{conjetura}{Un tri\'angulo es
is\'osceles si y s\'olo si los
\'angulos de la base son congruentes.}{Tri\'angulo is\'osceles}
 \end{conjetura}
\section{Proporciones}
\begin{definicion}{ Si $a,b,c,d \in I\! \! R $
 tal que se cumple $\dfrac{a}{b}=\dfrac{c}{d}, $ cuando $c,d\neq 0$, entonces
se dice que $a,b,c$ y $d$ son proporcionales.
}{Proporciones}
 \end{definicion}
    \subsection{Propiedades de las proporciones}
\begin{lista}
\item Si $\dfrac{a}{b}=\dfrac{c}{d}, $ entonces $ad=bc.$
\item Si $\dfrac{a}{b}=\dfrac{c}{d}, $ entonces $\dfrac{a+b}{b}=\dfrac{c+d}{d}.$
\item Si $\dfrac{a}{b}=\dfrac{c}{d}, $ entonces $\dfrac{a-b}{b}=\dfrac{c-d}{d}.
$
\item Si $\dfrac{a}{b}=\dfrac{c}{d}, $ entonces $\dfrac{a}{c}=\dfrac{b}{d}. $
\item Si $ad=bc, $ entonces $\dfrac{a}{b}=\dfrac{c}{d}. $
\section{Teorema fundamental de la proporcionalidad}
\end{lista}
\begin{definicion}{
Cuatro segmentos son proporcionales si y s\'olo si sus longitudes son
proporcionales
}{Segmentos proporcionales}
 
\end{definicion}

\begin{teorema}{ Si una recta paralela a un lado de un tri\'angulo interseca a
los otros dos lados, entonces divide a \'estos dos en segmentos proporcionales
}{Teorema fundamental}
\begin{figura}{
\definecolor{ccqqtt}{rgb}{0.8,0,0.2}
\definecolor{ttttff}{rgb}{0.2,0.2,1}
\definecolor{uququq}{rgb}{0.25,0.25,0.25}
\definecolor{xdxdff}{rgb}{0.49,0.49,1}
\definecolor{qqqqff}{rgb}{0,0,1}
\begin{tikzpicture}[line cap=round,line join=round,>=triangle
45,x=1.0cm,y=1.0cm]
\clip(-1.2,-4.82) rectangle (8.26,3.98);
\draw (0.24,-2.68)-- (4.3,3.22);
\draw (4.3,3.22)-- (7.3,-3.24);
\draw (7.3,-3.24)-- (0.24,-2.68);
\draw [domain=-1.2:8.26] plot(\x,{(-3.84--0.56*\x)/-7.06});
\draw [color=ttttff] (4.3,3.22)-- (2.33,0.36);
\draw [color=ccqqtt] (2.33,0.36)-- (0.24,-2.68);
\draw [color=ttttff] (4.3,3.22)-- (5.75,0.09);
\draw [color=ccqqtt] (5.75,0.09)-- (7.3,-3.24);
\draw (0.06,0.54) node[anchor=north west] {$l$};
\fill [color=qqqqff] (4.3,3.22) circle (1.5pt);
\draw[color=qqqqff] (4.4,3.64) node {$A$};
\fill [color=qqqqff] (0.24,-2.68) circle (1.5pt);
\draw[color=qqqqff] (-0.16,-2.98) node {$B$};
\fill [color=qqqqff] (7.3,-3.24) circle (1.5pt);
\draw[color=qqqqff] (7.52,-3.42) node {$C$};
\fill [color=xdxdff] (2.33,0.36) circle (1.5pt);
\draw[color=xdxdff] (1.84,0.14) node {$D$};
\fill [color=uququq] (5.75,0.09) circle (1.5pt);
\draw[color=uququq] (5.98,0.56) node {$E$};
\draw[color=ttttff] (3.12,2.12) node {$e$};
\draw[color=ccqqtt] (1.06,-0.82) node {$f$};
\draw[color=ttttff] (5.18,2.04) node {$g$};
\draw[color=ccqqtt] (6.78,-1.52) node {$h$};
\end{tikzpicture}
}{Proporci\'on}{propr}
 
\end{figura}
\end{teorema}
\begin{teorema}{Si una recta interseca a dos lados de un tri\'angulo y los
divide proporcionalmente, entonces la recta es paralela al tercer lado.
}{Rec\'iproco del teorema fundamental}
\end{teorema}
\section{Pol\'igonos semejantes}
\begin{definicion}{Dos pol\'igonos son semejantes si hay una correspondencia
entre los v\'ertices tal que los \'angulos correspondientes sean congruentes y
los lados correspondientes sean proporcionales
}{Pol\'igonos semejantes}
\end{definicion}
\subsection{Tri\'angulos semejantes}
\begin{postulado}{
Dos tri\'angulos son semejantes si existe una correspondencia entre ellos
tal que Dos parejas de a\'ngulos correspondientes son congruentes.}{A.A}
\end{postulado}
\begin{teorema}{Dos tri\'angulos son semejantes si y s\'olo si al establecer
una correspondencia entre ellos los \'angulos correspondientes son
congruentes.}{A.A.A}
 \end{teorema}
\begin{teorema}{Dos tri\'angulos son semejantes si y s\'olo si al establecer
una correspondencia entre los dos tri\'angulos sus lados correspondientes son
proporcionales }{L.L.L}
\end{teorema}
\begin{teorema}{Si un \'angulo de un tri\'angulo es congruente con un \'angulo
de otro tri\'angulo, y si los lados correspondientes que forman los \'angulos,
son proporcionales, entonces los tri\'angulos son semejantes.
}{L.A.L}
\end{teorema}
\subsection{Semejanza en tri\'angulos rect\'angulos}
\begin{definicion}{
Se dice que $x \in I\! \! R $ es media geom\'etrica de $a,b \in I\! \! R$ si y
s\'olo si se cumple $\dfrac{x}{a}=\dfrac{b}{x}, a,x \neq 0$
}{Media geom\'etrica}
\end{definicion}

\begin{teorema}{En un tri\'angulo rect\'angulo, la longitud de la altura a la
hipotenusa es la media geom\'etrica entre las longitudes de los dos segmentos
de la hip\'otenusa.}{Media geom\'etrica}
\end{teorema}
\begin{teorema}{Dados un tri\'angulo rect\'angulo y la altura a la
hipotenusa, cada cateto es la media geom\'etrica entre la longitud de la
hip\'otenusa y la longitud del segmento de la hipotenusa adyacente al
cateto}{Altura-hipotenusa}
\end{teorema}



\section{Paralelogramos}
\begin{definicion}{Un cuadril\'atero es un pol\'igono de cuatro
lados}{Cuadril\'atero}
\end{definicion}
\begin{definicion}{Un trapecio es un cuadril\'atero que tiene exactamente dos
lados paralelos}{Trapecio}
\begin{figura}{\definecolor{xdxdff}{rgb}{0.49,0.49,1}
\definecolor{qqqqff}{rgb}{0,0,1}
\begin{tikzpicture}[scale=0.8,font=\fontsize{9}{8}\selectfont,line
cap=round,line join=round,>=triangle 45,x=1.0cm,y=1.0cm,line width=1.2pt]
\clip(-1.07,-1.25) rectangle (5.06,2.06);
\draw (-0.91,-0.76)-- (-0.21,1.43);
\draw (-0.21,1.43)-- (3.48,1.39);
\draw (3.48,1.39)-- (4.77,-0.81);
\draw (4.77,-0.81)-- (-0.91,-0.76);
\draw (1.46,1.50) node[anchor=north west] {3.69};
\draw (1.7,-0.650) node[anchor=north west] {5.69};
\fill [color=qqqqff] (-0.21,1.43) circle (1.5pt);
\draw[color=qqqqff] (-0.40,1.55) node {$A$};
\fill [color=qqqqff] (3.48,1.39) circle (1.5pt);
\draw[color=qqqqff] (3.55,1.7) node {$B$};
\fill [color=xdxdff] (-0.91,-0.76) circle (1.5pt);
\draw[color=xdxdff] (-0.9,-1.1) node {$D$};
\fill [color=xdxdff] (4.77,-0.81) circle (1.5pt);
\draw[color=xdxdff] (4.85,-1.1) node {$E$};
\end{tikzpicture}}{Trapecio}{tra2}
\nota: Los lados paralelos de trapecio se llaman bases.
\end{figura}
\end{definicion}
\begin{construccion}{Para construir un trapecio, se construye dos segmentos
paralelos,  como se explic\'o en las construcci\'on de
paraleleas y luego se construye el cuadril\'atero. }{Trapecio}
\begin{figura}{
\definecolor{qqwuqq}{rgb}{0,0.39,0}
\definecolor{xdxdff}{rgb}{0.49,0.49,1}
\definecolor{ffttww}{rgb}{1,0.2,0.4}
\definecolor{ffzzqq}{rgb}{1,0.6,0}
\definecolor{ttttff}{rgb}{0.2,0.2,1}
\definecolor{uququq}{rgb}{0.25,0.25,0.25}
\definecolor{cccccc}{rgb}{0.8,0.8,0.8}
\definecolor{qqqqff}{rgb}{0,0,1}
\begin{tikzpicture}[scale=0.8,font=\fontsize{7}{6}\selectfont,line
cap=round,line
join=round,>=triangle 45,x=1.0cm,y=1.0cm,line width=1.2pt]
\clip(-3.55,-2.62) rectangle (6.25,6.56);
\draw [shift={(1,2)},color=qqwuqq,fill=qqwuqq,fill opacity=0.1] (0,0) --
(0.09:0.35) arc (0.09:63.43:0.35) -- cycle;
\draw [shift={(0,0)},color=qqwuqq,fill=qqwuqq,fill opacity=0.1] (0,0) --
(0:0.35) arc (0:63.43:0.35) -- cycle;
\draw [line width=1.6pt] (1,2)-- (0,0);
\draw (0,0)-- (4,0);
\draw [line width=1.6pt,dash pattern=on 1pt off
1pt,color=cccccc,fill=cccccc,fill opacity=0.25] (0,0) circle (2.24cm);
\draw [line width=1.6pt,dash pattern=on 3pt off 3pt,color=ttttff] (1,2)--
(2.24,0);
\draw [line width=1.6pt,dash pattern=on 3pt off 3pt,color=ffzzqq] (0,0)--
(2.24,0);
\draw [line width=1.6pt,dash pattern=on 3pt off
3pt,color=ffttww,fill=ffttww,fill opacity=0.05] (2.24,0) circle (2.35cm);
\draw [line width=1.2pt,dash pattern=on 1pt off
1pt,color=cccccc,fill=cccccc,fill opacity=0.15] (1,2) circle (2.24cm);
\draw [line width=1.6pt,dash pattern=on 3pt off 3pt,color=ffzzqq] (0,0)-- (2,4);
\draw [line width=1.6pt,dash pattern=on 1pt off
1pt,color=ffttww,fill=ffttww,fill opacity=0.05] (2,4) circle (2.35cm);
\draw [line width=1.6pt,dash pattern=on 3pt off 3pt,color=ttttff] (2,4)--
(3.24,2);
\draw [line width=1.6pt,dash pattern=on 3pt off 3pt,color=ffzzqq] (1,2)--
(-0.34,3.79);
\draw [line width=1.6pt,dash pattern=on 3pt off 3pt,color=ffzzqq] (1,2)-- (6,2);
\draw [line width=1.6pt,dash pattern=on 3pt off 3pt] (-3,0)-- (6,0);
\fill [color=qqqqff] (0,0) circle (1.5pt);
\draw[color=qqqqff] (0.04,-0.13) node {$A$};
\fill [color=qqqqff] (4,0) circle (1.5pt);
\draw[color=qqqqff] (4.1,0.15) node {$B$};
\fill [color=qqqqff] (1,2) circle (1.5pt);
\draw[color=qqqqff] (0.98,1.73) node {$C$};
\draw[color=cccccc] (-1.09,1.76) node {$c$};
\fill [color=uququq] (2.24,0) circle (1.5pt);
\draw[color=uququq] (2.33,-0.11) node {$D$};
\draw[color=ttttff] (1.49,0.99) node {$e$};
\draw[color=ffzzqq] (0.69,0.74) node {$f$};
\fill [color=xdxdff] (2,4) circle (1.5pt);
\draw[color=xdxdff] (1.95,4.23) node {$E$};
\fill [color=uququq] (-0.34,3.79) circle (1.5pt);
\fill [color=xdxdff] (3.24,2) circle (1.5pt);
\draw[color=xdxdff] (3.3,1.81) node {$H$};
\draw (3.3,1.81)-- (4.1,0.15);
\end{tikzpicture}}{Trapecio}{Tparll}
\end{figura}
En la figura(\ref{Tparll}) se observa el trapecio $ABHC$.
\end{construccion}

\begin{definicion}{Un paralelogramos es un cuadril\'atero que tiene sus lados
opuestos paralelos}{Paralelogramo}
\begin{figura}{\definecolor{uququq}{rgb}{0.25,0.25,0.25}
\definecolor{qqqqff}{rgb}{0,0,1}
\begin{tikzpicture}[line cap=round,line join=round,>=triangle
45,x=1.0cm,y=1.0cm]
\clip(1.4,-3.26) rectangle (10.18,1.4);
\draw (2.14,-2.04)-- (7.36,-2.04);
\draw (3.78,0.12)-- (2.14,-2.04);
\draw (3.78,0.12)-- (9,0.12);
\draw (9,0.12)-- (7.36,-2.04);
\fill [color=qqqqff] (2.14,-2.04) circle (1.5pt);
\draw[color=qqqqff] (1.9,-2.08) node {$A$};
\fill [color=qqqqff] (7.36,-2.04) circle (1.5pt);
\draw[color=qqqqff] (7.28,-2.32) node {$B$};
\fill [color=qqqqff] (3.78,0.12) circle (1.5pt);
\draw[color=qqqqff] (3.94,0.38) node {$C$};
\fill [color=uququq] (9,0.12) circle (1.5pt);
\draw[color=uququq] (9.16,0.38) node {$D$};
\end{tikzpicture}}{Paralelogramo}{parreg}
\end{figura}

\end{definicion}
\begin{definicion}{Un rombo es un paralelogramo que tiene sus lados
congruentes}{Rombo}
\end{definicion}
\begin{definicion}{Un rect\'angulo es un paralelogramo que tiene dos lados
contiguos perpendiculares.}{Rect\'angulo}
\end{definicion}
\begin{definicion}{El cuadrado es un paralelogramo que es rombo y
rect\'angulo a la vez}{Cuadrado}
\begin{figura}{
\definecolor{qqwuqq}{rgb}{0,0.39,0}
\definecolor{uququq}{rgb}{0.25,0.25,0.25}
\definecolor{zzttqq}{rgb}{0.6,0.2,0}
\definecolor{qqqqff}{rgb}{0,0,1}
\begin{tikzpicture}[scale=0.8,font=\fontsize{7}{6}\selectfont,line
cap=round,line join=round,>=triangle 45,x=1.0cm,y=1.0cm]
\clip(2.22,-4.1) rectangle (7.44,0.74);
\fill[color=zzttqq,fill=zzttqq,fill opacity=0.1] (4.04,0) -- (3.04,-2.2) --
(5.24,-3.2) -- (6.24,-1) -- cycle;
\draw[color=qqwuqq,fill=qqwuqq,fill opacity=0.1] (5.42,-2.81) -- (5.03,-2.64) --
(4.85,-3.02) -- (5.24,-3.2) -- cycle;
\draw [color=zzttqq] (4.04,0)-- (3.04,-2.2);
\draw [color=zzttqq] (3.04,-2.2)-- (5.24,-3.2);
\draw [color=zzttqq] (5.24,-3.2)-- (6.24,-1);
\draw [color=zzttqq] (6.24,-1)-- (4.04,0);
\fill [color=qqqqff] (4.04,0) circle (1.5pt);
\draw[color=qqqqff] (4.2,0.26) node {$A$};
\fill [color=qqqqff] (3.04,-2.2) circle (1.5pt);
\draw[color=qqqqff] (2.78,-2.1) node {$B$};
\fill [color=uququq] (5.24,-3.2) circle (1.5pt);
\draw[color=uququq] (5.24,-3.48) node {$C$};
\fill [color=uququq] (6.24,-1) circle (1.5pt);
\draw[color=uququq] (6.4,-0.74) node {$D$};
\draw[color=qqwuqq] (5.22,-2.3) node {$\beta = 90\textrm{\degre}$};
\end{tikzpicture}}{Cuadrado}{cua}
\end{figura}
 \end{definicion}
\begin{teorema}{
Los \'angulos opuestos de un paralelogramo son congruentes.
}{LOPC}

\end{teorema}
\begin{teorema}{
Los \'angulos opuestos de un paralelogramos son congruentes
}{AOPC}

\end{teorema}

\begin{teorema}{
Los pares de \'angulos adyacentes de un paralelogramo son \'angulos
suplementarios.
}{PAAPC}

\end{teorema}
\begin{teorema}{
Si los lados opuestos de un cuadri\'atero son congruentes, entonces el
cuadril\'atero es un paralelogramo
}{RLOPC}

\end{teorema}

\begin{teorema}{
Si los \'angulos opuestos de un cuadril\'atero son congruentes, entonces el
cuadril\'atero es un paralelogramo
}{RPAAPC}
\end{teorema}
\begin{teorema}{
El segmento que une los puntos medios de dos lados de un tri\'angulo es
paralelo al tercer lado y tiene la mitad de su longitud
}{Segmento medio}
\end{teorema}

\begin{teorema}{
Los puntos medios de los lados de un cuadril\'atero son v\'ertices de un
paralelogramp
}{PMCDP}

\end{teorema}
\begin{teorema}{
Un paralelogramo es un rect\'angulo si, y s\'olo si, sus diagonales son
congruentes,
}{Diagonales de un paralelogramo}

\end{teorema}
\begin{teorema}{
Un paralelogramo es un rombo si, y s\'olo si sus diagonales son perpendiculares
entre s\'i.
}{ERSDP}
\end{teorema}

\begin{teorema}{
Un paralelogramo es un rombo si, y s\'olo si, cada diagonal biseca a un par de
\'angulos opuestos
}{ERDBA}

\end{teorema}
\begin{teorema}{
El segmento que une los puntos medios de dos lados no paralelos de un trapecio
es paralelo a las dos bases y tiene una longitud igual a la semisuma de las
longitudes de las bases.
}{Segmento medio del trapecio}

\end{teorema}

\begin{teorema}{
En un trapecio con sus lados no paralelos congruentes, los \'angulos de la base
y las diagonales son congruentes
}{trapecio is\'osceles}

\end{teorema}

\begin{teorema}{
La suma de las medidas de los \'angulos exteriores de un pol\'igono, en cada 
uno de sus v\'ertices es, $360^\circ$.
}{}

\end{teorema}
 \begin{teorema}{
La suma de los \'angulos de un pol\'igono convexo de $n$ lados es
$(n-2)180^\circ$
}{}

\end{teorema}      

\problema{
\item Utilice las tablas de verdad para verificar que se cumplen las relaciones
 logicas propuestas en el  ejemplo \ref{ej2_11}
\item Demuestre el teorema \ref{pro2}
\item Demuestre el teorema \ref{pro3}}
\end{document}
