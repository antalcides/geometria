 %%%%%%%%%%%%%%%%%%%%%%%%%%%% Lista %%%%%%%%%%%%%%%%%%%%%%%%%%%%%%%%%%%%%
 %%%%%%%%%%%%%%%%%%%%%%%%%%%%%%%%%%%Lista    %%%%%%%%%%%%%%%%%%%%%%%%%%%%%
\newenvironment{lista}{
\begin{itemize}
 \renewcommand{\labelitemi}{{
 \colorbox{wood!90!black}{\color{white}{\ding{42}}}
 }}
}{\end{itemize}}
%%%%%%%%%%%%%%%%%%%%%%%%%%%%%%%%%%%%%%%%%%%%%%%%%%%%%%%%%%%%%%%%%%%
%%%%%%%%%%%%%%%%%%%%%%%%%%%%%%%%%%%%%%%%%%%% Ejemplo%%%%%%%%%%%%%%%%%%%%
\newcounter{ejemplo}
\newenvironment{ejemplo}[1]{\vskip 5pt
\stepcounter{ejemplo}
%\begin{tikzpicture}
%\tikzstyle{mybox} = [draw=red, fill=blue!20, very thick,
%    rectangle, rounded corners, inner sep=10pt, inner ysep=20pt]
%\tikzstyle{fancytitle} =[fill=red, text=black]
%\node [mybox] (box){%
%     \begin{minipage}[r]{\textwidth}
%                                     {#1}
%                                 \end{minipage}};
% \node[fancytitle, right=10pt] at (box.north west) {Ejemplo. \thechapter.\theejemplo};
% \node[fancytitle, rounded corners] at (box.east) {$\blacktriangle$};
%      \end{tikzpicture}
%                        \hspace*{10pt}  
\begin{tcolorbox}[colback=GreenTea!5,colframe=GreenTea!75!black,title=Ejemplo \ \thechapter.\ \thesection .\ \theejemplo]
  #1
\end{tcolorbox}
       }{

                \vskip 5pt
 }%\greenline \vskip 0.1pt}
%%%%%%%%%%%%%%%%%%%%%%%%%%%%%%%%%%%%%% postulado %%%%%%%%%%%%%%%%%%%%%%%%%%%%%%
\newcounter{postulado}
\newenvironment{postulado}[2]{\vskip 5pt
\stepcounter{postulado}
  %\begin{tikzpicture}
%\tikzstyle{mybox} = [draw=blue, fill=red!10, very thick,
%    rectangle, rounded corners, inner sep=10pt, inner ysep=20pt]
%\tikzstyle{fancytitle} =[fill=blue, text=white]
%\node [mybox] (box){ %
%     \begin{minipage}[r]{\textwidth}
%                                    {#1}
%                                 \end{minipage}};
% \node[fancytitle, right=10pt] at (box.north west) {Postulado.\thechapter.\thepostulado. \ \bf{#2}};
% \node[fancytitle, rounded corners] at (box.east) {$\clubsuit$};
%     \end{tikzpicture}
%              \hspace*{10pt} 
   \begin{tcolorbox}[colback=thegrey!5,colframe=thegrey!75!black,title=Postulado.\thechapter.\thepostulado . \ \bf{#2}]
 #1
\end{tcolorbox}            }{

                \vskip 5pt
 }%\greenline \vskip 0.1pt}
%%%%%%%%%%%%%%%%%%%%%%%%%%%%%%%%%%%%%%%%%%%%%%%%%%%%%%%%%%%%%%%%%%%%
%%%%%%%%%%%%%%%%%%%%%%%%%%%%%%%%%%%%%% Axioma %%%%%%%%%%%%%%%%%%%%%%%%%%%%%%
\newcounter{axioma}
\newenvironment{axioma}[2]{\vskip 5pt
\stepcounter{axioma}
%  \begin{tikzpicture}
%\tikzstyle{mybox} = [draw=blue, fill=red!10, very thick,
%    rectangle, rounded corners, inner sep=10pt, inner ysep=20pt]
%\tikzstyle{fancytitle} =[fill=blue, text=white]
%\node [mybox] (box){ %
%     \begin{minipage}[r]{\textwidth}
%                                    {#1}
%                                 \end{minipage}};
% \node[fancytitle, right=10pt] at (box.north west)
%{Postulado.\thechapter.\thepostulado. \ \bf{#2}};
% \node[fancytitle, rounded corners] at (box.east) {$\clubsuit$};
%     \end{tikzpicture}
%              \hspace*{10pt}  
\begin{tcolorbox}[colback=doc!5,colframe=doc!75!black,title=Axioma.\thechapter.\theaxioma. \ \bf{#2}]
 #1
\end{tcolorbox}              }{

                \vskip 5pt
 }%\greenline \vskip 0.1pt}
%%%%%%%%%%%%%%%%%%%%%%%%%%%%%%%%%%%%%%%%%%%%%%%%%%%%%%%%%%%%%%%%%%%%
%%%%%%%%%%%%%%%%%%%%%%%%%%%%%%%%%%%%%% Conjetura %%%%%%%%%%%%%%%%%%%%%%%%%%%%%%
\newcounter{conjetura}
\stepcounter{conjetura}
\newenvironment{conjetura}[1]{\vskip 5pt
%  \begin{tikzpicture}
%\tikzstyle{mybox} = [draw=blue, fill=green!20, very thick,
%    rectangle, rounded corners, inner sep=10pt, inner ysep=20pt]
%\tikzstyle{fancytitle} =[fill=blue, text=white, ellipse]
%\node [mybox] (box){ %
%     \begin{minipage}[r]{\textwidth}
%                                    {#1}
%                                 \end{minipage}};
% \node[fancytitle, right=10pt] at (box.north west) {Conjetura. \thechapter.\theconjetura};
% \node[fancytitle, rounded corners] at (box.east) {$\clubsuit$};
%     \end{tikzpicture}
%              \hspace*{10pt}  
\begin{tcolorbox}[colback=burl!5,colframe=burl!75!black,title=Conjetura. \thechapter.\theconjetura]
  #1
\end{tcolorbox}              }{

                \vskip 5pt
 }%\greenline \vskip 0.1pt}
%%%%%%%%%%%%%%%%%%%%%%%%%%%%%%%%%%%%%%%%%%%%%%%%%%%%%%%%%%%%%%%%%%%%
%%%%%%%%%%%%%%%%%%%%%%%%%%%%%%%%%%%%%% Definicion %%%%%%%%%%%%%%%%%%%%%%%%%%%%%%
\newcounter{definicion}
\newenvironment{definicion}[2]{\vskip 3pt
\stepcounter{definicion}
%  \begin{tikzpicture}
%\tikzstyle{mybox} = [draw=blue, fill=yellow!20, very thick,
%    rectangle, rounded corners, inner sep=20pt, inner ysep=20pt]
%\tikzstyle{fancytitle} =[fill=red!60, text=white, ellipse]
%\stepcounter{definicion}
%\hspace*{-15pt}
%\node [mybox] (box){ %
%     \begin{minipage}[r]{\textwidth}
%                                    {#1}
%                                 \end{minipage}};
% \node[fancytitle, right=10pt] at (box.north west) {Definici\'on. \thechapter.\thedefinicion. \ \bf{#2}};
% \node[fancytitle, rounded corners] at (box.east) {$\heartsuit$};
%     \end{tikzpicture}
%              \hspace*{10pt} 
     \begin{tcolorbox}[colback=caper!5,colframe=caper!75!black,title=Definici\'on. \theechap@cx.\thedefinicion. \ \bf{#2}]
#1
\end{tcolorbox}        
  }{

                \vskip 5pt
 }%\greenline \vskip 0.1pt}
%%%%%%%%%%%%%%%%%%%%%%%%%%%%%%%%%%%%%%%%%%%%%%%%%%%%%%%%%%%%%%%%%%%%
%%%%%%%%%%%%%%%%%%%%%%%%%%%%%%%%%%%%%% Teorema %%%%%%%%%%%%%%%%%%%%%%%%%%%%%%
\newcounter{teorema}
\newenvironment{teorema}[2]{\vskip 5pt
%  \begin{tikzpicture}
%\tikzstyle{mybox} = [draw=blue, fill=red!10, very thick,
%    rectangle, rounded corners, inner sep=10pt, inner ysep=20pt]
%\tikzstyle{fancytitle} =[fill=blue, text=white, ellipse]
\stepcounter{teorema}
%\hspace*{-15pt}
%\node [mybox] (box){ %
%     \begin{minipage}[r]{0.85\textwidth}
%                                    {#1}
%                                 \end{minipage} };
% \node[fancytitle, right=10pt] at (box.north west) {Teorema. \thechapter.\theteorema.  \ \bf{#2}};
% \node[fancytitle, rounded corners] at (box.east) {$\clubsuit$};
%     \end{tikzpicture}
%              \hspace*{10pt}     
\begin{tcolorbox}[colback=wood!5,colframe=wood!75!black,title=Teorema. \theechap@cx.\theteorema. \ \bf{#2}]
#1
\end{tcolorbox}            }{

                \vskip 5pt
 }%\greenline \vskip 0.1pt}
%%%%%%%%%%%%%%%%%%%%%%%%%%%%%%%%%%%%%%%%%%%%%%%%%%%%%%%%%%%%%%%%%%%%
%%%%%%%%%%%%%%%%%%%%%%%%%%%%%%%%%%%%%%%%%%%%%%%%%% construccion%%%%%%%%%%%%%%%%%%%%
\newcounter{construccion}
\newenvironment{construccion}[2]{\vskip 5pt
\stepcounter{construccion}
%\begin{tikzpicture}
%\tikzstyle{mybox} = [draw=red, fill=blue!20, very thick,
%    rectangle, rounded corners, inner sep=10pt, inner ysep=20pt]
%\tikzstyle{fancytitle} =[fill=red, text=black]
%\node [mybox] (box){%
%     \begin{minipage}[r]{\textwidth}
%                                     {#1}
%                                 \end{minipage}};
% \node[fancytitle, right=10pt] at (box.north west) {Construcci\'on. \thechapter.\theconstruccion \ \bf{#2}};
% \node[fancytitle, rounded corners] at (box.east) {$\bigstar$};
%      \end{tikzpicture}
%                        \hspace*{10pt}  
\begin{tcolorbox}[colback=mesh!5,colframe=mesh!75!black,title=Construcci\'on. \theechap@cx .\theconstruccion. \ \bf{#2}]
#1
\end{tcolorbox}        }{

                \vskip 5pt
 }%\greenline \vskip 0.1pt}
%%%%%%%%%%%%%%%%%%%%%%%%%%%%%%%%%%%%%% Ideas  %%%%%%%%%%%%%%%%%%%%%%%%%%%%%%
\newcounter{ideas}
\newenvironment{ideas}[2]{\vskip 5pt
\stepcounter{ideas}
%   \begin{tikzpicture}
%\tikzstyle{mybox} = [draw=blue, fill=yellow!20, very thick,
%    rectangle, rounded corners, inner sep=20pt, inner ysep=20pt]
%\tikzstyle{fancytitle} =[fill=blue, text=white, ellipse]
% \hspace*{-15pt}
%   \node [mybox] (box){ %
% \begin{minipage}[r]{0.9\textwidth}
%                                     #1
%                                          \end{minipage}};
%\node[fancytitle, right=10pt] at (box.north west) {T\'ermino. \thechapter. \theideas. \ \bf{#2}};
%\node[fancytitle, rounded corners] at (box.east) {$\clubsuit$};
% \end{tikzpicture}
\begin{tcolorbox}[colback=Sienna!5,colframe=Sienna!75!black,title=T\'ermino. \theechap@cx .\theideas. \ \bf{#2}]
#1
\end{tcolorbox} 
                         }
                              {\vskip 5pt }
%%%%%%%%%%%%%%%%%%%%%%%%%%%%%%%%%%%%%% NoDefinicion %%%%%%%%%%%%%%%%%%%%%%%%%%%%%%
\newcounter{tndefinido}
\newenvironment{tndefinido}[2]{\vskip 5pt
%  \begin{tikzpicture}
%\tikzstyle{mybox} = [draw=blue, fill=yellow!20, very thick,
%    rectangle, rounded corners, inner sep=20pt, inner ysep=20pt]
%\tikzstyle{fancytitle} =[fill=blue, text=white, ellipse]
\stepcounter{tndefinido}
% \hspace*{-15pt}
%\node [mybox] (box){ %
%     \begin{minipage}[r]{0.9\textwidth}
%                                    #1
%                                 \end{minipage}};
% \node[fancytitle, right=10pt] at (box.north west) {T\'ermino no definido. \thechapter. \thetndefinido. \ \bf{#2}};
% \node[fancytitle, rounded corners] at (box.east) {$\clubsuit$};
%     \end{tikzpicture}
      \begin{tcolorbox}[colback=rhodo!5,colframe=rhodo!75!black,title=T\'ermino no definido. \theechap@cx .\thedefinicion. \ \bf{#2}]
#1
\end{tcolorbox}                        }
                             {\vskip 5pt }%\greenline \vskip 0.1pt}
%%%%%%%%%%%%%%%%%%%%%%%%%%%%%%%%%%%%%%%%%%%%%%%%%%%%%%%%%%%%%
\fontfamily{pbk}
\selectfont
%%%%%%%%%%%%%%%%%%%%%%%%%%%%%% Figura %%%%%%%%%%%%%%%%%%%%%%%%%%
\newenvironment{figura}[3]{\begin{figure}[H]
\centering
                               #1
                              \caption{#2}
                              \label{#3}
                              \end{figure}
}{ \vskip 5pt }
%%%%%%%%%%%%%%%%%%%%%%%%cambiando margen%%%%%%%%%%%%%%%%
%\newenvironment{changemargin}[2]
%{
%\begin{list}{}
%{
%\setlength{\topsep}{0pt}
%\setlength{\evensidemargin}{0pt}%
%\setlength{\oddsidemargin}{0pt}
%\setlength{\leftmargin}{#1}%
%\setlength{\rightmargin}{#2}%
%\setlength{\listparindent}{\parindent}%
%\setlength{\itemindent}{\parindent}%
%\setlength{\parsep}{\parskip}%
%}
%\item[]
%}
%{\end{list}}

%%%%%%%%%%%%%%%%%%%%%%%%%%%%%%%%%% Notas %%%%%%%%%%%%%%%%%%%%%%%%%%%%%%%%%%%%%%%
\newenvironment{notas}[2]{\vskip 5pt  \colorbox{wood!30}{\color{white}{#1:\, #2}}

                                     }{ \vskip 5pt
                                        %\blackline
                                        \vskip 0.2pt}
   %%%%%%%%%%%%%%%%%%%%%%%%%%%%%%%%%%%%% Prueba %%%%%%%%%%%%%%%%%%%%%%%%%%%%%%%
\newenvironment{prueba}[2]{\renewcommand{\arraystretch}{1.06}
\vskip 5pt  {
 \colorbox{teal!30}{\color{white}{Prueba:}}
 }\vskip 5pt
$ %
\begin{tabular}
[t]{ll||l}\hline
\multicolumn{2}{c||}{\mbox{Afirmaciones}} & \mbox{Razones}\\\hline\hline
1. & 
#1 
& \multicolumn{1}{|l}{\mbox{Dado}}\\
#2
\end{tabular}
\ \ \ $ 
\vskip 5pt}{\hfill  \raggedright{ \rule{0.5em}{0.5em} }\vskip 5pt }
 %%%%%%%%%%%%%%%%%%%%%%%%%%%%%%%%%%%%%%%%%%%%%%%%%%%%%%%%%%%%%%%%%%%%%%%%%%%
 %%%%%%%%%%%%%%%%%%%%%%%%%%%%%%%%%%%%% Solucion %%%%%%%%%%%%%%%%%%%%%%%%%%%%%%%
\newenvironment{sol}[2]{\renewcommand{\arraystretch}{2.06}
\newcommand{\tn}{\tabularnewline}
\newcommand{\rr}{\raggedright}
\fontsize{7}{6}\selectfont
\vskip 5pt  {
 \colorbox{teal!30}{\color{white}{Soluci\'on:}}
 }\vskip 5pt
 %
\begin{center}
 \begin{tabular}
[t]{ll||l}\hline
\multicolumn{2}{c||}{Afirmaciones} & Razones\\\hline\hline
1. & #1 & \multicolumn{1}{|l}{Dado}\\
#2
\end{tabular}\end{center}
\ \ \  \vskip 5pt}{\hfill  \raggedright{ \rule{0.5em}{0.5em} }\vskip 5pt }
 %%%%%%%%%%%%%%%%%%%%%%%%%%%%%%%%%%%%%%%%%%%%%%%%%%%%%%%%%%%%%%%%%%%%%%%%%%%
 %%%%%%%%%%%%%%%%%%%%%%%%%%%%%%%%%5Probar %%%%%%%%%%%%%%%%%%%%%
 \newenvironment{probar}[1]
  {\vskip 5pt
   
  \colorbox{teal!30}{\color{white}{Prueba}}\vskip 5pt
 
 #1 
  }
 {\hfill  \raggedright{ \rule{0.5em}{0.5em} }\vskip 5pt }
 %%%%%%%%%%%%%%%%%%%%%%%%%%%%%%%%%%%%%%%%%%%%%%%%%%%%%%%%%%%%



%%%%%%%%%%%%%%%%%%%%%%%%%%%%%%%%%%%%%%%%%%%%%%%%%%%%%%%%%%%%%%%%%%%%%%%%%%%%%%%%%%%
%$$$$$$$$$$$$$$$$$$$$$$$ Comandos $$$$$$$$$$$$$$$$$$$$$$$$$$$$$$$$$$$$$$$$$$
%%%%%%%%%%%%%%%%%%%%%%%%%%%%%%%%%%%%%% solucion %%%%%%%%%%%%%%%%%%%%%%%%%%%%%%%%%%%%%%%%%%
\newcommand{\solucion}
 {\vskip 5pt
  {
 \colorbox{teal!30}{\color{white}{Soluci\'on: \,}}
 }
}
%%%%%%%%%%%%%%%%%%%%%%%%%%%%%%%%%%%%%%%%% problemas %%%%%%%%%%%%%%%%%%%%%%%%%%%
\newcommand{\problema}[1]
{
%\noindent
\section{Problemas}
\begin{multicols}{2}
 #1
\end{multicols}
{\setlength{\parindent}{0mm}\color{black}\rule{\linewidth}{1mm}}
}
%%%%%%%%%%%%%%%%%%%%%%%%%%%% Arco %%%%%%%%%%%%%%%%%%%%%%%%%%%%%%%%%%%%%%%%
\newcommand{\arco}[2][30pt]{\overset{\tikz \draw[rotate=60,line width=1pt]
 (#1,0pt) arc (0:60:#1);}{#2}}
%%%%%%%%%%%%%%%%%%%%%%%%%%%%%%%%%%%%%%%%%%%%%%%%%Nota y solución%%%%%%%%%%%%%%%%%%%%%%%%%%%%%%%%%%%%%%%%%%%%%%%%%%%%%%%%%%%%%%%%
%\newcommand{\solucion}{\colorbox{black}{\color{white}{Soluciön:}}}
\newcommand{\nota}{\colorbox{teal!20!white}{\color{black}{Nota:}}}
\newcommand{\degre}{\ensuremath{^\circ}}
%%%%%%%%%%%%%%%%%%%%%%%%%%% Cierra Introducción de tipos  @ %%%%%%%%%%%%%%%%%%%%%%%%%%%%%%%%
\makeatother
%%%%%%%%%%%%%%%%%%%%%%%%%%%Regla y transportador %%%%%%%%%%%%%%%%%%%%%%%%%%
\newcommand{\regla}[1]{
\begin{center}
\begin{tikzpicture}[scale=#1,font=\fontsize{7}{7}\selectfont]
% La regla de plastico
\draw[thick,red] (-0.25,0.0) rectangle (25.25,2.5);
% Marcas de milímetros
\foreach \x in{0.0, 0.1,...,25.0}
	\draw[blue,shift={(\x,0)}] (0,0.5pt) -- (0,5pt);
% Marcas de centímetros	
\foreach \x in{0, 1, ..., 25}
	\draw[thick,shift={(\x,0)}] (0,-0.5pt) -- (0,7.5pt) node[above]{$\x$};
% Para que escriban su nombre los niños:
%\node[blue,right]	at (0,1.75) {\textbf{Nombre:} \rule{7.5cm}{0.5pt}};
\end{tikzpicture}
\end{center}
}
%%%%%%%%%%%%%%%%%%%%%%%%%%%%%%%%%%%%%%%%%%%%%%%%%%%%%%%%%%%%%%%%%%%%%%%%%%
\newcommand{\transportador}[1]{
\begin{center}
\begin{tikzpicture}[scale=#1,font=\fontsize{7}{7}\selectfont]
\draw[thick] (5,0) arc (0:180:5);
\draw[thick] (5,0) -- (5,-5pt) -- (-5,-5pt) -- (-5,0);
%
\draw[red,fill=red] (0,0) circle (3.5pt);
% Grados
\foreach \ang in {0,1,...,180}
	\draw[rotate=\ang] (4.9,0) -- (5.0,0);
% Multiplos de 5
\foreach \mult in {0, 5,..., 180}
	\draw[thick,rotate=\mult] (4.8,0) -- (5.0,0);
% Multiplos de 10 (Solamente nodos)
\foreach \nodo in {0, 10, ...,180}
	\node[red] at (\nodo:4.5) {\nodo};
% Multiplos de 10 [los últimos]
\foreach \ult in {0, 10, ...,180}
	\draw[very thick,rotate=\ult,red]	(4.8,0) -- (5.0,0);
\end{tikzpicture}
\end{center}
}
%%%%%%%%%%%%%%%%%%%%%%%%%%%%%%%%%%%%%%%%%%%%%%%%%%%%%%%%%%%%%%%%%%%%%%%%
\newcommand{\instruccion}[1]{
	\vspace{\fill}
	\begin{center}
	\cooltooltip
	[1 1 0]
	{Instrucciones:}
	{Imprima en hojas tamaño carta sin lineas. \\
	Utilice sólo regla sin graduar y compás.	
	}
	{http://www.uninorte.edu.co}{Catálogo Web}
	{#1\strut}
	\end{center}
}
%%%%%%%%%%%%%%%%%%%%%%%%%%%%%triangulo    %%%%%%%%%%%%%%%%%%%%%%%%%%%%%%
\newcommand{\triangulo}{\tikz \filldraw[scale=0.5,fill=teal!20] (0,0) -- ++(60:1) -- ++(-60:1) -- cycle ;
                 }
%%%%%%%%%%%%%%%%%%%%%%%%%%%%%%%%%%%%%%%%%%%%%%%%%%%%%%%%%%%%%%%%%%%%%%%%%%