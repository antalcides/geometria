\chapter[
     image=cap3a,%genetics-dogs,
     image caption={The global option open right, triggers the typesetting of chapter on odd pages only. There are a  couple of layouts that must be typeset on an even pages.}]%
         %{1}
{Congruencia y semejanza}
{\begin{itemize}
\item En este cap\'itulo seguiremos con la construci\'on axiomática de la geometr\'ia y paralelamente presentaremos construcciones con regla y c\'ompas, que luego al
final de este cap\'itulo empezaremos a demostrar.
\item Del párrafo anterior se nos presenta una duda, ¿Qué es demostrar? 
\item Podemos considerar la demostración de una proposición $q$ como una cadena finita
de conclusiones que se realizan mediante reglas lógicas y que se forman a partir
de proposiciones verdaderas o supuestamente verdaderas y las cuales nos conducen
a una proposición $p$.
\item La proposición o relación resultante de otras mediante el proceso de la
demostración se llama \texttt{teorema} y este proceso podemos dividirlo
tres partes (figura\ref{demos} ):
\end{itemize}
\begin{lista}
\item  Conocer la proposición que se trata de demostrar. En esta parte es
importante diferenciar muy claramente la información que nos dan (la
hipótesis) de lo que nos solicitan que demostremos (la tesis).
\item Los fundamentos empleados como base de la demostración. Estos fundamentos
   están constituidos por los términos primitivos, las definiciones, los
postulados y las proposiciones o teoremas ya demostrados.
\item  El procedimiento usado para lograr que la proposición quede demostrada
(elegir el método adecuado).
\end{lista}}
\seccion{Congruencia de triágulos}
\subsection{Correspondencia entre triángulos}
\begin{figura}{
\definecolor{zzttqq}{rgb}{0.27,0.27,0.27}
\definecolor{qqqqff}{rgb}{0.33,0.33,0.33}
\begin{tikzpicture}[scale=0.4,font=\fontsize{7}{6}\selectfont,line
cap=round,line join=round,>=triangle 45,x=1.0cm,y=1.0cm]
\clip(-4.13,-7.89) rectangle (16.39,4.53);
\fill[color=zzttqq,fill=zzttqq,fill opacity=0.1] (-3.07,-3.91) -- (-0.41,0.79)
-- (3.41,-4.18) -- cycle;
\fill[color=zzttqq,fill=zzttqq,fill opacity=0.1] (6.97,-3.94) -- (10.58,0.76) --
(13.9,-4.57) -- cycle;
\draw [color=zzttqq] (-3.07,-3.91)-- (-0.41,0.79);
\draw [color=zzttqq] (-0.41,0.79)-- (3.41,-4.18);
\draw [color=zzttqq] (3.41,-4.18)-- (-3.07,-3.91);
\draw [color=zzttqq] (6.97,-3.94)-- (10.58,0.76);
\draw [color=zzttqq] (10.58,0.76)-- (13.9,-4.57);
\draw [color=zzttqq] (13.9,-4.57)-- (6.97,-3.94);
\draw[color=qqqqff,->] (4.73,2.82) -- (4.82,2.82) -- (4.91,2.81)
-- (5,2.8) -- (5.09,2.79) -- (5.18,2.77) -- (5.27,2.76) -- (5.37,2.75) --
(5.46,2.73) -- (5.55,2.72) -- (5.65,2.7) -- (5.74,2.68) -- (5.84,2.66) --
(5.93,2.64) -- (6.03,2.62) -- (6.12,2.6) -- (6.22,2.57) -- (6.32,2.55) --
(6.42,2.53) -- (6.51,2.5) -- (6.61,2.47) -- (6.71,2.44) -- (6.81,2.41) --
(6.91,2.38) -- (7.01,2.35) -- (7.11,2.32) -- (7.21,2.29) -- (7.32,2.25) --
(7.42,2.22) -- (7.52,2.18) -- (7.63,2.14) -- (7.73,2.1) -- (7.83,2.07) --
(7.94,2.02) -- (8.04,1.98) -- (8.15,1.94) -- (8.26,1.9) -- (8.36,1.85) --
(8.47,1.81) -- (8.58,1.76) -- (8.69,1.71) -- (8.79,1.66) -- (8.9,1.61) --
(9.01,1.56) -- (9.12,1.51) -- (9.23,1.46) -- (9.35,1.41) -- (9.46,1.35) --
(9.57,1.3) -- (9.68,1.24) -- (9.79,1.18) -- (9.91,1.12) -- (10.02,1.06) --
(10.14,1) -- (10.25,0.94) -- (10.37,0.88) -- (10.48,0.81) -- (10.54,0.78) --
(10.57,0.77) -- (10.58,0.76) -- (10.58,0.76) -- (10.58,0.76) -- (10.58,0.76) --
(10.58,0.76) -- (10.58,0.76) -- (10.58,0.76) -- (10.58,0.76) --
(10.58,0.76)(-0.41,0.79) -- (-0.41,0.79) -- (-0.41,0.79) -- (-0.41,0.79) --
(-0.41,0.79) -- (-0.41,0.79) -- (-0.41,0.79) -- (-0.41,0.79) -- (-0.41,0.79) --
(-0.41,0.79) -- (-0.41,0.79) -- (-0.41,0.79) -- (-0.41,0.79) -- (-0.41,0.79) --
(-0.41,0.79) -- (-0.41,0.79) -- (-0.41,0.79) -- (-0.4,0.8) -- (-0.39,0.81) --
(-0.37,0.83) -- (-0.34,0.87) -- (-0.28,0.94) -- (-0.22,1) -- (-0.17,1.06) --
(-0.11,1.12) -- (-0.05,1.18) -- (0.01,1.24) -- (0.07,1.29) -- (0.12,1.35) --
(0.18,1.4) -- (0.24,1.46) -- (0.31,1.51) -- (0.37,1.56) -- (0.43,1.61) --
(0.49,1.66) -- (0.55,1.71) -- (0.62,1.76) -- (0.68,1.8) -- (0.75,1.85) --
(0.81,1.89) -- (0.87,1.94) -- (0.94,1.98) -- (1.01,2.02) -- (1.07,2.06) --
(1.14,2.1) -- (1.21,2.14) -- (1.28,2.18) -- (1.34,2.21) -- (1.41,2.25) --
(1.48,2.28) -- (1.55,2.32) -- (1.62,2.35) -- (1.69,2.38) -- (1.77,2.41) --
(1.84,2.44) -- (1.91,2.47) -- (1.98,2.5) -- (2.06,2.52) -- (2.13,2.55) --
(2.2,2.57) -- (2.28,2.6) -- (2.35,2.62) -- (2.43,2.64) -- (2.51,2.66) --
(2.58,2.68) -- (2.66,2.7) -- (2.74,2.71) -- (2.82,2.73) -- (2.89,2.75) --
(2.97,2.76) -- (3.05,2.77) -- (3.13,2.79) -- (3.21,2.8) -- (3.29,2.81) --
(3.38,2.82) -- (3.46,2.82) -- (3.54,2.83) -- (3.62,2.84) -- (3.71,2.84) --
(3.79,2.85) -- (3.87,2.85) -- (3.96,2.85) -- (4.04,2.85) -- (4.13,2.85) --
(4.22,2.85) -- (4.3,2.85) -- (4.39,2.85) -- (4.48,2.84) -- (4.57,2.84) --
(4.66,2.83) -- (4.73,2.82);
\draw[color=qqqqff,->] (5.4,-3.01) -- (5.43,-3.02) --
(5.46,-3.02) -- (5.49,-3.02) -- (5.51,-3.03) -- (5.54,-3.03) -- (5.57,-3.03) --
(5.6,-3.04) -- (5.62,-3.04) -- (5.65,-3.05) -- (5.68,-3.06) -- (5.71,-3.06) --
(5.73,-3.07) -- (5.76,-3.08) -- (5.79,-3.09) -- (5.82,-3.1) -- (5.85,-3.11) --
(5.87,-3.12) -- (5.9,-3.13) -- (5.93,-3.14) -- (5.96,-3.15) -- (5.98,-3.16) --
(6.01,-3.18) -- (6.04,-3.19) -- (6.07,-3.2) -- (6.09,-3.22) -- (6.12,-3.23) --
(6.15,-3.25) -- (6.18,-3.26) -- (6.2,-3.28) -- (6.23,-3.3) -- (6.26,-3.31) --
(6.29,-3.33) -- (6.31,-3.35) -- (6.34,-3.37) -- (6.37,-3.39) -- (6.4,-3.41) --
(6.42,-3.43) -- (6.45,-3.45) -- (6.48,-3.47) -- (6.51,-3.49) -- (6.53,-3.52) --
(6.56,-3.54) -- (6.59,-3.56) -- (6.62,-3.59) -- (6.64,-3.61) -- (6.67,-3.63) --
(6.7,-3.66) -- (6.73,-3.69) -- (6.75,-3.71) -- (6.78,-3.74) -- (6.81,-3.77) --
(6.84,-3.8) -- (6.86,-3.82) -- (6.89,-3.85) -- (6.92,-3.88) -- (6.95,-3.91) --
(6.96,-3.93) -- (6.97,-3.94) -- (6.97,-3.94) -- (6.97,-3.94) -- (6.97,-3.94) --
(6.97,-3.94) -- (6.97,-3.94) -- (6.97,-3.94) -- (6.97,-3.94) -- (6.97,-3.94) --
(6.97,-3.94)(3.41,-4.18) -- (3.41,-4.18) -- (3.41,-4.18) -- (3.41,-4.18) --
(3.41,-4.18) -- (3.41,-4.18) -- (3.41,-4.18) -- (3.41,-4.18) -- (3.41,-4.18) --
(3.41,-4.18) -- (3.41,-4.18) -- (3.41,-4.18) -- (3.41,-4.18) -- (3.41,-4.18) --
(3.41,-4.18) -- (3.41,-4.17) -- (3.41,-4.17) -- (3.42,-4.17) -- (3.42,-4.16) --
(3.43,-4.15) -- (3.45,-4.13) -- (3.48,-4.1) -- (3.51,-4.06) -- (3.53,-4.03) --
(3.56,-4) -- (3.59,-3.97) -- (3.62,-3.94) -- (3.65,-3.91) -- (3.67,-3.88) --
(3.7,-3.85) -- (3.73,-3.82) -- (3.76,-3.79) -- (3.79,-3.76) -- (3.81,-3.73) --
(3.84,-3.71) -- (3.87,-3.68) -- (3.9,-3.66) -- (3.93,-3.63) -- (3.96,-3.61) --
(3.98,-3.58) -- (4.01,-3.56) -- (4.04,-3.53) -- (4.07,-3.51) -- (4.1,-3.49) --
(4.12,-3.47) -- (4.15,-3.45) -- (4.18,-3.42) -- (4.21,-3.4) -- (4.24,-3.38) --
(4.26,-3.37) -- (4.29,-3.35) -- (4.32,-3.33) -- (4.35,-3.31) -- (4.38,-3.29) --
(4.4,-3.28) -- (4.43,-3.26) -- (4.46,-3.25) -- (4.49,-3.23) -- (4.52,-3.22) --
(4.54,-3.2) -- (4.57,-3.19) -- (4.6,-3.17) -- (4.63,-3.16) -- (4.66,-3.15) --
(4.68,-3.14) -- (4.71,-3.13) -- (4.74,-3.12) -- (4.77,-3.11) -- (4.79,-3.1) --
(4.82,-3.09) -- (4.85,-3.08) -- (4.88,-3.07) -- (4.91,-3.06) -- (4.93,-3.06) --
(4.96,-3.05) -- (4.99,-3.04) -- (5.02,-3.04) -- (5.05,-3.03) -- (5.07,-3.03) --
(5.1,-3.02) -- (5.13,-3.02) -- (5.16,-3.02) -- (5.18,-3.02) -- (5.21,-3.01) --
(5.24,-3.01) -- (5.27,-3.01) -- (5.3,-3.01) -- (5.32,-3.01) -- (5.35,-3.01) --
(5.38,-3.01) -- (5.4,-3.01);
\draw[<-,color=qqqqff,->] (7.32,-5.85) -- (7.45,-5.84) --
(7.57,-5.84) -- (7.7,-5.84) -- (7.83,-5.84) -- (7.96,-5.83) -- (8.08,-5.83) --
(8.21,-5.82) -- (8.34,-5.82) -- (8.46,-5.81) -- (8.59,-5.8) -- (8.71,-5.79) --
(8.83,-5.79) -- (8.96,-5.78) -- (9.08,-5.76) -- (9.2,-5.75) -- (9.32,-5.74) --
(9.44,-5.73) -- (9.57,-5.71) -- (9.69,-5.7) -- (9.81,-5.68) -- (9.93,-5.67) --
(10.04,-5.65) -- (10.16,-5.63) -- (10.28,-5.61) -- (10.4,-5.59) -- (10.52,-5.57)
-- (10.63,-5.55) -- (10.75,-5.53) -- (10.86,-5.51) -- (10.98,-5.49) --
(11.09,-5.46) -- (11.21,-5.44) -- (11.32,-5.41) -- (11.44,-5.39) --
(11.55,-5.36) -- (11.66,-5.33) -- (11.77,-5.3) -- (11.89,-5.27) -- (12,-5.24) --
(12.11,-5.21) -- (12.22,-5.18) -- (12.33,-5.15) -- (12.44,-5.11) --
(12.54,-5.08) -- (12.65,-5.04) -- (12.76,-5.01) -- (12.87,-4.97) --
(12.97,-4.94) -- (13.08,-4.9) -- (13.19,-4.86) -- (13.29,-4.82) -- (13.4,-4.78)
-- (13.5,-4.74) -- (13.61,-4.7) -- (13.71,-4.65) -- (13.81,-4.61) --
(13.86,-4.59) -- (13.89,-4.58) -- (13.9,-4.57) -- (13.9,-4.57) -- (13.9,-4.57)
-- (13.9,-4.57) -- (13.9,-4.57) -- (13.9,-4.57) -- (13.9,-4.57) -- (13.9,-4.57)
-- (13.9,-4.57)(-3.07,-3.91) -- (-3.07,-3.91) -- (-3.07,-3.91) -- (-3.07,-3.91)
-- (-3.07,-3.91) -- (-3.07,-3.91) -- (-3.07,-3.91) -- (-3.07,-3.91) --
(-3.07,-3.91) -- (-3.07,-3.91) -- (-3.07,-3.91) -- (-3.07,-3.91) --
(-3.07,-3.91) -- (-3.06,-3.91) -- (-3.06,-3.91) -- (-3.06,-3.91) --
(-3.05,-3.92) -- (-3.04,-3.92) -- (-3.01,-3.93) -- (-2.96,-3.95) --
(-2.85,-3.98) -- (-2.77,-4.01) -- (-2.69,-4.04) -- (-2.61,-4.06) --
(-2.53,-4.09) -- (-2.44,-4.12) -- (-2.36,-4.14) -- (-2.28,-4.17) -- (-2.2,-4.19)
-- (-2.12,-4.22) -- (-2.04,-4.24) -- (-1.96,-4.27) -- (-1.89,-4.29) --
(-1.81,-4.32) -- (-1.73,-4.34) -- (-1.65,-4.36) -- (-1.57,-4.39) --
(-1.49,-4.41) -- (-1.41,-4.44) -- (-1.33,-4.46) -- (-1.25,-4.48) -- (-1.17,-4.5)
-- (-1.1,-4.53) -- (-1.02,-4.55) -- (-0.94,-4.57) -- (-0.86,-4.59) --
(-0.78,-4.62) -- (-0.7,-4.64) -- (-0.63,-4.66) -- (-0.55,-4.68) -- (-0.47,-4.7)
-- (-0.39,-4.72) -- (-0.32,-4.74) -- (-0.24,-4.76) -- (-0.16,-4.78) --
(-0.09,-4.8) -- (-0.01,-4.82) -- (0.07,-4.84) -- (0.14,-4.86) -- (0.22,-4.88) --
(0.3,-4.9) -- (0.37,-4.92) -- (0.45,-4.94) -- (0.53,-4.96) -- (0.6,-4.98) --
(0.68,-5) -- (0.75,-5.01) -- (0.83,-5.03) -- (0.9,-5.05) -- (0.98,-5.07) --
(1.05,-5.08) -- (1.13,-5.1) -- (1.2,-5.12) -- (1.28,-5.13) -- (1.35,-5.15) --
(1.43,-5.17) -- (1.5,-5.18) -- (1.58,-5.2) -- (1.65,-5.22) -- (1.73,-5.23) --
(1.8,-5.25) -- (1.87,-5.26) -- (1.95,-5.28) -- (2.02,-5.29) -- (2.09,-5.31) --
(2.17,-5.32) -- (2.24,-5.33) -- (2.31,-5.35) -- (2.39,-5.36) -- (2.46,-5.38) --
(2.53,-5.39) -- (2.61,-5.4) -- (2.68,-5.42) -- (2.75,-5.43) -- (2.82,-5.44) --
(2.89,-5.45) -- (2.97,-5.47) -- (3.04,-5.48) -- (3.11,-5.49) -- (3.18,-5.5) --
(3.25,-5.51) -- (3.33,-5.52) -- (3.4,-5.53) -- (3.47,-5.55) -- (3.54,-5.56) --
(3.61,-5.57) -- (3.68,-5.58) -- (3.75,-5.59) -- (3.82,-5.6) -- (3.89,-5.61) --
(3.96,-5.62) -- (4.03,-5.63) -- (4.1,-5.63) -- (4.17,-5.64) -- (4.24,-5.65) --
(4.31,-5.66) -- (4.38,-5.67) -- (4.45,-5.68) -- (4.52,-5.69) -- (4.59,-5.69) --
(4.66,-5.7) -- (4.73,-5.71) -- (4.8,-5.72) -- (4.87,-5.72) -- (4.94,-5.73) --
(5.01,-5.74) -- (5.08,-5.74) -- (5.14,-5.75) -- (5.21,-5.75) -- (5.28,-5.76) --
(5.35,-5.77) -- (5.42,-5.77) -- (5.48,-5.78) -- (5.55,-5.78) -- (5.62,-5.79) --
(5.69,-5.79) -- (5.75,-5.8) -- (5.82,-5.8) -- (5.89,-5.8) -- (5.96,-5.81) --
(6.02,-5.81) -- (6.09,-5.82) -- (6.16,-5.82) -- (6.22,-5.82) -- (6.29,-5.82) --
(6.42,-5.83) -- (6.55,-5.83) -- (6.69,-5.84) -- (6.82,-5.84) -- (6.95,-5.84) --
(7.08,-5.85) -- (7.21,-5.85) -- (7.32,-5.85);
\fill [color=qqqqff] (-3.07,-3.91) circle (1.5pt);
\draw[color=qqqqff] (-3.36,-3.91) node {$A$};
\fill [color=qqqqff] (-0.41,0.79) circle (1.5pt);
\draw[color=qqqqff] (-0.44,1.21) node {$B$};
\fill [color=qqqqff] (3.41,-4.18) circle (1.5pt);
\draw[color=qqqqff] (3.81,-4.26) node {$C$};
\fill [color=qqqqff] (6.97,-3.94) circle (1.5pt);
\draw[color=qqqqff] (6.49,-3.96) node {$D$};
\fill [color=qqqqff] (10.58,0.76) circle (1.5pt);
\draw[color=qqqqff] (10.79,1.11) node {$E$};
\fill [color=qqqqff] (13.9,-4.57) circle (1.5pt);
\draw[color=qqqqff] (14.08,-4.23) node {$F$};
\end{tikzpicture}}{Correspondencia entre tri\'angulos}{corres}
\end{figura}
Dados dos tri\'angulos como se muestra en la figura(\ref{corres}) siempre
podemos establecer una correspondencia entre los v\'ertices, como por ejemplo:\\
Podemos establecer $A\longleftrightarrow F , C \longleftrightarrow D$ y
$B\longleftrightarrow E$.
\begin{definicion}{La correspondencia establecida entre los v\'ertices
de de dos tri\'angulos determina tres parejas de elementos del tri\'angulo
llamados partes correspondientes. }{Partes correspondientes}
 La correspondencia establecida en la figura \ref{corres} determina las
siguientes partes correspondientes.\\
$\angle A \longleftrightarrow \angle F, \angle B \longleftrightarrow \angle E,
\angle C \longleftrightarrow \angle D, \overline{AB}\longleftrightarrow
\overline{FE}, \overline{BC}\longleftrightarrow
\overline{ED}$ y $\overline{AC}\longleftrightarrow
\overline{DF}.$
\end{definicion}
\begin{definicion}{Dos tri\'angulos son congruentes si y s\'olo si existe una
correspondencia entre los tr\'angulos de tal forma que sus partes
correspodientes, sean congruentes.
}{Tri\'angulos congruentes}

\end{definicion}
\subsection{Postulados de congruencia}
\begin{postulado}{Si dos lados y el \'angulo comprendido de un tri\'angulo
son respectivamente congruentes con dos lados y un \'angulo comprendido de
otro tri\'angulo, entonces los dos tri\'angulos son congruentes.}{LAL}
\end{postulado}
\subsection{Postulados de congruencia}
\begin{postulado}{Si dos \'angulos y el lado del tri\'angulo que es com\'un,
son respectivamente congruentes con dos \'angulos y el lado comprendido de
otro tri\'angulo, entonces los tri\'angulos son congruentes.}{ALA}
\end{postulado}
\subsection{Postulados de congruencia}
\begin{postulado}{Si Los tres lados de un tri\'angulo son respectivamente
congruentes con tres lados de otro tri\'angulos, los tri\'angulos son
congruentes.}{LAL}
\end{postulado}
\begin{teorema}{
En un tri\'angulo is\'oseles, la mediana al lado de la base forma dos
tri\'angulos congruentes. punto medio 
}{Tri\'angulo is\'osseles}
\end{teorema}

\begin{teorema}{
Si en un tri\'angulo , dos \'angulos y un lado opuesto a uno de los \'angulos
son congruentes con dos \'angulos y el lado correspondiente de un segundo
tri\'angulo, entonces los tri\'angulos son congruentes.
}{LAA}
\end{teorema}

\begin{teorema}{
Si la hip\'otenusa y un \'angulo agudo de un tri\'angulo rect\'angulo son
congruentes con la hipotenusa y un \'angulo agudo de otro tri\'angulo
recta\'angulo, entonces los tri\'angulos son congruentes,
}{Hipotenusa y \'angulo}
\end{teorema}

\begin{teorema}{si la hipotenusa y un cateto de un tri\'angulo rect\'angulo son
congruentes con la hipotenusa y un cateto de otro tri\'angulo, entonces los
tri\'angulos son congruentes.
}{hipotenusa y el cateto}
\end{teorema}
\begin{teorema}{
Si un punto equidista de los extremos de segmento, entonces pertenece a la
mediatriz del segmento.
}{Mediatriz}
 \end{teorema}
\begin{teorema}{
El circuncentro equidista de los v\'ertices del tri\'angulo.
}{circuncentro}
\end{teorema}
\begin{teorema}{
El incentro de un tri\'angulo equidista de los lados del tri\'angulo
}{Incentro}
\end{teorema}
\begin{teorema}{
El centroide de un tri\'angulo se encuentra a dos tercios de la longitud de
cada mediana. 
}{Centroide}
 
\end{teorema}
\begin{teorema}{
Si las medidas de dos \'angulos de un tri\'angulo son diferentes, entonces la
longitud del lado opuesto al \'angulo mayor, es mayor que la longitud del
lado opuesto al \'angulo menor.
}{desigualdad triangular}
\end{teorema}
\begin{conjetura}{Un tri\'angulo es
is\'osceles si y s\'olo si los
\'angulos de la base son congruentes.}{Tri\'angulo is\'osceles}
 \end{conjetura}
\seccion{Proporciones}
\begin{definicion}{ Si $a,b,c,d \in I\! \! R $
 tal que se cumple $\dfrac{a}{b}=\dfrac{c}{d}, $ cuando $c,d\neq 0$, entonces
se dice que $a,b,c$ y $d$ son proporcionales.
}{Proporciones}
 \end{definicion}
    \subsection{Propiedades de las proporciones}
\begin{lista}
\item Si $\dfrac{a}{b}=\dfrac{c}{d}, $ entonces $ad=bc.$
\item Si $\dfrac{a}{b}=\dfrac{c}{d}, $ entonces $\dfrac{a+b}{b}=\dfrac{c+d}{d}.$
\item Si $\dfrac{a}{b}=\dfrac{c}{d}, $ entonces $\dfrac{a-b}{b}=\dfrac{c-d}{d}.
$
\item Si $\dfrac{a}{b}=\dfrac{c}{d}, $ entonces $\dfrac{a}{c}=\dfrac{b}{d}. $
\item Si $ad=bc, $ entonces $\dfrac{a}{b}=\dfrac{c}{d}. $
\seccion{Teorema fundamental de la proporcionalidad}
\end{lista}
\begin{definicion}{
Cuatro segmentos son proporcionales si y s\'olo si sus longitudes son
proporcionales
}{Segmentos proporcionales}
 
\end{definicion}

\begin{teorema}{ Si una recta paralela a un lado de un tri\'angulo interseca a
los otros dos lados, entonces divide a \'estos dos en segmentos proporcionales
}{Teorema fundamental}
\begin{figura}{
\definecolor{ccqqtt}{rgb}{0.8,0,0.2}
\definecolor{ttttff}{rgb}{0.2,0.2,1}
\definecolor{uququq}{rgb}{0.25,0.25,0.25}
\definecolor{xdxdff}{rgb}{0.49,0.49,1}
\definecolor{qqqqff}{rgb}{0,0,1}
\begin{tikzpicture}[line cap=round,line join=round,>=triangle
45,x=1.0cm,y=1.0cm]
\clip(-1.2,-4.82) rectangle (8.26,3.98);
\draw (0.24,-2.68)-- (4.3,3.22);
\draw (4.3,3.22)-- (7.3,-3.24);
\draw (7.3,-3.24)-- (0.24,-2.68);
\draw [domain=-1.2:8.26] plot(\x,{(-3.84--0.56*\x)/-7.06});
\draw [color=ttttff] (4.3,3.22)-- (2.33,0.36);
\draw [color=ccqqtt] (2.33,0.36)-- (0.24,-2.68);
\draw [color=ttttff] (4.3,3.22)-- (5.75,0.09);
\draw [color=ccqqtt] (5.75,0.09)-- (7.3,-3.24);
\draw (0.06,0.54) node[anchor=north west] {$l$};
\fill [color=qqqqff] (4.3,3.22) circle (1.5pt);
\draw[color=qqqqff] (4.4,3.64) node {$A$};
\fill [color=qqqqff] (0.24,-2.68) circle (1.5pt);
\draw[color=qqqqff] (-0.16,-2.98) node {$B$};
\fill [color=qqqqff] (7.3,-3.24) circle (1.5pt);
\draw[color=qqqqff] (7.52,-3.42) node {$C$};
\fill [color=xdxdff] (2.33,0.36) circle (1.5pt);
\draw[color=xdxdff] (1.84,0.14) node {$D$};
\fill [color=uququq] (5.75,0.09) circle (1.5pt);
\draw[color=uququq] (5.98,0.56) node {$E$};
\draw[color=ttttff] (3.12,2.12) node {$e$};
\draw[color=ccqqtt] (1.06,-0.82) node {$f$};
\draw[color=ttttff] (5.18,2.04) node {$g$};
\draw[color=ccqqtt] (6.78,-1.52) node {$h$};
\end{tikzpicture}
}{Proporci\'on}{propr}
 
\end{figura}
\end{teorema}
\begin{teorema}{Si una recta interseca a dos lados de un tri\'angulo y los
divide proporcionalmente, entonces la recta es paralela al tercer lado.
}{Rec\'iproco del teorema fundamental}
\end{teorema}
\seccion{Pol\'igonos semejantes}
\begin{definicion}{Dos pol\'igonos son semejantes si hay una correspondencia
entre los v\'ertices tal que los \'angulos correspondientes sean congruentes y
los lados correspondientes sean proporcionales
}{Pol\'igonos semejantes}
\end{definicion}
\subsection{Tri\'angulos semejantes}
\begin{postulado}{
Dos tri\'angulos son semejantes si existe una correspondencia entre ellos
tal que Dos parejas de a\'ngulos correspondientes son congruentes.}{A.A}
\end{postulado}
\begin{teorema}{Dos tri\'angulos son semejantes si y s\'olo si al establecer
una correspondencia entre ellos los \'angulos correspondientes son
congruentes.}{A.A.A}
 \end{teorema}
\begin{teorema}{Dos tri\'angulos son semejantes si y s\'olo si al establecer
una correspondencia entre los dos tri\'angulos sus lados correspondientes son
proporcionales }{L.L.L}
\end{teorema}
\begin{teorema}{Si un \'angulo de un tri\'angulo es congruente con un \'angulo
de otro tri\'angulo, y si los lados correspondientes que forman los \'angulos,
son proporcionales, entonces los tri\'angulos son semejantes.
}{L.A.L}
\end{teorema}
\subsection{Semejanza en tri\'angulos rect\'angulos}
\begin{definicion}{
Se dice que $x \in I\! \! R $ es media geom\'etrica de $a,b \in I\! \! R$ si y
s\'olo si se cumple $\dfrac{x}{a}=\dfrac{b}{x}, a,x \neq 0$
}{Media geom\'etrica}
\end{definicion}

\begin{teorema}{En un tri\'angulo rect\'angulo, la longitud de la altura a la
hipotenusa es la media geom\'etrica entre las longitudes de los dos segmentos
de la hip\'otenusa.}{Media geom\'etrica}
\end{teorema}
\begin{teorema}{Dados un tri\'angulo rect\'angulo y la altura a la
hipotenusa, cada cateto es la media geom\'etrica entre la longitud de la
hip\'otenusa y la longitud del segmento de la hipotenusa adyacente al
cateto}{Altura-hipotenusa}
\end{teorema}



\seccion{Paralelogramos}
\begin{definicion}{Un cuadril\'atero es un pol\'igono de cuatro
lados}{Cuadril\'atero}
\end{definicion}
\begin{definicion}{Un trapecio es un cuadril\'atero que tiene exactamente dos
lados paralelos}{Trapecio}
\begin{figura}{\definecolor{xdxdff}{rgb}{0.49,0.49,1}
\definecolor{qqqqff}{rgb}{0,0,1}
\begin{tikzpicture}[scale=0.8,font=\fontsize{9}{8}\selectfont,line
cap=round,line join=round,>=triangle 45,x=1.0cm,y=1.0cm,line width=1.2pt]
\clip(-1.07,-1.25) rectangle (5.06,2.06);
\draw (-0.91,-0.76)-- (-0.21,1.43);
\draw (-0.21,1.43)-- (3.48,1.39);
\draw (3.48,1.39)-- (4.77,-0.81);
\draw (4.77,-0.81)-- (-0.91,-0.76);
\draw (1.46,1.50) node[anchor=north west] {3.69};
\draw (1.7,-0.650) node[anchor=north west] {5.69};
\fill [color=qqqqff] (-0.21,1.43) circle (1.5pt);
\draw[color=qqqqff] (-0.40,1.55) node {$A$};
\fill [color=qqqqff] (3.48,1.39) circle (1.5pt);
\draw[color=qqqqff] (3.55,1.7) node {$B$};
\fill [color=xdxdff] (-0.91,-0.76) circle (1.5pt);
\draw[color=xdxdff] (-0.9,-1.1) node {$D$};
\fill [color=xdxdff] (4.77,-0.81) circle (1.5pt);
\draw[color=xdxdff] (4.85,-1.1) node {$E$};
\end{tikzpicture}}{Trapecio}{tra2}
\nota: Los lados paralelos de trapecio se llaman bases.
\end{figura}
\end{definicion}
\begin{construccion}{Para construir un trapecio, se construye dos segmentos
paralelos,  como se explic\'o en las construcci\'on de
paraleleas y luego se construye el cuadril\'atero. }{Trapecio}
\begin{figura}{
\definecolor{qqwuqq}{rgb}{0,0.39,0}
\definecolor{xdxdff}{rgb}{0.49,0.49,1}
\definecolor{ffttww}{rgb}{1,0.2,0.4}
\definecolor{ffzzqq}{rgb}{1,0.6,0}
\definecolor{ttttff}{rgb}{0.2,0.2,1}
\definecolor{uququq}{rgb}{0.25,0.25,0.25}
\definecolor{cccccc}{rgb}{0.8,0.8,0.8}
\definecolor{qqqqff}{rgb}{0,0,1}
\begin{tikzpicture}[scale=0.8,font=\fontsize{7}{6}\selectfont,line
cap=round,line
join=round,>=triangle 45,x=1.0cm,y=1.0cm,line width=1.2pt]
\clip(-3.55,-2.62) rectangle (6.25,6.56);
\draw [shift={(1,2)},color=qqwuqq,fill=qqwuqq,fill opacity=0.1] (0,0) --
(0.09:0.35) arc (0.09:63.43:0.35) -- cycle;
\draw [shift={(0,0)},color=qqwuqq,fill=qqwuqq,fill opacity=0.1] (0,0) --
(0:0.35) arc (0:63.43:0.35) -- cycle;
\draw [line width=1.6pt] (1,2)-- (0,0);
\draw (0,0)-- (4,0);
\draw [line width=1.6pt,dash pattern=on 1pt off
1pt,color=cccccc,fill=cccccc,fill opacity=0.25] (0,0) circle (2.24cm);
\draw [line width=1.6pt,dash pattern=on 3pt off 3pt,color=ttttff] (1,2)--
(2.24,0);
\draw [line width=1.6pt,dash pattern=on 3pt off 3pt,color=ffzzqq] (0,0)--
(2.24,0);
\draw [line width=1.6pt,dash pattern=on 3pt off
3pt,color=ffttww,fill=ffttww,fill opacity=0.05] (2.24,0) circle (2.35cm);
\draw [line width=1.2pt,dash pattern=on 1pt off
1pt,color=cccccc,fill=cccccc,fill opacity=0.15] (1,2) circle (2.24cm);
\draw [line width=1.6pt,dash pattern=on 3pt off 3pt,color=ffzzqq] (0,0)-- (2,4);
\draw [line width=1.6pt,dash pattern=on 1pt off
1pt,color=ffttww,fill=ffttww,fill opacity=0.05] (2,4) circle (2.35cm);
\draw [line width=1.6pt,dash pattern=on 3pt off 3pt,color=ttttff] (2,4)--
(3.24,2);
\draw [line width=1.6pt,dash pattern=on 3pt off 3pt,color=ffzzqq] (1,2)--
(-0.34,3.79);
\draw [line width=1.6pt,dash pattern=on 3pt off 3pt,color=ffzzqq] (1,2)-- (6,2);
\draw [line width=1.6pt,dash pattern=on 3pt off 3pt] (-3,0)-- (6,0);
\fill [color=qqqqff] (0,0) circle (1.5pt);
\draw[color=qqqqff] (0.04,-0.13) node {$A$};
\fill [color=qqqqff] (4,0) circle (1.5pt);
\draw[color=qqqqff] (4.1,0.15) node {$B$};
\fill [color=qqqqff] (1,2) circle (1.5pt);
\draw[color=qqqqff] (0.98,1.73) node {$C$};
\draw[color=cccccc] (-1.09,1.76) node {$c$};
\fill [color=uququq] (2.24,0) circle (1.5pt);
\draw[color=uququq] (2.33,-0.11) node {$D$};
\draw[color=ttttff] (1.49,0.99) node {$e$};
\draw[color=ffzzqq] (0.69,0.74) node {$f$};
\fill [color=xdxdff] (2,4) circle (1.5pt);
\draw[color=xdxdff] (1.95,4.23) node {$E$};
\fill [color=uququq] (-0.34,3.79) circle (1.5pt);
\fill [color=xdxdff] (3.24,2) circle (1.5pt);
\draw[color=xdxdff] (3.3,1.81) node {$H$};
\draw (3.3,1.81)-- (4.1,0.15);
\end{tikzpicture}}{Trapecio}{Tparll}
\end{figura}
En la figura(\ref{Tparll}) se observa el trapecio $ABHC$.
\end{construccion}

\begin{definicion}{Un paralelogramos es un cuadril\'atero que tiene sus lados
opuestos paralelos}{Paralelogramo}
\begin{figura}{\definecolor{uququq}{rgb}{0.25,0.25,0.25}
\definecolor{qqqqff}{rgb}{0,0,1}
\begin{tikzpicture}[line cap=round,line join=round,>=triangle
45,x=1.0cm,y=1.0cm]
\clip(1.4,-3.26) rectangle (10.18,1.4);
\draw (2.14,-2.04)-- (7.36,-2.04);
\draw (3.78,0.12)-- (2.14,-2.04);
\draw (3.78,0.12)-- (9,0.12);
\draw (9,0.12)-- (7.36,-2.04);
\fill [color=qqqqff] (2.14,-2.04) circle (1.5pt);
\draw[color=qqqqff] (1.9,-2.08) node {$A$};
\fill [color=qqqqff] (7.36,-2.04) circle (1.5pt);
\draw[color=qqqqff] (7.28,-2.32) node {$B$};
\fill [color=qqqqff] (3.78,0.12) circle (1.5pt);
\draw[color=qqqqff] (3.94,0.38) node {$C$};
\fill [color=uququq] (9,0.12) circle (1.5pt);
\draw[color=uququq] (9.16,0.38) node {$D$};
\end{tikzpicture}}{Paralelogramo}{parreg}
\end{figura}

\end{definicion}
\begin{definicion}{Un rombo es un paralelogramo que tiene sus lados
congruentes}{Rombo}
\end{definicion}
\begin{definicion}{Un rect\'angulo es un paralelogramo que tiene dos lados
contiguos perpendiculares.}{Rect\'angulo}
\end{definicion}
\begin{definicion}{El cuadrado es un paralelogramo que es rombo y
rect\'angulo a la vez}{Cuadrado}
\begin{figura}{
\definecolor{qqwuqq}{rgb}{0,0.39,0}
\definecolor{uququq}{rgb}{0.25,0.25,0.25}
\definecolor{zzttqq}{rgb}{0.6,0.2,0}
\definecolor{qqqqff}{rgb}{0,0,1}
\begin{tikzpicture}[scale=0.8,font=\fontsize{7}{6}\selectfont,line
cap=round,line join=round,>=triangle 45,x=1.0cm,y=1.0cm]
\clip(2.22,-4.1) rectangle (7.44,0.74);
\fill[color=zzttqq,fill=zzttqq,fill opacity=0.1] (4.04,0) -- (3.04,-2.2) --
(5.24,-3.2) -- (6.24,-1) -- cycle;
\draw[color=qqwuqq,fill=qqwuqq,fill opacity=0.1] (5.42,-2.81) -- (5.03,-2.64) --
(4.85,-3.02) -- (5.24,-3.2) -- cycle;
\draw [color=zzttqq] (4.04,0)-- (3.04,-2.2);
\draw [color=zzttqq] (3.04,-2.2)-- (5.24,-3.2);
\draw [color=zzttqq] (5.24,-3.2)-- (6.24,-1);
\draw [color=zzttqq] (6.24,-1)-- (4.04,0);
\fill [color=qqqqff] (4.04,0) circle (1.5pt);
\draw[color=qqqqff] (4.2,0.26) node {$A$};
\fill [color=qqqqff] (3.04,-2.2) circle (1.5pt);
\draw[color=qqqqff] (2.78,-2.1) node {$B$};
\fill [color=uququq] (5.24,-3.2) circle (1.5pt);
\draw[color=uququq] (5.24,-3.48) node {$C$};
\fill [color=uququq] (6.24,-1) circle (1.5pt);
\draw[color=uququq] (6.4,-0.74) node {$D$};
\draw[color=qqwuqq] (5.22,-2.3) node {$\beta = 90\textrm{\degre}$};
\end{tikzpicture}}{Cuadrado}{cua}
\end{figura}
 \end{definicion}
\begin{teorema}{
Los \'angulos opuestos de un paralelogramo son congruentes.
}{LOPC}

\end{teorema}
\begin{teorema}{
Los \'angulos opuestos de un paralelogramos son congruentes
}{AOPC}

\end{teorema}

\begin{teorema}{
Los pares de \'angulos adyacentes de un paralelogramo son \'angulos
suplementarios.
}{PAAPC}

\end{teorema}
\begin{teorema}{
Si los lados opuestos de un cuadri\'atero son congruentes, entonces el
cuadril\'atero es un paralelogramo
}{RLOPC}

\end{teorema}

\begin{teorema}{
Si los \'angulos opuestos de un cuadril\'atero son congruentes, entonces el
cuadril\'atero es un paralelogramo
}{RPAAPC}
\end{teorema}
\begin{teorema}{
El segmento que une los puntos medios de dos lados de un tri\'angulo es
paralelo al tercer lado y tiene la mitad de su longitud
}{Segmento medio}
\end{teorema}

\begin{teorema}{
Los puntos medios de los lados de un cuadril\'atero son v\'ertices de un
paralelogramp
}{PMCDP}

\end{teorema}
\begin{teorema}{
Un paralelogramo es un rect\'angulo si, y s\'olo si, sus diagonales son
congruentes,
}{Diagonales de un paralelogramo}

\end{teorema}
\begin{teorema}{
Un paralelogramo es un rombo si, y s\'olo si sus diagonales son perpendiculares
entre s\'i.
}{ERSDP}
\end{teorema}

\begin{teorema}{
Un paralelogramo es un rombo si, y s\'olo si, cada diagonal biseca a un par de
\'angulos opuestos
}{ERDBA}

\end{teorema}
\begin{teorema}{
El segmento que une los puntos medios de dos lados no paralelos de un trapecio
es paralelo a las dos bases y tiene una longitud igual a la semisuma de las
longitudes de las bases.
}{Segmento medio del trapecio}

\end{teorema}

\begin{teorema}{
En un trapecio con sus lados no paralelos congruentes, los \'angulos de la base
y las diagonales son congruentes
}{trapecio is\'osceles}

\end{teorema}

\begin{teorema}{
La suma de las medidas de los \'angulos exteriores de un pol\'igono, en cada 
uno de sus v\'ertices es, $360^\circ$.
}{}

\end{teorema}
 \begin{teorema}{
La suma de los \'angulos de un pol\'igono convexo de $n$ lados es
$(n-2)180^\circ$
}{}

\end{teorema}      

